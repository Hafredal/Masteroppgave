\section{Kvalitativ analyse}
\label{section:kvalitativ_analyse} 

\subsection{Stegvis-deduktiv induktiv metode}
\citet{Tjora} beskriver stegvis-deduktiv induktiv metode som en metode for analyse
av kvalitative data. Denne metoden er induktiv i den forstand at man jobber fra
data mot teori, samtidig er den stegvis-deduktiv da man sjekker teorien opp mot det
empiriske. Denne tilnærmingen ble fulgt ved bearbeiding av datamaterialet.

\noindent
Feltnotatene fra observasjonene ble transkribert, og deretter kodet og kategorisert ved bruk av dataprogrammet RStudio \citep{Rstudio}. Kodingen av analysedataene resulterte i nærmere 100 koder, og forskerne jobbet på denne måten induktivt med materialet. Det høye antallet koder forklares ved at forskerne genererte detaljerte, \textit{tekstnære}, koder. Tekstnær koding er oppnådd dersom kodene ikke kunne vært definerte før datainnsamlingen, og er dermed generert fra data, og ikke fra forskningsspørsmål eller planlagte temaer. For å luke ut empiri som ikke var relevant for forskningsområdet ble kodene kategorisert i 11 kategorier. Kategorisering innebærer å sortere og samle koder relevante for forskningsspørsmålene i grupper. På denne måten avgjør forskningsområdet, og ikke empirien hva som er relevant \citep{Tjora}. I denne studien ble ikke forskningsspørsmålene formulert før etter den innledende observasjonsperioden. Kategoriseringen var dermed svært nyttig da forskningsområdet skulle avgrenses, slik at forskerne kunne plukke ut de temaene de ønsket å ta med videre. Arbeidet videre kan sies å ha vært stegvis-deduktivt, da funnene til en viss grad ble forsøkt beskrevet av teori, og det ble formulert forskningsspørsmål på bakgrunn av disse. Da forskningsområdet i andre observasjonsperiode var avgrenset, ble mengden feltnotater noe mindre. Analysedataene fra denne perioden ble også kodet, men forskerne kunne i stor grad gjenbruke koder fra tidligere. Etter observasjonsperioden utpekte det seg sentrale temaer som videre formet intervjuguidene. Lydopptakene fra intervjuene ble transkribert og brukt som støtte til observasjonsdataene, for å utdype, sammenligne og avdekke sprik mellom forskernes og informantenes oppfatninger. Avslutningsvis var arbeidet i større grad deduktivt, da forskerne forsøkte å beskrive avdekkede funn med relevant teori.

\subsection{Datamaterialets kvalitet}

