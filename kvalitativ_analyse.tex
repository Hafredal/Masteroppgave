\section{Datamaterialets kvalitet}
\label{kvalitativ_analyse}
Kriteriene pålitelighet, gyldighet og generaliserbarhet benyttes ofte som indikatorer på datamaterialets kvalitet \citep{Tjora}, og forskerne vil i denne delen redegjøre for forhold som kan ha påvirket resultatene. 

\subsubsection{Pålitelighet}
Pålitelighet omhandler hvorvidt forskerne selv kan ha preget forskningsarbeidet. Forskernes kunnskap og forutinntattheter om tematikken som studeres, og eventuelle personlige relasjoner til informantene kan påvirke forskningen, og det er viktig å gjøre rede for slike interne forhold \citep{Tjora}. \citet{Tjora} understreker at fullstendig nøytralitet er umulig, og at forskernes kunnskap om emnet er en ressurs så fremt det eksplisitt gjøres rede for hvordan denne er brukt i analysen. Triangulering, som innebærer å sammenligne funn, har vært sentralt i arbeidet, hvor både triangulering av forskere, data og metode for datainnsamling er blitt benyttet \citep{Oates}. 

\noindent
Med bakgrunn i sin prosjektoppgave hadde forskerne mye kunnskap og visse forutinntattheter om hvordan pasientsignalsystemet var tenkt brukt, og hvilke utfordringer som eksisterer. Dette kan derimot hevdes å ha vært en ressurs da det ble gjort funn som til en viss grad ikke stemte overens med tidligere funn. Da tidligere arbeid i stor grad har fokusert på endringer i funksjonaliteten til IP-telefonen, var det overraskende at sykepleierne interagerte med denne i så liten grad ved innkommende pasientsignal. Dette medførte at forskningsområdet ble endret fra videre fordypning i endring av funksjonalitet, til å omhandle temaer som forskerne ikke hadde forutsett. Deler av det teoretiske materialet er hentet fra \citep{Sund13}, men forskerne har vært bevisste på å ikke la tidligere arbeid styre analysen av datamaterialet. Arbeidet kan derfor sies å ha blitt utført i tråd med SDI-metoden, hvor målsetningen er en større empirisk forankring hvor man lar empirien forme forskningen i de første stadiene, og teorien i de siste \citep{Tjora}. 

\noindent
Teoretisk materiale ble i stor grad innhentet ved bruk av Googles søkemotor for akademisk litteratur, Google Scholar. Bøker og artikler publisert i tidsskrifter har hovedsakelig blitt brukt som kilder da disse anses å være pålitelige. Der det er brukt elektroniske kilder (nettsider) er dette dokumenter for spesifikke fakta, samt brukerveiledninger for pasientsignalsystemet funnet via St. Olavs Hospitals egne hjemmesider.

\noindent
Når det gjelder valg av avdelinger for observasjon og videre intervjuer, ble disse valgt av praktiske årsaker. Da veileder for oppgaven tidligere hadde observert ved avdelingene anså forskerne det som lettere å få tilgang til disse. I tillegg ble de vurdert som ulike nok i fysisk utforming og pleieoppgaver, til at det kunne være mulig å potensielt avdekke ulik arbeidspraksis. Funnene avdekket blant annet et sentralt skille mellom avdeling A1 som ikke bruker telefonen for mottak av pasientsignal, og avdelingene A2 og A3. Hvorvidt funnene kan sies å være representative for hele sykehuset er uvisst, men de la til rette for diskusjon omkring de ulikhetene som ble avdekket, og hvorfor disse har oppstått. 

\noindent
Da intervjuene ble holdt i sykepleiernes arbeidstid hadde forskerne i liten grad anledning til å stille kriterier til intervjuobjektene. Seksjonslederne sto dermed fritt til å velge informanter selv og de kan ha hatt ulike motiver til hvorfor nettopp disse ble valgt. 

\noindent
Videre kan man ved kvalitative studier spørre seg om resultatene ville blitt de samme dersom en annen forsker gjorde den samme jobben \citep{Tjora}. Ved observasjon er det  vanskelig å garantere at en forsker observerer det samme som det en annen ville gjort. Da to forskere observerte både alene og sammen over flere dager og på ulike tidspunkt, mener forskerne å ha bidratt til å styrke datamaterialets kvalitet. I tillegg ble feltnotater og lydopptak transkribert for å kunne gjengi direkte sitater fra informantene.

\subsubsection{Generalisering}
Spørsmålet om hvorvidt generalisering er nødvendig innen kvalitativ forskning er blitt diskutert over lengre tid \citep{Tjora}. Mens noen hevder at kvalitativ forskning ikke har til hensikt å generalisere, men heller å avdekke særegenheter innenfor den gitte kontekst \citep{Creswell, Oates}, mener andre at dette er en veletablert kvalitetsindikator som også har sin plass i det kvalitative \citep{Tjora}. \citet{Tjora} presenterer begrepet konseptuell generalisering, hvor hensikten er å fremstille funn som ikke er spesifikke for en gitt case, men å forklare disse i lys av tidligere forskning og teori, og på en slik måte støtte opp under større gyldighet og generaliserbarhet. Dette tilsvarer et av de siste stegene i SDI-analysen, og forskningsarbeidet har dermed hatt til hensikt å redegjøre for at funnene i denne oppgaven kan generaliseres på et konseptuelt nivå. 

\noindent
For å kunne avdekke eventuelle sammenhenger mellom erfaring, kjønn og alder ble informantene etterspurt slik informasjon. Da forskerne i tillegg stilte intervjupregede spørsmål under observasjonene ble det innsamlede datamaterialet nyansert. Utvalget kan dermed antas å være representativt for hver enkelt avdeling. 

- få informanter
	- ulike holdninger på A1
	 hvem
- kort periode 

\subsubsection{Gyldighet}
Gyldighet knyttes til spørsmålet om hvorvidt avdekke funn faktisk svarer på forskningsspørsmålene som er blitt stilt. Dette kan styrkes med åpenhet om hvordan forskningen er gjennomført og begrunnelser for de valgene som er tatt med tanke på metoder for datagenerering og teoretiske innspill til analysen. 

- informantenes erfaring

- forskereffekt

- valg av metode for datagenerering
	- ingen kvantitative mål