\section{Datamaterialets kvalitet}
\label{kvalitativ_analyse}
Kriteriene \textit{pålitelighet}, \textit{gyldighet} og \textit{generaliserbarhet} benyttes ofte som indikatorer på datamaterialets kvalitet \citep{Tjora}.

\subsubsection{Pålitelighet}
Pålitelighet omhandler hvorvidt forskerne selv kan ha preget forskningsarbeidet. Forskernes kunnskap og forutinntattheter om tematikken som studeres, og eventuelle personlige relasjoner til informantene kan påvirke forskningsarbeidet, og det er viktig å gjøre rede for slike interne forhold \citep{Tjora}. \citet{Tjora} understreker at fullstendig nøytralitet er umulig, og at forskernes kunnskap om emnet er en ressurs så fremt det eksplisitt gjøres rede for hvordan denne er brukt i analysen. Triangulering, som innebærer å sammenligne funn, har vært sentralt i arbeidet, hvor både triangulering av forskere, data og metode for datainnsamling er blitt benyttet \citep{Oates}. 

\noindent
Med bakgrunn i sin prosjektoppgave hadde forskerne mye kunnskap og visse forutinntattheter om hvordan pasientsignalsystemet var tenkt brukt, og hvilke utfordringer som eksisterer. Dette kan derimot hevdes å ha vært en ressurs, da det ble gjort nye funn som til en viss grad ikke stemte overens med tidligere funn. Da tidligere arbeid i stor grad har fokusert på endringer i funksjonaliteten til IP-telefonen, var det overraskende at sykepleierne interragerte med denne i så liten grad ved innkommende pasientsignal. Dette medførte at forskningsområdet ble endret fra videre fordypning i endring av funksjonalitet, til å omhandle temaer som forskerne ikke hadde forutsett. Arbeidet kan derfor sies å ha blitt utført i tråd med SDI-metoden, hvor målsetningen er en større empirisk forankring \citep{Tjora}.

- valg av teori

- valg av informanter/avdelinger

- ærlige?

- rollen som observatør/intervjuer



\noindent
Videre kan man ved kvalitative studier spørre seg om resultatene ville blitt de samme dersom en annen forsker gjorde den samme jobben \citep{Tjora}. Ved observasjon er det  vanskelig å garantere at en forsker observerer det samme som det en annen ville gjort. Da to forskere observerte både alene og sammen over flere dager, på ulike tidspunkt, mener forskerne å ha bidratt til å styrke datamaterialets kvalitet. 


I tillegg ble feltnotater og lydopptak transkribert, for å kunne gjengi direkte sitater fra informantene.



\subsubsection{Generalisering}
Spørsmålet om hvorvidt generalisering er nødvendig innen kvalitativ forskning er blitt diskutert over lengre tid \citep{Tjora}. Mens noen hevder at kvalitativ forskning ikke har til hensikt å generalisere, men heller å avdekke særegenheter innenfor den gitte kontekst \citep{Creswell, Oates}, mener andre at dette er en veletablert kvalitetsindikator som også har sin plass i det kvalitative \citep{Tjora}. \citet{Tjora} presenterer begrepet \textit{konseptuell generalisering}, hvor hensikten er å fremstille funn som ikke er spesifikke for en gitt case, men å forklare disse i lys av tidligere forskning og teori, og på en slik måte støtte opp under større gyldighet og generaliserbarhet. Dette tilsvarer et av de siste stegene i SDI-analysen, og forskningsarbeidet har dermed hatt til hensikt å redegjøre for at funnene i denne oppgaven kan generaliseres på et konseptuelt nivå. 

- få informanter/ hvem
- kort periode 

\subsubsection{Gyldighet}
- informantenes erfaring/ trekk frem JA

- forskereffekt

- valg av metode for datagenerering
	- ingen kvantitative mål