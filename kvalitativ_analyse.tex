\section{Kvalitativ analyse}
\label{section:kvalitativ_analyse} 

Analyse av kvalitative data betegner en prosess hvor forskerne forsøker å forstå de empiriske dataene som er samlet inn.


\subsection{Stegvis-deduktiv induktiv metode}
Analysen av datamaterialet er blitt utført etter en \textit{stegvis-deduktiv induktiv (SDI)} metodisk tilnærming, se figur \ref{SDI}. \citet{Tjora} beskriver SDI-metoden som \textit{$"$en skjematisk modell for kvalitativ forskning, hvor grunnprinsippet er en induktiv utvikling fra empiri til konsepter eller teorier, med deduktive trinnvise tilbakekoblinger. Målet er konseptutvikling og kvalitetssikring. $"$} De oppadgående pilene viser den induktive prosessen, hvor forskningen er empiridrevet, mens de nedadgående pilene viser det deduktive arbeidet, hvor forskningen i større grad er teoridrevet.

\begin{figure}[H]
\centering
\includegraphics[scale=0.1]{SDI.jpg}
\caption{Stegvis-deduktiv induktiv metode \citep{Tjora})}
\label{SDI}
\end{figure}

\noindent
Feltnotatene fra første observasjonsperiode ble transkribert, og deretter kodet og kategorisert ved bruk av dataprogrammet RStudio \citep{Rstudio}. Kodingen av observasjonsdataene resulterte i nærmere 100 koder, og forskerne jobbet på denne måten induktivt med materialet. Det høye antallet koder forklares ved at forskerne genererte detaljerte, \textit{tekstnære} koder fra en stor mengde data \citep{Tjora}. For å kunne luke ut empiri som ikke var relevant for videre forskning ble kodene kategorisert i 11 kategorier. Dette ga forskerne en strukturert oversikt over forskningsområdet, og de formulerte forskningsspørsmål som det videre arbeidet søkte svar på. Forskningsarbeidet beveget seg dermed fra konseptutviklingsfasen til en ny runde med generering av empiriske data. Feltnotatene fra andre observasjonsperiode ble kodet og kategorisert med nærmere 50 koder og 5 kategorier. Etter observasjonsperioden utpekte det seg dermed sentrale temaer som videre formet intervjuguidene. Lydopptakene fra intervjuene ble transkribert og brukt som støtte til observasjonsdataene for å utdype, sammenligne og avdekke sprik mellom forskernes og informantenes oppfatninger. Avslutningsvis forsøkte forskerne å konseptualisere og diskutere avdekkede funn i lys av relevant teori. 

\subsection{Datamaterialets kvalitet}
Kriteriene \textit{pålitelighet}, \textit{gyldighet} og \textit{generaliserbarhet} benyttes ofte som indikatorer på datamaterialets kvalitet \citep{Tjora}.

\subsubsection{Pålitelighet}
Pålitelighet omhandler hvorvidt forskerne selv kan ha preget forskningsarbeidet. Forskernes kunnskap og forutinntattheter om tematikken som studeres, og eventuelle personlige relasjoner til informantene kan påvirke forskningsarbeidet, og det er viktig å gjøre rede for slike interne forhold \citep{Tjora}. \citet{Tjora} understreker at fullstendig nøytralitet er umulig, og at forskernes kunnskap om emnet er en ressurs så fremt det eksplisitt gjøres rede for hvordan denne er brukt i analysen. Triangulering, som innebærer å sammenligne funn, har vært sentralt i arbeidet, hvor både triangulering av forskere, data og metode for datainnsamling er blitt benyttet \citep{Oates}. 

\noindent
Med bakgrunn i sin prosjektoppgave hadde forskerne mye kunnskap og visse forutinntattheter om hvordan pasientsignalsystemet var tenkt brukt, og hvilke utfordringer som eksisterer. Dette kan derimot hevdes å ha vært en ressurs, da det ble gjort nye funn som til en viss grad ikke stemte overens med tidligere funn. Da tidligere arbeid i stor grad har fokusert på endringer i funksjonaliteten til IP-telefonen, var det overraskende at sykepleierne interragerte med denne i så liten grad ved innkommende pasientsignal. Dette medførte at forskningsområdet ble endret fra videre fordypning i endring av funksjonalitet, til å omhandle temaer som forskerne ikke hadde forutsett. Arbeidet kan derfor sies å ha blitt utført i tråd med SDI-metoden, hvor målsetningen er en større empirisk forankring \citep{Tjora}.

\noindent
Videre kan man ved kvalitative studier spørre seg om resultatene ville blitt de samme dersom en annen forsker gjorde den samme jobben \citep{Tjora}. Ved observasjon er det  vanskelig å garantere at en forsker observerer det samme som det en annen ville gjort. Da to forskere observerte både alene og sammen over flere dager, på ulike tidspunkt, mener forskerne å ha bidratt til å styrke datamaterialets kvalitet. 


I tillegg ble feltnotater og lydopptak transkribert, for å kunne gjengi direkte sitater fra informantene.



\subsubsection{Generalisering}
Spørsmålet om hvorvidt generalisering er nødvendig innen kvalitativ forskning er blitt diskutert over lengre tid \citep{Tjora}. Mens noen hevder at kvalitativ forskning ikke har til hensikt å generalisere, men heller å avdekke særegenheter innenfor den gitte kontekst \citep{Creswell, Oates}, mener andre at dette er en veletablert kvalitetsindikator som også har sin plass i det kvalitative \citep{Tjora}. \citet{Tjora} presenterer begrepet \textit{konseptuell generalisering}, hvor hensikten er å fremstille funn som ikke er spesifikke for en gitt case, men å forklare disse i lys av tidligere forskning og teori, og på en slik måte støtte opp under større gyldighet og generaliserbarhet. Dette tilsvarer et av de siste stegene i SDI-analysen, og forskningsarbeidet har dermed hatt til hensikt å redegjøre for at funnene i denne oppgaven kan generaliseres på et konseptuelt nivå. 

\subsubsection{Gyldighet}
