\chapter{Diskusjon}
\label{chp:diskusjon}

Vi vil her forsøke å forstå hvilke faktorer som kan være årsak til forskjellene identifisert i kapittel \ref{chp:resultater}, gjennom å analysere disse i lys av teorien som er presentert i kapittel \ref{chp:teori}. 

Tidligere arbeid foreslår i stor grad endringer i teknologi som løsning på utfordringene ved pasientsignalsystemet. Forskerne oppdaget imidlertid gjennom sine observasjoner at problemene ikke nødvendigvis ligger i selve teknologien, men i andre faktorer. 

- ISTA som sier at likt system ofte gir helt ulik bruk

\section{Tilpasning mellom teknologi og oppgave}	
Pleierne fortalte at IP-telefonen ofte brukes for å komme i direkte kontakt med andre pleiere. Dette gir en mer effektiv arbeidshverdag da de tidligere brukte mye tid på å lete etter hverandre. Pleierne anser telefonen i slike tilfeller som et nyttig verktøy, og har akseptert og integrert den i sin arbeidshverdag (jf. \ref{sec:implementering}). Videre kan man si at teknologien har påvirket det sosiale systemet, da pleierne bruker mindre tid på å kommunisere ansikt-til-ansikt (jf. \ref{section:ista-rammeverket}). Det ble derimot understreket som problematisk at pasientsignaler varsles i telefonsamtaler, da dette oppleves som svært forstyrrende. Dette skaper både frustrasjon hos pleierne og forsinker deres arbeid. Dette er i tråd med \citet{Grundgeiger09} som påpeker at avbrudd kan gi negative effekter. Det kan samtidig argumenteres for at denne varslingen har en positiv effekt da signalet oppfattes, og det ville vært en risiko dersom alle aktuelle pleiere snakket i telefonen uten å bli varslet. Systemet er designet slik at pasientsignal skal varsles uansett, og dette er dermed en tilsiktet funksjon. Dette medfører at pleiere i roller hvor de bruker telefonen mye til samtaler enten velger å ikke motta pasientsignaler, går med to telefoner, eller bruker tuntelefonen for utgående samtaler. Dette er eksempler på workarounds, da pleierne finner løsninger på det de opplever som en sperre i systemet (jf. \ref{{section:sosioteknisk}}). Disse tilpasningene av teknologien kan i tråd med \citet{Coiera07} sees på som implisitte signaler på en svakhet i systemet, og kan tvinge frem en endring i teknologien (jf. \ref{section:ista-rammeverket}). Det er planlagt en endring hvor pasientsignaler ikke varsles under telefonsamtaler, men heller sendes videre til neste mottaker. Dette er en form for avbruddshåndtering som forebygger avbrytelser ved å blokkere innkommende signaler \citep{Grandhi10}.
	
\noindent
Generelt opplever pleierne at å motta pasientsignaler på telefonen medfører flere avbrytelser i arbeidshverdagen. Varslingen generer støy som i flere situasjoner oppleves som svært forstyrrende. Mange oppgaver krever pleiernes fulle oppmerksomhet, og avbrytelser i form av pasientsignal kan være uheldig dersom deres kognitive kapasitet overskrides, da dette kan hemme oppmerksomheten og føre til feil og ineffektivitet \citep{Ebright10, Parker00}. Det oppstår dermed en avveining mellom hva som er viktigst, å fullføre en oppgave uten forstyrrelser, eller å bli varslet om pasientsignaler. Under observasjonene så forskerne aldri at en sykepleier forlot et pasientrom for å besvare et annet pasientsignal, men at de ofte avbrøt oppgaver som ikke involverte pasienter direkte. Dualiteten ved disse avbruddene gjør det utfordrende å tilpasse systemet til sykepleiernes oppgaver, da de både gir uttrykk for at de ønsker å bli varslet, samtidig som de ønsker å ha fullt fokus på pasientene. Dette medfører også at systemet brukes ulikt, og annerledes enn tenkt. Data presentert av \citet{Rygh13} viser at pasientene ikke blir forstyrret i like stor grad som pleierne tror, og at pasientene heller ikke opplever at pleierne forstyrres. Både \citet{Rygh13} og egne funn tyder derimot på at pasientene vegrer seg for å utløse signaler dersom de tror at pleierne har mye å gjøre, og at de dermed likevel påvirkes av varslingene. For å unngå å forstyrre pasienter som sover, velger noen pleiere å legge igjen telefonen utenfor pasientrommet om natten, fordi lyden på telefonen ikke kan skrues av.

\noindent
For avdeling A1 har varslingen på telefon vært et så stort problem at de ikke ønsker å bruke telefonene til dette. En manglende tilpasning mellom teknologi og oppgave er svært fremtredende i situasjoner hvor pleiere på avdeling A1 står i stell, da de av hygieniske årsaker ikke har anledning til å ta telefonen opp av lommen. Pleierne kan i slike tilfeller ikke avvise signalet, noe som fører til mye støy da signalet varsles helt til det sendes videre. Da dette er en situasjon som ofte oppstår har pleierne på denne avdelingen problemer med å se nødvendigheten av slik bruk da de uansett ikke kan ta opp telefonen, samtidig som de får varslingen på veggpaneler. Bruk av veggpaneler og telefon for mottak av pasientsignaler gir en redundans av data, og dermed også en sikkerhet for at signalene blir oppfattet. Ved å ikke benytte telefonen mister avdeling A1 denne sikkerheten. Pleiere ved de to andre avdelingene fortalte at selv om de i utgangspunktet mottar signaler på telefonene, glemmer de iblant å logge seg på. Ved å ikke være pålogget telefonen for mottak av pasientsignaler forsvinner redundansen av data, og i tilfeller hvor pleierne ikke tilstedemarkerer seg på pasientrommet vil de være helt isolert fra omgivelsene. Dette kan utgjøre en risiko for at pasienter ikke får hjelp når de trenger det.

\noindent
Sykepleierne samarbeider på tvers av sengetunene, men teknologien er kun til en viss grad tilpasset dette da utformingen av sløyfer ikke støtter slikt samarbeid fullstendig. En signifikant forskjell på avdeling A1 og de to andre, er at pleiere på A1 ikke er avhengige av å bruke telefonen for å motta signaler fra andre tun, da hele avdelingen ligger på samme sløyfe og alle signaler varsles på veggpanelene. Avdelingene A2 og A3 er derimot på grunn av sløyfenes utforming nødt til å bruke telefonen.

\noindent	
Tilpasningen mellom teknologi og oppgave vil avhenge av sykepleiernes pleiemodell, da denne kan sees på som en måte å organisere oppgaver på (jf. \ref{sec:pleie}). Pasientsignaler varsles først hos primæransvarlig for pasienten, noe som vitner om at systemet først og fremst er tilpasset primærsykepleiemodellen. Forskerne observerte derimot ingen tydelig pleiemodell hos de tre avdelingene. Dagskiftene bar preg av å være organisert etter en primærsykepleiemodell, mens på kvelds- og nattskift var oppgavene i større grad organisert etter teamsykepleiemodellen. Dette kan da bety at det er bedre tilpasning mellom teknologien og oppgavene på dagskift, enn på kveld og natt. Dette fører til at sykepleierne, som observert, logger seg på systemet med ulike roller avhengig av skift. Dette kan anses som en workaround dersom hensikten er at pleierne skal være pålogget med primæransvar for pasientene.

		
\section{Tilpasning mellom teknologi og individ}
Forskerne observerte derimot ikke at signaler ble avvist på andre avdelinger heller.

- har vært fokus på å markere seg, men la ikke stort merke til det

- kultur for å la det ringe lenge /interagerte lite med telefonene

- frustrasjon over å bli avbrutt
	- stressende

- infeksjon klager på lyd, men vil ikke bruke det uansett
	- ser ikke nytten
	
	 telefonene skal være "system nr. 1", men blir nr. 2 pga forsinkelsen,
	 
- noen bruker kanskje mer fordi de "må" - ortopedi	 
	 
	 infeksjon visste at det var en prøveperiode, kanskje ikke såå positive andre gangen.

\section{Tilpasning mellom individ og oppgave}

- flere snakker om det med vane.
 - noe er en rutine for noen



- Alle er opptatt av at systemet må TILPASSES

SVAR PÅ FORSKNINGSSPØRSMÅL!!