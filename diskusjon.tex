\chapter{Diskusjon}
\label{chp:diskusjon}

Vi vil her forsøke å forstå hvilke faktorer som kan være årsak til forskjellene identifisert i kapittel \ref{chp:resultater}, gjennom å analysere disse i lys av teorien som er presentert i kapittel \ref{chp:teori}. 



Teknologi
Tidligere arbeid foreslår i stor grad endringer i teknologi som løsning på utfordringene ved pasientsignalsystemet. Forskerne oppdaget imidlertid gjennom sine observasjoner at problemene ikke nødvendigvis ligger i selve teknologien, men i andre faktorer.

 telefonene skal være "system nr. 1", men blir nr. 2 pga forsinkelsen,
 
 
 - ulike sløyfer, hjerte og ortopedi er helt avhengige av telefonen for å motta stans.	
	- noen bruker kanskje mer fordi de "må" - ortopedi
	
- infeksjon klager på lyd, men vil ikke bruke det uansett
	- ser ikke nytten
	
	
	
INDIVID
infeksjon visste at det var en prøveperiode, kanskje ikke såå positive andre gangen.

- flere snakker om det med vane.


I tilfeller hvor pleierne vurderer denne som lav er det større sannsynlighet for at signalet ringer lenger, noe som genererer mer støy.

- Alle er opptatt av at systemet må TILPASSES