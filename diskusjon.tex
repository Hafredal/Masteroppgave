\chapter{Diskusjon}
\label{chp:diskusjon}
Kompleksiteten i det sosiotekniske samspillet kan føre til at et identisk system blir brukt svært forskjellig, noe som er tydelig i sykepleiernes bruk av pasientsignalsystemet. Som understreket i både FITT- og ISTA-rammeverket får teknologien ofte skylden for eventuelle problemer ved innføring av et system. Tidligere arbeid foreslår i stor grad endringer i teknologi som løsning på utfordringene ved pasientsignalsystemet. Forskerne oppdaget imidlertid gjennom sine observasjoner at problemene ikke nødvendigvis ligger i selve teknologien, men i andre faktorer. 

\noindent
Der kapittel \ref{chp:resultater} ser på objektene i FITT-rammeverket isolert, vil dette kapittelet se på tilpasningen mellom disse i lys av presentert teori for å besvare forskningsspørsmålene:

\begin{enumerate}
\item Hvordan brukes pasientsignalsystemet ved St.Olavs Hospital forskjellig i, og mellom ulike avdelinger? 
\item Hvilke faktorer kan være årsak til disse forskjellene?
\end{enumerate}

\section{Tilpasning mellom teknologi og oppgave}	

\subsubsection{Pleiemodell}	
I tråd med \citet{Orlikowski92} vil aksept og integrasjon av et nytt system påvirkes av organisasjonens strukturelle elementer, og dersom teknologien ikke er tilpasset disse vil den sannsynligvis ikke skape effektiv samhandling uten at strukturen endres. Sykepleiernes pleiemodell kan anses som et strukturelt element, da denne er en måte å organisere oppgaver på, og tilpasningen mellom teknologi og oppgave vil derfor avhenge av denne (jf. \ref{sec:pleie}).

\noindent
Pasientsignaler varsles først hos primæransvarlig for pasienten, noe som vitner om at systemet først og fremst er tilpasset en primærsykepleiemodell. Forskerne observerte derimot ingen tydelig pleiemodell hos de tre avdelingene. På dagskift ga pleierne ved alle avdelingene uttrykk for at de forsøker å besvare signaler fra pasientene de har primæransvar for, og observasjonene viste også at pleierne på avdelingene A2 og A3 logget på systemet som primæransvarlig for de enkelte pasienter. Det ble ikke observert at pleierne forlot pasientrom for å besvare andre pasientsignaler, noe som kan tyde på at pleierne stoler på at andre besvarer signalene, og at avdelingene har en arbeidspraksis som bærer preg av både primærsykepleie og teamsykepleie. Avdelingene er dermed organisert etter det som kan betegnes som en modulær sykepleiemodell, hvor pleierne har ansvar for den totale omsorgen og distribuerer oppgaver innenfor sengetunet (jf. \ref{sec:strukturelle_elementer}).

\noindent
På kvelds- og nattskift var oppgavene i større grad organisert etter en teamsykepleiemodell, noe pleierne selv bekreftet, hvor de i større grad delte på ansvaret for pasientsignalene og ofte tok ansvar for ulike oppgaver, som middagsservering og medisinering. Forskerne observerte også at pleierne normalt logget seg på systemet som disp på hele sengetunet og ikke som primær for de enkelte pasientene på kvelds- og nattskift. Dersom hensikten er at pleierne skal være pålogget med primæransvar for pasientene kan det anses som en workaround at pleierne velger å gjøre noe annet som bedre passer deres arbeidspraksis.

\noindent
Det er tydelig at avdelingenes strukturelle elementer påvirker pleiernes adopsjon og bruk av systemet. Avvik fra tenkt bruk kan forklares ved manglende samsvar mellom avdelingenes arbeidspraksis og systemets design, noe som også underbygges i ISTA-rammeverket (jf. \ref{sec:ista-rammeverket}). 

\subsubsection{Varsling under telefonsamtaler}
Pleierne fortalte at IP-telefonen ofte brukes for å komme i direkte kontakt med andre pleiere. Dette gir en mer effektiv arbeidshverdag da de tidligere brukte mye tid på å lete etter hverandre. Pleierne anser telefonen i slike tilfeller som et nyttig verktøy og har akseptert og integrert den i sin arbeidshverdag (jf. \ref{sec:implementering}). Videre kan man si at teknologien har påvirket det sosiale systemet da pleierne bruker mindre tid på å kommunisere ansikt-til-ansikt (jf. \ref{sec:ista-rammeverket}). 

\noindent
Systemet er designet slik at pasientsignal skal varsles uansett, og dette er dermed en tilsiktet funksjon. Det oppleves derimot som svært forstyrrende at pasientsignaler varsles under telefonsamtaler, og disse avbrytelsene medfører negative effekter som frustrasjon og forsinkelse i pleiernes arbeid \citep{Grundgeiger09}. Det kan samtidig argumenteres for at denne varslingen har en positiv effekt da signalet oppfattes, og det ville utgjort en risiko dersom alle aktuelle pleiere snakket i telefonen uten å bli varslet. Dette resulterer i at pleiere i roller hvor de ringer mye enten velger å ikke motta pasientsignaler, går med to telefoner eller bruker fasttelefonen på tunet for utgående samtaler. Det oppstår dermed workarounds, hvor pleierne finner løsninger på det de opplever som en sperre i systemet (jf. \ref{section:sosioteknisk}). Disse tilpasningene av teknologien kan i tråd med \citet{Coiera07} sees på som implisitte signaler om en svakhet i systemet og et behov for endring. Det er planlagt en endring hvor pasientsignaler ikke varsles under telefonsamtaler, men heller sendes videre til neste mottaker. Dette er en form for avbruddshåndtering som forebygger avbrytelser ved å blokkere innkommende signaler \citep{Grandhi10}. Her er det tydelig at pleiernes lokale tilpasninger har tvunget frem en endring i teknologien (jf. \ref{sec:ista-rammeverket}).
	
\subsubsection{Avbrytelser}
Generelt opplever pleierne at å motta pasientsignaler på telefonen medfører flere avbrytelser i arbeidshverdagen. Varslingen generer støy som i flere situasjoner oppleves som svært forstyrrende. Mange oppgaver krever pleiernes fulle oppmerksomhet, og avbrytelser i form av pasientsignal kan være uheldig dersom deres kognitive kapasitet overskrides. Dette kan hemme oppmerksomheten og føre til feil og ineffektivitet \citep{Ebright10, Parker00}. For avdeling A1 har varslingen på telefon vært et så stort problem at de ikke ønsker å bruke telefonene til dette. En manglende tilpasning mellom teknologi og oppgave er svært fremtredende i situasjoner hvor pleiere på avdeling A1 står i stell, da de av hygieniske årsaker ikke har anledning til å interagere med telefonen. Pleierne kan i slike tilfeller ikke avvise signalet, noe som fører til mye støy da signalet varsles helt til det sendes videre. Da dette er en situasjon som ofte oppstår har pleierne på denne avdelingen problemer med å se nødvendigheten av slik bruk da de uansett ikke kan ta opp telefonen, samtidig som de får varslingen fra alle tun på veggpaneler.

\noindent
Det ble ikke observert at sykepleierne forlot et pasientrom for å besvare et annet pasientsignal, men at de ofte avbrøt oppgaver som ikke involverte pasienter direkte, eksempelvis journalføring. Dette er i tråd med \citet{klemets13} som viser at sykepleiernes håndtering av pasientsignaler avhenger av kontekst. Om natten ønsker ikke pleierne å forstyrre sovende pasienter, samtidig som de ønsker å bli varslet om pasientsignaler og hasteanrop. For å løse dette velger noen pleiere å legge igjen telefonen utenfor pasientrommet. Dette er et tydelig eksempel på en uønsket konsekvens som gir en arbeidspraksis som i verste fall kan være en risiko for pasientsikkerhet dersom signaler ikke oppfattes. En teknisk løsning kunne vært at telefonen automatisk blir satt til lydløs når pleieren går inn på pasientrommet. Problemet med begge løsningene er imidlertid at det oppstår en risiko for at signaler ikke oppfattes, noe som kan ha negativ innvirkning på pasientsikkerheten. Til tross for at sykepleierne uttrykker forskjellige behov avhengig av tid på døgnet er de opptatt av at pasientsikkerheten må ivaretas. Det oppstår dermed en avveining mellom hva som er viktigst, å fullføre en oppgave uten forstyrrelser, eller å bli varslet om pasientsignaler. Denne dualiteten gjør tilpasningen svært vanskelig da sykepleiernes behov stadig endres avhengig av kontekst og arbeidsoppgave, som videre fører til at pleierne bruker systemet ulikt, og ofte annerledes enn slik det er tenkt. 

\subsubsection{Utforming av teknisk system}
Å levere pasientsignaler gjennom to systemer, det faste og det trådløse, er et eksempel på redundans av data (jf. \ref{sec:redundans}). Det kan dermed argumenteres for at det i tiden før signalene varsles på telefon ikke eksisterer slik redundans. Samtidig kan det argumenteres for at det faste systemet i seg selv gir redundans av data da varslingen skjer på flere veggpaneler. 

\noindent
Sykepleierne samarbeider i stor grad på tvers av sengetunene og har dermed behov for å motta signaler fra alle disse. Ved avdeling A1 er det faste systemet godt tilpasset dette, da alle tunene ligger på samme sløyfe. For denne avdelingene kan det derfor argumenteres for at det finnes redundans av data da alle signaler varsles på alle panel, noe pleierne antageligvis anser som tilstrekkelig da de velger å ikke motta signalene på telefon. Ved de to andre avdelingene er derimot tunene fordelt på flere sløyfer, og det eksisterer dermed ikke full redundans av data i det faste systemet alene. Pleierne ved A2 og A3 anser det derfor som nødvendig å motta signaler på telefon for å oppnå slik redundans. 

\noindent
Avdeling A3 ønsker å motta pasientsignaler fra tun i etasjen under på kvelds- og nattskift, men ikke på dagtid. Som forklart av IKT-rådgiver ved sykehuset er sløyfene forsøkt tilpasset sykepleiernes behov, men da det er ressurskrevende å endre disse har avdelingene måttet tilpasse sin bruk av systemet etter dette. For å motta alle ønskede signaler må de derfor logge seg på som disp på tun fra andre sløyfer. Det kan dermed argumenteres for at det faste systemet er for rigid i forhold til sykepleiernes arbeidspraksis. I tråd med \citep{Ackermann00}, oppstår det nye normer for bruk som bidrar til å gjøre systemet mer fleksibelt. Det er dermed en klar sammenheng mellom sløyfenes utforming og sykepleiernes bruk av systemet.

\section{Tilpasning mellom teknologi og individ}

\subsubsection{Veggpanel som foretrukket kilde til informasjon}
Det er tydelig at veggpanelene brukes oftere enn telefonen som kilde til informasjon om utløste pasientsignaler. Det er hovedsakelig to årsaker til dette. For det første fører forsinkelsen i det trådløse systemet til at signalene først varsles på veggpanelene. Til tross for at \citet{Sletten09} hevder at denne forsinkelsen har minimal betydning for sengetunene viser derimot egne funn at denne kan være kritisk, spesielt ved utløste hasteanrop da pleierne ønsker å motta disse umiddelbart. I tillegg kan forsinkelsen føre til redundans av innsats i tilfeller hvor flere pleiere går for å besvare samme signal, og en unødvendig avbrytelse i arbeidet til de pleierne som blir overflødige. For det andre vil det ofte være enklere å se på panelene, og i noen tilfeller er det problematisk å ta telefonen opp av lommen.  Dette er i tråd med \citet{klemets13} som påpeker at det i visse situasjoner er utfordrende å bruke telefonen som kilde til informasjon. Dette medfører at pleierne i liten grad interagerer med telefonen, og heller ikke bruker funksjonen for å godta og avvise signaler. Dette gir mer støy som et resultat av at systemet ikke brukes slik det er tenkt.

\noindent
Flere pleiere ønsker seg større paneler da det fra avstand kan være vanskelig å lese hva som står. På avdeling A1 ønsker de også store paneler i taket. Dette antyder at teknologien ikke er tilpasset den fysiske settingen hvor den er tatt i bruk (jf. \ref{sec:ista-rammeverket}). Det er dermed overraskende at pleierne ikke bruker telefonene i større grad for å løse dette problemet, spesielt ved avdeling A1 hvor dette etterspørres av flere. At pleierne gir uttrykk for de har et behov som ikke er dekket kan bety at de ikke har forstått det nye systemets egenskaper, da dette problemet kunne vært løst ved å oftere bruke telefonen som kilde til informasjon (jf. \ref{sec:implementering}). Fra utviklernes side kan panelene ha blitt laget så små fordi de forventet mer bruk av telefonene, og dermed ikke forutså at sykepleierne ikke alltid ville ha mulighet til å interagere med denne og derfor ville ha behov for større panel. 

\subsubsection{Assistanseknapp}
Avdeling A1 er eneste avdeling med mulighet til å utløse assistansesignal. Alternativet for de to andre avdelingene er å utløse et hasteanrop. I opplæringsdokumentene til sykehuset er hasteanropet beskrevet som et signal som kan utløses ved behov for assistanse. Det ble derimot avdekket en sterk felles oppfatning av at hasteanropet kun brukes ved nødsituasjoner. Sykepleierne uttrykte at de synes det er problematisk å utløse hasteanrop ved mindre alvorlige situasjoner, og det ble observert at pleierne ved A2 og A3 heller stikker hodet ut i gangen for å be om hjelp. Det har dermed oppstått en workaround fordi fortolkningen av systemet skiller seg fra tenkt bruk (jf. \ref{sec:implementering}, \ref{sec:ista-rammeverket}). Det er dermed tydelig at sykepleierne har behov for både et assistansesignal og et signal beregnet for nødsituasjoner.
  
\subsubsection{Støy}
Varslingene av pasientsignaler medfører mye støy i pleiernes arbeidshverdag, noe som skaper stress og frustrasjon. Sykepleierne uttrykker blant annet stor misnøye med høyt volum og at varslingen fortsetter på telefonene etter at signalet er blitt avstilt. At noen pleiere for eksempel teiper over høyttaleren på telefonen for å dempe lyden vitner om at tilpasningen mellom teknologi og individ ikke er optimal. Tidligere kunne lyden skrues av, men denne muligheten ble fjernet etter ønske fra sykepleierne da det utgjorde en risiko for at signaler ikke ble oppfattet. I dag er det derimot flere pleiere som ønsker tilbake denne muligheten samtidig som de ser positive effekter av å bli varslet. Dualiteten ved disse avbruddene gjør det utfordrende å tilpasse systemet til sykepleiernes oppgaver, da de både gir uttrykk for at de ønsker å bli varslet samtidig som de ønsker å ha fullt fokus på pasientene. 

\noindent
At sykepleierne sjelden forlater pasientrom for å besvare andre signaler kan forklare hvorfor det ble observert at signalene ofte ringer lenger ved avdeling A1, da de ofte utfører oppgaver som på grunn av hygienerutiner er tidkrevende å påbegynne og avslutte. For det andre gir assisterende seksjonsleder ved avdelingen sterkt utttrykk for at de ikke ser noen gevinst, men kun ulemper ved å bruke telefonen til mottak av pasientsignal. Disse ulempene er hovedsakelig støy- og forstyrrelsesrelaterte. Pleierne på avdelingen har en oppfatning av at andre avdelinger i stor grad bruker funksjonene for å godta og avvise signaler for å redusere støynivået, og argumenterer med at dette i mange situasjoner ikke er mulig for dem. De påpeker også at de har en pasientgruppe som utløser mange signaler. Obervasjonene avdekket imidlertid ikke tydelige forskjeller i antall pasientsignaler på de forskjellige avdeligene, og som tidligere nevnt lite bruk av funksjonene for å godta og avvise signaler.

\noindent 
Til tross for at støy er et hovedargument for hvorfor avdeling A1 ikke ønsker å benytte telefonen for mottak av pasientsignaler uttrykte pleierne likevel at de ikke nødvendigvis vil ta den i bruk selv om det blir gjort endringer i lyden. Det er hovedsakelig to årsaker til dette. De ser for det første ikke nødvendigheten av å motta pasientsignaler på telefonen da de likevel får varslingene på veggpanelene. Samtidig utfører de ofte oppgaver hvor de ikke har mulighet til å interagere med telefonen. I tråd med TAM er det grunn til å anta at pleiernes manglende aksept av systemet skyldes at de ikke ser nytten av slik bruk og heller ikke opplever det som enkelt å bruke (jf. \ref{sec:kognitive_elementer}).

\noindent
Ved å ikke benytte telefonen mister avdeling A1 redundansen av data som denne gir.
Pleiere ved de to andre avdelingene fortalte at selv om de i utgangspunktet mottar signaler på telefonene, glemmer de iblant å logge seg på. Ved å ikke være pålogget telefonen for mottak av pasientsignaler forsvinner redundansen av data, og i tilfeller hvor pleierne i tillegg ikke tilstedemarkerer seg på pasientrommet vil de være helt isolert fra omgivelsene. Dette kan utgjøre en risiko for at pasienter ikke får hjelp når de trenger det. Pleiere ved avdelingene A2 og A3 uttrykte også misnøye med støyen, men i mindre grad enn pleierne ved avdeling A1. Dette kan skyldes at sykepleierne ved disse to avdelingene er avhengige av å bruke telefonen for mottak av pasientsignaler, og dermed ser større nytte av slik bruk. Selv om pleierne ved avdeling A1 er så avhengige av å tilstedemarkere seg på pasientrom ble det ikke observert at de gjorde dette oftere enn pleierne på de to andre avdelingene. Til tross for at pleierne ønsker å motta hasteanrop umiddelbart ble det ved alle tre avdelinger observert at de iblant glemmer, eller velger å ikke tilstedemarkere seg på pasientrom. Ved avdeling A1 anser de ikke dette som et stort nok problem til at de ønsker å bruke telefonen for å oppnå en ekstra sikkerhet. 

\noindent
Pleierne ved avdeling A1 argumentere også med at varsling av signaler på telefon er forstyrrende og stressende for pasientene. Data presentert av \citet{Rygh13} viser derimot at pasientene ikke blir like forstyrret av varslingen av pasientsignaler som pleierne gjerne tror, og at pasientene heller ikke opplever at pleierne forstyrres. Både \citet{Rygh13} og egne funn tyder derimot på at pasientene likevel vegrer seg for å utløse signaler dersom de tror at pleierne har mye å gjøre, og at de dermed påvirkes av varslingene.

\noindent
Argumentene avdeling A1 har for hvorfor de ikke kan bruke systemet slik det er tenkt kan forklares med den mangelfulle tilpasningen mellom avdelingens arbeidspraksis og systemet. Det kan derfor argumenteres for at dette er den underliggende årsaken til at avdelingen har så store vanskeligheter med å bruke systemet slik det er tenkt. 

\subsubsection{Motstand mot endring}
Det er avdekket tydelige tegn på motstand mot det nye pasientsignalsystemet, spesielt ved avdeling A1 som ikke bruker telefonen slik de er pålagt (jf. \ref{sec:motstand}). \citet{Jacobsen12} trekker frem faglig uenighet rundt nødvendigheten av endringen eller valg av løsning som en mulig årsak til motstand. Som understreket flere ganger, ser ikke pleierne ved avdeling A1 nødvendigheten av å motta pasientsignaler på telefon da disse likevel varsles via veggpaneler. I tillegg opplever avdelingen kun ulemper ved den valgte løsningen, da den ikke er i samsvar med deres arbeidspraksis. Pleierne ved denne avdelingen så på systemet i det gamle sykehuset som velfungerende, og antok at de ville få et tilsvarende system etter innflytting i nye lokaler. Problemene som oppsto med det nye systemet, skapte derimot tidlig negative holdninger, og pleierne ønsket tilbake både panel i taket og assistanseknapp, og ønsket ikke å benytte telefonen for mottak av pasientsignaler. Dette kan beskrives som det \citet{Lapointe05} betegner som aktiv motstand, da pleierne tydelig ytret sine opposisjonerende meninger og nektet å bruke systemet slik det er tenkt. Det er også en form for passiv motstand at pleierne ikke er villige til å endre sine arbeidsmåter. Avdelingens ikke-bruk er dermed aktiv, motivert, overveid og strukturert i tråd med \citet{Satchell09}. Dette resulterte i at de fikk assistanseknappen tilbake da de igjen flyttet inn i nye lokaler. Her har brukernes lokale tilpasning av systemet ført til at ledelsen har vært nødt til å gjøre endringer i systemet (jf. \ref{sec:ista-rammeverket}). 

\noindent
I tråd med \citet{Orlikowski92} kan avdelingens manglende adopsjon av systemet skyldes at pleierne har en dårlig eller feilaktig forståelse av det nye systemet, noe som fører til at de ikke ser verdien av å ta det i bruk og derfor velger å ikke integrere det i sitt arbeid. Samtidig er et slikt system avhengig av et tilstrekkelig antall brukere \citep{Ackermann00}. Det vil derfor ikke ha noen hensikt for pleieren ved avdeling A1, som uttrykte en mer positiv holdning til bruk av systemet å ta i bruk dette alene. Det krever dermed en endring i både avdelingens strukturelle elementer, og pleiernes mentale modell for at de skal se nytten av slik bruk og endre sin arbeidspraksis.

\noindent
Til tross for at systemet brukes i større grad slik det er tenkt ved de to andre avdelingene, blir det også her påpekt utfordringer ved systemet. Dette er i tråd med \citet{Jacobsen12} som sier at fravær av motstand ikke nødvendigvis betyr at alle er enige i løsningen. Videre påpeker \citet{Berg99} at en av de største utfordringene ved utvikling av CSCW-systemer er det brede spekteret av brukere som ofte fører til individuelle holdninger. Ved å se på motstanden i avdeling A1 som noe positivt, og som en kritisk innvending til behovet for endring og valg av løsning, kan man dermed avdekke problemer som ikke utelukkende eksisterer her \citep{Jacobsen}. 

\subsubsection{Opplæring}	 
Som påpekt av \citet{Venkatesh99} er opplæring essensielt for at brukere skal forstå og akseptere ny teknologi.	Som funnene avdekker eksisterer det ulikheter i bruk av pasientsignalsystemet og pleiernes opplæring i bruk av systemet kan være en av årsakene til dette. I opplæringsmaterialet som foreligger brukes for eksempel begrepene $"$signal$"$, $"$anrop$"$ og $"$alarm$"$ for å beskrive et pasientsignal, samtidig som det ble observert at sykepleierne ofte bruker begrepet $"$klokke$"$ om signalene. Da opplæringsmateriell bør ha som hensikt å fremstille teknologi som enkel å bruke, er det grunn til å anta at det her virker mot sin hensikt. Det at sykepleierne gir uttrykk for at de ikke har tydelige retningslinjer på hvordan systemet skal brukes kan tyde på at dette opplæringsmaterialet benyttes i liten grad. Da opplæring trekkes frem som essensielt for at brukere skal oppleve et system som enkelt og nyttig å bruke, er det grunn til å anta at opplæringen, spesielt ved avdeling A1, har vært mangelfull (jf. \ref{sec:kognitive_elementer}). 

\noindent
Ulike fortolkninger og tilpasninger fører til avvik fra tenkt bruk som former nye normer for hvordan systemet brukes. Et eksempel på dette er pleiernes fortolkning av hasteanropet. Da avdelingene selv er ansvarlige for opplæring av nyansatte vil denne bruken videreføres, uten at den som blir opplært nødvendigvis vet at systemet er tenkt brukt annerledes. Som det ble påpekt av P1-A2 kan det oppstå en risiko dersom opplæringen har vært mangelfull og pleiere for eksempel ikke vet at de manuelt må sette seg som disp på andre sengetun for å motta signaler, inkludert hasteanrop, fra disse.

\section{Tilpasning mellom individ og oppgave}

\subsubsection{Ansvarsfordeling}
Sykepleierne ga uttrykk for at de har tilstrekkelig kunnskap om pasientene til at de kan besvare alle pasientsignaler, og det eksisterer dermed redundans av funksjon (jf. \ref{sec:redundans}). Funnene avdekket derimot delte meninger om hvorvidt pleierne besvarer signaler fra pasienter de ikke har primæransvar for. Forskerne har forsøkt å kategorisere pleiernes holdninger i tre grupper. (1) De som kun ønsker å besvare signaler fra pasienter de har primæransvar for. (2) De som  først og fremst ønsker å besvare signaler fra egne pasienter, men også besvarer andre dersom primæransvarlig er opptatt. (3) De som besvarer alle signaler, men ikke ønsker å bli forstyrret dersom de er opptatt hos en annen pasient. Slik systemet fungerer i dag er det best tilpasset pleierne i gruppe to. Pleiere i gruppe én ønsker ikke å bli varslet om signaler fra andres pasienter, og opplever derfor disse signalene som støy. Pleierne fra gruppe tre opplever varslinger som støy i visse kontekster, eksempelvis dersom de er opptatt med en annen pasient.

