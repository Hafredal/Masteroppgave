\chapter{Diskusjon}
\label{chp:diskusjon}

Vi vil her forsøke å forstå hvilke faktorer som kan være årsak til forskjellene identifisert i kapittel \ref{chp:resultater}, gjennom å analysere disse i lys av teorien som er presentert i kapittel \ref{chp:teori}. 

\noindent
Tidligere arbeid foreslår i stor grad endringer i teknologi som løsning på utfordringene ved pasientsignalsystemet. Forskerne oppdaget imidlertid gjennom sine observasjoner at problemene ikke nødvendigvis ligger i selve teknologien, men i andre faktorer. 

- ISTA som sier at likt system ofte gir helt ulik bruk

FORSKNINGSSPØRSMÅL

\section{Tilpasning mellom teknologi og oppgave}	

\subsubsection{Varsling under telefonsamtaler}
Pleierne fortalte at IP-telefonen ofte brukes for å komme i direkte kontakt med andre pleiere. Dette gir en mer effektiv arbeidshverdag da de tidligere brukte mye tid på å lete etter hverandre. Pleierne anser telefonen i slike tilfeller som et nyttig verktøy, og har dermed akseptert og integrert den i sin arbeidshverdag (jf. \ref{sec:implementering}). Videre kan man si at teknologien har påvirket det sosiale systemet, da pleierne bruker mindre tid på å kommunisere ansikt-til-ansikt (jf. \ref{section:ista-rammeverket}). Det er derimot problematisk at pasientsignaler varsles i telefonsamtaler, da dette oppleves som svært forstyrrende. Dette skaper både frustrasjon hos pleierne og forsinker deres arbeid. Dette er i tråd med \citet{Grundgeiger09} som påpeker at avbrudd kan gi negative effekter. Det kan samtidig argumenteres for at denne varslingen har en positiv effekt da signalet oppfattes, og det ville vært en risiko dersom alle aktuelle pleiere snakket i telefonen uten å bli varslet. Systemet er designet slik at pasientsignal skal varsles uansett, og dette er dermed en tilsiktet funksjon. Dette medfører at pleiere i roller hvor de ringer mye enten velger å ikke motta pasientsignaler, går med to telefoner, eller bruker tuntelefonen for utgående samtaler. Dette er eksempler på workarounds, da pleierne finner løsninger på det de opplever som en sperre i systemet (jf. \ref{section:sosioteknisk}). Disse tilpasningene av teknologien kan i tråd med \citet{Coiera07} sees på som implisitte signaler på en svakhet i systemet, og kan tvinge frem en endring i teknologien (jf. \ref{sec:ista-rammeverket}). Det er planlagt en endring hvor pasientsignaler ikke varsles under telefonsamtaler, men heller sendes videre til neste mottaker. Dette er en form for avbruddshåndtering som forebygger avbrytelser ved å blokkere innkommende signaler \citep{Grandhi10}.
	
\subsubsection{Avbrytelser}
Generelt opplever pleierne at å motta pasientsignaler på telefonen medfører flere avbrytelser i arbeidshverdagen. Varslingen generer støy som i flere situasjoner oppleves som svært forstyrrende. Mange oppgaver krever pleiernes fulle oppmerksomhet, og avbrytelser i form av pasientsignal kan være uheldig dersom deres kognitive kapasitet overskrides, da dette kan hemme oppmerksomheten og føre til feil og ineffektivitet \citep{Ebright10, Parker00}. Det oppstår dermed en avveining mellom hva som er viktigst, å fullføre en oppgave uten forstyrrelser, eller å bli varslet om pasientsignaler. Under observasjonene så forskerne aldri at en sykepleier forlot et pasientrom for å besvare et annet pasientsignal, men at de ofte avbrøt oppgaver som ikke involverte pasienter direkte. Dualiteten ved disse avbruddene gjør det utfordrende å tilpasse systemet til sykepleiernes oppgaver, da de både gir uttrykk for at de ønsker å bli varslet, samtidig som de ønsker å ha fullt fokus på pasientene. Dette medfører også at systemet brukes ulikt, og annerledes enn tenkt. For å unngå å forstyrre pasienter som sover, velger noen pleiere å legge igjen telefonen utenfor pasientrommet om natten, fordi lyden på telefonen ikke kan skrues av.

\noindent
For avdeling A1 har varslingen på telefon vært et så stort problem at de ikke ønsker å bruke telefonene til dette. En manglende tilpasning mellom teknologi og oppgave er svært fremtredende i situasjoner hvor pleiere på avdeling A1 står i stell, da de av hygieniske årsaker ikke har anledning til å ta telefonen opp av lommen. Pleierne kan i slike tilfeller ikke avvise signalet, noe som fører til mye støy da signalet varsles helt til det sendes videre. Da dette er en situasjon som ofte oppstår har pleierne på denne avdelingen problemer med å se nødvendigheten av slik bruk da de uansett ikke kan ta opp telefonen, samtidig som de får varslingen på veggpaneler.

\noindent
Data presentert av \citet{Rygh13} viser at pasientene ikke blir forstyrret av varslingen av pasientsignaler i like stor grad som pleierne tror, og at pasientene heller ikke opplever at pleierne forstyrres. Både \citet{Rygh13} og egne funn tyder derimot på at pasientene vegrer seg for å utløse signaler dersom de tror at pleierne har mye å gjøre, og at de dermed likevel påvirkes av varslingene.

\subsubsection{Utforming av teknisk system}
Til tross for at sykepleierne i stor grad samarbeider på tvers av sengetunene, er teknologien kun til en viss grad tilpasset dette, da utformingen av sløyfer ikke støtter slikt samarbeid fullstendig. Som forklart av IKT-rådgiver ved sykehuset er sløyfene forsøkt tilpasset sykepleiernes behov. Da det er ressurskrevende å gjøre endringer i sløyfene, har avdelingene måttet tilpasse seg etter disse. En signifikant forskjell på avdeling A1 og de to andre, er at pleiere på A1 ikke er avhengige av å bruke telefonen for å motta signaler fra andre tun, da hele avdelingen ligger på samme sløyfe og alle signaler varsles på veggpanelene. Avdelingene A2 og A3 er derimot, på grunn av sløyfenes utforming, nødt til å bruke telefonen. Det kan dermed argumenteres for at det faste systemet er for rigid i forhold til sykepleiernes arbeidspraksis. Og det er en klar sammenheng mellom sløyfenes utforming og sykepleiernes bruk av systemet.

\subsubsection{Forsinkelse i det trådløse systemet}
Det er likevel tydelig at veggpanelene brukes oftere som kilde til informasjon om utløste pasientsignaler, enn telefonen. Det er hovedsakelig to årsaker til dette. For det første fører forsinkelsen i det trådløse systemet til at signalene først varsles på veggpanelene. Til tross for at \citep{Sletten09} hevder at denne forsinkelsen har minimal betydning for sengetunene, viser derimot egne funn at denne kan være kritisk, spesielt ved utløste stansalarmer. Det kan argumenteres for at det i tiden før signalene varsles på telefon ikke er redundans av data for den enkelte pleier, da signalet kun varsles på panel. Da pleierne gir uttrykk for at de har tilstrekkelig kunnskap om pasientene og det dermed oppstår redundans av funksjon, kan det likevel argumenteres for at redundansen av data er opprettholdt siden signalene varsles på flere paneler samtidig (jf. \ref{sec:redundans}). For avdeling A1 


 og fortsetter på telefonene etter at det er blitt avstilt. 

 Denne forsinkelsen skaper frustrasjon hos pleierne,
 
 Flere pleiere ønsker seg større paneler, da det kan være vanskelig å lese hva som står på avstand

- små paneler
	- bak dør - interaksjon2
- trykker seg ikke inn - sikkerhet ved stans
- redundans av innsats
- ulv ulv
- panel i taket
	- det de hadde før

\subsubsection{Pleiemodell}	
Tilpasningen mellom teknologi og oppgave vil avhenge av sykepleiernes pleiemodell, da denne kan sees på som en måte å organisere oppgaver på (jf. \ref{sec:pleie}). Pasientsignaler varsles først hos primæransvarlig for pasienten, noe som vitner om at systemet først og fremst er tilpasset primærsykepleiemodellen. Forskerne observerte derimot ingen tydelig pleiemodell hos de tre avdelingene. Dagskiftene bar preg av å være organisert etter en primærsykepleiemodell, mens på kvelds- og nattskift var oppgavene i større grad organisert etter en teamsykepleiemodell. Forskerne observerte også at pleierne ofte logget seg på systemet som disp, og ikke primær på kvelds- og nattskift. Dersom hensikten er at pleierne skal være pålogget med primæransvar for pasientene, kan det anses som en workaround at pleierne velger å gjøre noe annet som bedre passer deres arbeidspraksis. Det er dermed grunn til å anta at det er bedre tilpasning mellom teknologi og oppgaven på dagskift, enn på kvelds- og nattskift. 

		
\section{Tilpasning mellom teknologi og individ}

\subsubsection{Motstand mot endring}
	 infeksjon visste at det var en prøveperiode, kanskje ikke såå positive andre gangen.
	 	- "alt var bedre før"
	 	
\subsubsection{Assistanseknapp}
Et annet skille mellom avdelingene er at avdeling A1 har mulighet til å utløse assistansesignal ved behov for assistanse, mens alternativet for de to andre avdelingene er å utløse et hasteanrop. I opplæringsdokumentene til sykehuset er hasteanropet beskrevet som et signal som kan utløses ved behov for assistanse. Det ble derimot avdekket en sterk felles oppfatning av at hasteanropet kun brukes ved nødsituasjoner. Sykepleierne uttrykte at de synes det er problematisk å utløse hasteanrop ved mindre alvorlige situasjoner, og det ble observert at pleierne ved A2 og A3 heller stikker hodet ut i gangen for å be om hjelp. Det har dermed oppstått en workaround fordi fortolkningen av systemet skiller seg fra tenkt bruk (jf. \ref{sec:implementering}, {sec:ista-rammeverket}). Det er dermed tydelig at sykepleierne har behov for både et assistansesignal og et signal beregnet for nødsituasjoner.

- masing om assistanse
- ut ifra opplæringsmateriale så finnes den
- men sykepleierne kalle det stans, og bruker den slik, slipper alt
- kultur
- to forskjellige behov, fra utviklernes side er behovet dekt
- tvunget til endring

  
\subsubsection{Avbrytelser}
- frustrasjon over å bli avbrutt
	- stressende

- infeksjon klager på lyd, men vil ikke bruke det uansett
	- ser ikke nytten


Forskerne observerte derimot ikke at signaler ble avvist på andre avdelinger heller.

- kultur for å la det ringe lenge /interagerte lite med telefonene

- frustrasjon over å bli avbrutt
	- stressende

- infeksjon klager på lyd, men vil ikke bruke det uansett
	- ser ikke nytten
	

\subsubsection{Opplæring}	 	
- opplæring 
	- vet ikke det med stans

- ulike brukere
	- ser tendens til felles verdier hos A1, noe mer varierende hos de andre



\section{Tilpasning mellom individ og oppgave}

\subsubsection{Tilstedemarkering}
Bruk av veggpaneler og telefon for mottak av pasientsignaler gir en redundans av data, og dermed også en sikkerhet for at signalene blir oppfattet. Ved å ikke benytte telefonen mister avdeling A1 denne sikkerheten. Pleiere ved de to andre avdelingene fortalte at selv om de i utgangspunktet mottar signaler på telefonene, glemmer de iblant å logge seg på. Ved å ikke være pålogget telefonen for mottak av pasientsignaler forsvinner redundansen av data, og i tilfeller hvor pleierne i tillegg ikke tilstedemarkerer seg på pasientrommet vil de være helt isolert fra omgivelsene. Dette kan utgjøre en risiko for at pasienter ikke får hjelp når de trenger det. Selv om avdeling A1 er så avhengige av å tilstedemarkere seg på pasientrom, ble det ikke observert at de brukte denne funksjonen oftere. 



- redundans av funksjon
	- men ikke alle svarer på alle
	
- flere snakker om det med vane.
 - noe er en rutine for noen
 
- At signalene varsles tidligere på veggpanelene gjør det ekstra viktig at pleierne tilstedemarkerer seg, slik at de ved utløste stansalarmer mottar signalet umiddelbart. 


\section{Oppsummering}
- noen bruker kanskje mer fordi de "må" - ortopedi	

telefonene skal være "system nr. 1", men blir nr. 2 pga forsinkelsen,

- Alle er opptatt av at systemet må TILPASSES

SVAR PÅ FORSKNINGSSPØRSMÅL!!