\chapter{Diskusjon}
\label{chp:diskusjon_rq}
Dette kapittelet vil oppsummere funnene som besvarer forskningsspørsmålene:

\begin{enumerate}
\item Hvordan brukes pasientsignalsystemet ved St.Olavs Hospital forskjellig i, og mellom ulike avdelinger? 
\item Hvilke faktorer kan være årsak til disse forskjellene?
\end{enumerate}

\noindent
Da pleiernes bruk av pasientsignalsystemet i mange tilfeller skiller seg fra slik det er tenkt, har det vært naturlig å også se på disse forskjellene.

\noindent
Den mest fremtredende forskjellen fra tenkt bruk, og mellom avdelingene, er at avdeling A1 ikke benytter telefonen for mottak av signaler. Det er hovedsakelig to årsaker til dette: (1) de opplever ingen nytte ved slik bruk, og (2) det er ofte problematisk å interagere med telefonen. Dette er helt i tråd med TAM som sier at et individs ønske om å bruke et system avhenger av nettopp disse to faktorene (jf. \ref{sec:kognitive_elementer}). 

\noindent
Videre er det flere årsaker til at pleierne ikke ser nytte av å bruke systemet. For det første er ikke avdelingen avhengig av telefonen for å motta signaler fra alle tun, da disse ligger på samme sløyfe. Mange av oppgavene pleierne utfører er tidkrevende å starte og avslutte, og de avbryter derfor sjelden disse for å besvare andre signaler. Selv om hensikten med varsling gjennom både det trådløse og faste systemet er å oppnå redundans av data, og en sikkerhet for at signaler oppfattes, ser ikke pleierne behovet for dette da de på grunn redundans av funksjon stoler de på at signaler blir besvart av andre. De ønsker likevel ikke å være helt isolert, og ser derfor nytten av å motta varslinger via veggpaneler da disse har større $"$ease of use$"$ og effektene av disse varslingene oppleves som positive. Pleierne ved avdeling A1 argumenterer også med at de ofte ikke har mulighet til å interagere med telefonen. At funksjonene for å godta eller avvise et pasientsignal dermed brukes i liten grad er et avvik fra tenkt bruk som fører til unødvendig støy som oppleves som svært forstyrrende både for pleiere og pasienter. Denne utfordringen eksisterer også ved de to andre avdelingene. Mange av oppgavene pleierne utfører krever deres fulle oppmerksomhet, og varsling på telefonen oppleves av pleiere ved avdeling A1 som en avbrytelse med bare negative effekter. Forsinkelsen i det trådløse systemet fører til at informasjonen varslingen gir allerede er mottatt og oppleves i flere tilfeller som unødvendig. Det kan dermed argumenteres for at det er bedre tilpasning mellom pleiernes arbeidspraksis og det varslingen i det faste systemet, enn varslingen i det trådløse.

\noindent
Da både avdeling A2 og A3 er fordelt på flere sløyfer ser de nytten av å motta signaler på telefon, nettopp for å veie opp for den til dels manglende redundansen av data. Til tross for eventuelt manglende $"$ease of use$"$ ser de derfor stor nok nytte av slik bruk til at de har akseptert og integrert bruk av telefon for mottak av signaler i sin arbeidshverdag.

\noindent
Det er tydelige individuelle forskjeller på hvordan sykepleierne på avdeling A2 og A3 bruker bemanningsplanen for å distribuere pasientsignaler. Systemet er bedre tilpasset pleiernes arbeidspraksis på dagtid, hvor de i større grad besvarer signaler fra pasienter de har primæransvar for, enn på kvelds- og nattskift, hvor pleierne i større grad deler på det totale ansvaret for utløste pasientsignaler. I tillegg fører avdelingenes tekniske utforming til at pleierne må logge seg på på ulike måter for å knytte tun på forskjelliger sløyfer sammen på den måten de ønsker. 

\noindent
Da pleierne uttrykte at de ser nytten av å tilstedemarkere seg på pasientrom kan det antas at årsaken til at dette ofte ikke gjøres er at det glemmes. Dette gir individuelle forskjeller da noen i større grad har dette som vane. 

\noindent
Det er avdekket flere forkjellige workarounds, hvor flere har til hensikt å redusere støy. Da workarounds beskrives som mekanismer for å jobbe rundt problemer, uten å forsøke å løse den underliggende årsaken, kan vi anta at støyen er et symptom på et bakenforliggende problem. Det er også avdekket workarounds som har oppstått som følge av at systemet er for rigid i forhold til pleiernes arbeidspraksis.  Pleiernes fortolkning av hasteanropet skiller seg fra slik det er tenkt, og medfører en workaround ved at pleierne heller åpner døren og ber om hjelp enn å utløse et sikt signal. Det oppstår dermed tydlige ulikheter da avdeling A1 har et eget assistansesignal. Disse workaroundsene oppstår som følge av lokale tilpasninger i avdelingene og fører til at systemet brukes ulikt mellom avdelingene. Da avdelingene selv står for opplæring av nyansatte vil lokale holdninger og arbeidspraksis videreføres, og forskjellene som eksisterer vil dermed bli opprettholdt.   

\noindent
Det er tydelig at samspillet mellom individ, teknologi og oppgave ikke er optimal. Dette er kommet til syne både gjennom avvik fra tenkt bruk og sykepleiernes ulike holdninger til systemet. For å bedre dette samspillet må egenskper ved objektene endres. Da systemet skal dekke mange og ulike behov for svært mange brukere vil forskerne argumentere for at det er nødvendig å gi avdelingene rom for å gjøre lokale tilpasninger. Tidligere arbeid har foreslått ulike tileggsfunksjoner som pasientmelding \citep{Rygh13, Selseth12} og distribusjon av informasjon om kollegers tilgjengelighet \citep{Sund13}. Forskerne vil derimot argumentere for å gjøre systemet enklere heller en å øke kompleksiteten ved å legge til flere funksjoner, spesielt dersom hensikten er å oppnå lik bruk over hele sykehuset.