\chapter{Konklusjon og videre arbeid}
\label{chp:konklusjon} 
Hensikten med dette forskningsarbeidet har vært å avdekke forskjeller i sykepleiernes bruk av pasientsignalsystemet og finne årsaker til disse forskjellene med en sosioteknisk tilnærming. Den tydeligste forskjellen er at avdeling A1 ikke anvender telefonen til mottak av pasientsignaler, noe som skyldes at pleierne ikke ser nytten av slik bruk og finner systemet vanskelig å bruke. Hos avdelingene som bruker telefonen til mottak av pasientsignaler er det forskjeller i hvordan de bruker bemanningsplanen for å distribuere disse avhengig av tid på døgnet. Ved alle tre avdelinger er det individuelle forskjeller i bruk av tilstedemarkering på pasientrom, uten at det er avdekket andre årsaker til dette enn at det glemmes. Som et resultat av manglende tilpasning mellom sykepleiernes arbeidspraksis og teknologi oppstår det workarounds som videre former nye arbeidspraksiser som videreføres da opplæring av nyansatte skjer lokalt på avdelingene. 

\noindent
Det sosiotekniske samspillet er svært komplekst, og omfatter svært mange aspekter. Forskningsarbeidet har berørt mange temaer, men på grunn av prosjektets begrensede omfang gir denne oppgaven kun en overfladisk analyse av noen av disse. 

\subsubsection{Videre arbeid}
Funnene i denne oppgaven kan inspirere til videre forskning som i større grad kan gå i dybden på de presenterte temaer. Da det er funnet manglende tilpassning mellom tilpasning mellom arbeidspraksis og teknologi kan det være interessant å se på hvorfor systemet ble designet slik. Det kan også være interessant å se på om det er hensiktsmessig å la avdelingene gjøre lokale tilpasninger eller om lik bruk bør etterstrebes. En annane interessant vinkling kan være å sammenligne kvalitative og kvntitative målinger for å se om sykepleiernes bruk av systemet påvirker mål som varslingstid, antall utløste signaler og pasientenes tilfredshet. For å øke datagrunnlaget og gyldigheten ved forskningen bør videre arbeid omfatte flere avdelinger. 