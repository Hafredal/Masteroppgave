\chapter{Informasjonsskriv vedrørende observasjonsstudie}
\label{chp:appendix_informasjon_observasjon}

\textbf{Informasjonsskriv og forespørsel om deltakelse ved observasjon i forskningsprosjekt (”Kommunikasjon ved bruk av trådløst pasientsignal”) til pleiere og pasienter. } 

\noindent
\textbf{Formål med studien}

\noindent
St. Olavs Hospital og enkelte andre norske sykehus har innført trådløst system for mottak av pasientsignal. Prosjektet CoCoCo (kommunikasjon ved bruk av trådløst pasientsignal) er et forskningsprosjekt ved NTNU. Prosjektet har som formål å kartlegge hvordan det nye trådløse pasientsignalsystemet faktisk brukes av helsearbeidere og pasienter når man kommuniserer seg imellom. Av særlig interesse er det å forstå hvordan helsearbeidere og pasienter selv opplever bruken av det trådløse pasientsignalsystemet. Å få kunnskap omkring brukerperspektivet (pasienter og helsearbeidere) er således sentralt i prosjektet. Ved å innhente kunnskap om bruk og erfaringer skal prosjektet også finne ut hvorvidt og eventuelt på hvilken måte et slikt system kan forbedres.

\noindent
Ansvarlig for prosjektet er norsk senter for elektronisk pasientjournal (NSEP) ved NTNU ved daglig leder, førsteamanuensis og lege Arild Faxvaag. Antall forskere tilknyttet prosjektet er 7, herunder 3 mastergradsstudenter. Forskningsprosjektet er et samarbeid mellom institutt for telematikk, institutt for data og informasjonsteknologi og NSEP, NTNU. 

\noindent
\textbf{Metode}

\noindent
Studien vil foregå ved deltagende observasjon av helsearbeidere og pasienter ved ulike avdelinger på sykehuset. Observasjonene vil foregå på to måter: 1) ved at forskerne er til stede ved avdelingen og observerer hva som foregår; og 2) ved at forskerne følger en helsearbeider og observerer kommunikasjonen han/hun er involvert i. Varighet på observasjon knyttet til sengetun ved avdelingene (1 ovenfor) vil være omtrent 3 timer, mens varighet på observasjon knyttet til rolle (2 ovenfor) vil være maksimalt 3 timer. I forhold til observasjon hvor pasient er involvert vil dette være betydelig kortere, kun i forbindelse med at pleier er inne på pasientrom.

\noindent
En forsker vil følge etter helsearbeider inn på pasientrom der hvor pasienten selv har sagt seg villig til å delta i prosjektet. Forskerne vil ta håndskrevne notater fra observasjonene. Ved å observere ønsker vi å innhente kunnskap om hvordan det trådløse pasientsignalsystemet benyttes mellom pleiere og pasienter og mellom ulike pleiere.

\noindent
\textbf{Deltakelse}

\noindent
Ønsker du/dere (pasient/pleier) å delta i prosjektet, si ifra til den personen som du mottok informasjonsskrivet fra, alternativt ta direkte kontakt med prosjektleder (se kontaktinfo nedenfor).

\noindent
Deltakelse i forskningsprosjektet er frivillig. Du kan når som helst trekke deg uten å oppgi noen begrunnelse. Du kan trekke deg fra studien også etter at observasjonene er gjennomført og frem til studiens slutt (1. november, 2017). Alle innsamlede opplysninger vedrørende deg vil da bli fullstendig slettet. Bruk i så fall kontaktinfo som står nedenfor.

\noindent
\textbf{Opplysninger som vil bli brukt}

\noindent
Data samlet inn fra observasjoner vil bli behandlet konfidensielt. Vi vil ikke registrere data som direkte kan identifisere deg som person (som navn eller fødselsdato). Informasjon om deg som vi vil registrere inkluderer yrkestittel (gjelder helsearbeider), kjønn og omtrentlig alder. Alt materiale fra datainnsamlingen -- notater fra observasjonene -- vil bli oppbevart forsvarlig slik at kun personer tilknyttet forskningsprosjektet har tilgang. Som deltaker har du full innsynsrett i dataene. 

\noindent
Innhentede data makuleres 5 år etter prosjektslutt (prosjektslutt er 1. november 2017). Dataene vil kunne brukes i relaterte prosjekter i denne perioden. Dataene vil benyttes som grunnlag for vitenskapelig publisering og undervisning i relevante fora.

\noindent
Prosjektet er vurdert av personvernombudet for forskning (NSD).

\noindent
Med vennlig hilsen,                                                                                               

\noindent
Prosjektleder Joakim Klemets

\noindent
Tlf: 40489542

\noindent
Epost: joakim@item.ntnu.no

\noindent
Institutt for telematikk og

\noindent
Norsk senter for elektronisk pasientjournal (NSEP),

\noindent
NTNU

\noindent
Samtykke til deltakelse i studien

\noindent
Jeg er villig til å delta i studien (observasjon)

\noindent
---------------------------------------------------------------------------------------------

\noindent
(Signert av prosjektdeltaker, dato)


\noindent
Jeg bekrefter å ha gitt informasjon om studien


\noindent
---------------------------------------------------------------------------------------------

\noindent
(Signert, rolle i studien, dato)

