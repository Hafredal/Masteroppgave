\section{Organisering av pleie}
\label{sec:pleie}

Sykepleiernes oppgaver kan organiseres på flere måter, blant annet teamsykepleie, primærsykepleie og oppgaveorientert sykepleie. Førstnevnte deler oppgavene inn i kategorier som medisinering og sårstell. Hver sykepleier får deretter sine oppgaver som de har ansvar for den aktuelle vakten. På denne måten vil sykepleierne som team tilsammen utføre alle oppgaver, for alle pasienter. Teamledere koordinerer omsorgen utført av teamet og har, i samarbeid med avdelingsleder, ansvar for pasientbehandling og kommunikasjon med annet halsepersonell. I primærsykepleie tildeles pasientene til individuelle sykepleiere. Primærsykepleier har ansvar for koordinering av omsorg og pleie for et lite antall paseinter fra innleggesle til utskrivelse, uten å måtte gjøre alle oppgaver knyttet til dette. Oppgaveorientert sykepleie prioriterer avdelingen rutiner over behovene til den enkelte pasient. Den enkelte sykepleier har sine faste oppgaver som utføres for alle pasientene på sengeposten. Oppgavene er fordelt basert på fedighetsnivå og kan være blant annet medisinutdeling eller sårskift \citep{Rygh13}.

\noindent
Forskning viser til at innføring av primærsykepleie i stor grad kan gjøre pleierne mer autonome i sitt arbeid, og også mer pasientorientert. Nyere forskning viser imidlertid at avdelinger ofte er organisert på måter som tar i bruk egenskaper fra de forskjellige modellene. Et eksempel på dette er modulær sykepleie som er organisert rundt relativt små geografiske grupperinger av pasienter, og hvor pleiepersonellet har ansvar for den totale omsorgen, og distribuerer oppgaver innenfor teamet \citep{Rygh13}.
