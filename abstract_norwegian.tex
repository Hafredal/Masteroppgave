\pagestyle{empty}
\renewcommand{\abstractname}{Sammendrag}
\begin{abstract}
\noindent
Ved utbyggingen av nye St. Olavs Hospital ble den leverte IKT-løsningen betegnet som Norges dyreste og mest kompliserte IKT-prosjekt. Infrastrukturen som ble implementert inneholder blant annet et fast og et trådløst pasientsignalsystem. Pasientsignaler utløses av pasienter ved behov for assistanse, og leveres til sykepleiere gjennom varsling på telefon og/eller veggpaneler. Denne studien omhandler sykepleiernes bruk av dette systemet.

\noindent
Det ble tidlig avdekket tydelige ulikheter i sykepleiernes anvendelse av pasientsignalsystemet. Et slikt system benyttes av mange brukere i ulike kontekster med ulike behov. Det er derfor blitt gjort lokale tilpasninger av systemet som skiller seg fra tenkt bruk.  Motivasjonen for forskningsarbeidet ble dermed å kartlegge disse variasjonene og å identifisere årsaker til hvorfor de har oppstått, med en sosioteknisk tilnærming. En slik tilnærming til IKT-systemer forsøker å forstå hvordan mellommenneskelige aspekter og tekniske systemer påvirker hverandre.

\noindent
Datagrunnlaget er basert på kvalitative metoder, hvor tre avdelinger ved sykehuset ble observert og pleiere fra disse ble intervjuet. Funnene avdekket noen hovedårsaker som i stor grad vitner om at det er manglende tilpasning mellom sykepleiernes arbeidspraksis og det tekniske systemet. 
\end{abstract}