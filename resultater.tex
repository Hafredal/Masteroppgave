\chapter{Resultater}
\label{chp:resultater} 

Forskerne vil i dette kapittelet redegjøre for funn gjort under observasjoner og intervjuer, og presentere disse i tråd med FITT-rammeverket presentert i kapittel \ref{sec:fitt-rammeverket}. Hensikten har vært å identifisere likheter og ulikheter i bruk, og disse vil bli brukt som utgangspunkt for videre diskusjon i kapittel \ref{chp:diskusjon}.

%som presentert i tabell \ref{detaljeravdelinger}.

%TABELL
\begin{table}[H]\centering
    \begin{tabular}{ |l|l|l|l|l|l| }
    \hline
    Avdeling & Intervjuobjekt & Referanse \\ \hline
       A2 & Seksjonsleder & L-A2 \\ \hline
       A2 & Sykepleier & P1-A2 \\ \hline
       A2 & Sykepleier & P2-A2 \\ \hline
       A1 & Assisterende seksjonsleder & L-A1 \\ \hline
       A1 & Hjelpepleier & P1-A1 \\ \hline
       A1 & Sykepleier & P2-A1 \\ \hline
       A3 & Seksjonsleder & L-A3 \\ \hline
       A3 & Sykepleier & P1-A3 \\ \hline
       A3 & Sykepleier & P2-A3 \\ \hline
       - & IKT-rådgiver ved St. Olavs Hospital &  \\ \hline
    \end{tabular}
    \caption {Detaljer for intervjuer}
    \label{detaljerintervju}
\end{table}

\section{Teknologi}
IP-telefonen som benyttes som en del av pasientsignalsystemet har vært et sentralt aspekt ved forskningsarbeidet, og det ble avdekket tydelige ulikheter i bruk og holdninger til denne. På alle tre avdelinger gikk sykepleierne med hver sin telefon, og brukte disse for å ringe, og motta telefonanrop. Både seksjonsledere og pleiere ga under intervjuene uttrykk for at telefonen ofte brukes i situasjoner hvor de ønsker å komme i direkte kontakt med andre pleiere. L-A1 påpeker at de bruker mindre tid på å lete etter hverandre nå som pleierne går med telefon. Dette bekreftes av P1-A1, som også sier at pleierne, på grunn av avdelingens fysiske utforming, bruker mye tid på å gå gjennom avdelingen. Generelt uttrykker flere at telefonen er et \textit{nyttig} verktøy for kommunikasjon med andre, og at det er dette de bruker den mest til. Det trekkes frem som et sentralt problem på alle avdelinger at pasientsignaler varsles i telefonsamtaler, da varslingen oppleves som svært forstyrrende, spesielt blant seksjonslederne og pleiere som ofte har roller hvor de ringer mye. P2-A2 forklarte det slik: \textit{ $"$... hvis jeg snakker i telefonen, [...] og så ringer [en klokke], så bryter den gjennom, og det er forferdelig å snakke i telefonen$"$}. L-A2 understreker at det i hans tilfelle vil være uaktuelt å motta pasientsignaler på sin telefon, da han har mange telefonsamtaler i løpet av en dag. P1-A3 forklarte at de løser dette problemet med å heller bruke fasttelefonen på tunet, mens L-A2 fortalte at ansvarshavende sykepleier ofte går med to telefoner, hvor en mottar pasientsignaler og den andre telefonsamtaler. I samtale med IKT-rådgiver ble forskerne gjort oppmerksomme på at systemet vil bli endret, slik at signaler ikke mottas på telefoner som brukes til samtaler, men heller sendes direkte videre til neste mottaker.

\noindent
Ved innflytting i nye lokaler forsøkte avdeling A1 å bruke telefonen slik den er tenkt med mottak av pasientsignaler, men etter en prøveperiode på to uker konkluderte de med at denne løsningen ikke fungerte. Dette til tross for at de er pålagt slik bruk. Seksjonsleder og pleiere viste til flere årsaker til dette valget. Pleierne står ofte i stell, og har av hygieniske årsaker ikke alltid mulighet til å besvare signalene som mottas. L-A1 påpekte videre at de har en pasientgruppe som hyppig utløser pasientsignaler. Dette resulterte i mye støy beskrevet av L-A1 som \textit{$"$uutholdbar$"$}. Avdelingen opplevde også at pasienter ble forstyrret av støyen, og en av sykepleierne fortalte under observasjonene at pasientene mistet tillit til pleierne da det ringte i lommen deres uten at de besvarte anropet. Ved de to andre avdelingene brukes telefonen i større grad slik den er tenkt, men også der trekkes støy frem som en vesentlig utfordring. P2-A2 beskriver opplevelsen av varslinger på telefon slik: \textit{ $"$...det med telefon og sånn, det synes jeg [er] veldig urolig, urolig hverdag, bråkete. Særlig hvis du står i stressende, pressende situasjoner så ringer det, og ringer, og ringer, og ringer...$"$}.

\noindent
Pleiere fra alle tre avdelinger påpeker at varslingen på telefon er et problem om natten, i tilfeller hvor sykepleieren er inne hos en pasient når det utløses et signal. P2-A2 ser nytten av å gå med telefonen på nattevakt, da hun ønsker å være tilgjengelig for kolleger og pasienter som trenger hjelp. Samtidig uttrykte hun at det er utfordringer knyttet til dette: \textit{ $"$... så har jeg jo opplevd at pasienter våkner og ikke får sove igjen. Og det er jo en bakdel$"$}. Hun fortalte videre at pleiere på avdelingen iblant legger igjen telefonen utenfor rommet for å unngå dette problemet, men at dette skaper en ny utfordring da de blir utilgjengelige på telefon. For avdeling A2 og A3 vil dette være alvorlig, da det innebærer at de ikke blir varslet om stansalarmer fra andre tun. Dette er et sentralt poeng da disse avdelingene, i motsetning til A1 ikke mottar alle signaler fra andre tun på panelene. Dette gjelder også på dagtid, og er dermed et av de mest fremtredende argumentene for å motta pasientsignaler på telefon. 

\noindent
Både pleiere og seksjonsledere fra avdelingene A1 og A3 trekker frem telefonens grensesnitt som lite brukervennlig, og beskriver den som tungvint å bruke. L-A1 uttrykker seg slik:

\noindent
\textit{ $"$... det jeg reagerer på når det gjelder IP-telefonen, er at man må sende en hel avdeling på kurs i flere timer for å forstå en telefon. [...] Det er ressurskrevende, og for komplisert. [...] . [Den har funksjoner] som er veldig tungvinte, og det er veldig lite selvforklarende, veldig lite brukervennlig, og skiller seg vesentlig fra mobiltelefoner...$"$}.

\noindent
Et interessant funn gjort under observasjonene var at i tilfeller hvor pleierne befant seg på tunet og pasientsignal ble utløst, så et fåtall av pleierne på telefonen. Istedet brukte de veggpanelene for å se hvor signalet ble utløst fra, og gikk inn til pasienten uten å bekrefte dette på telefonen. Dette ble bekreftet under intervjuene, og 

- bruker paneler mer enn telefon 

- forsinkelse - telefonene skal være "system nr. 1", men blir nr. 2 pga forsinkelsen, "man blir litt immune".

-panelene små for å tvinge de til å bruke telefon
	- ønsker seg panel i taket




	- INFEKSJON - hjelper ikke hvor mye telefonen endres!!

- grensesnitt



- spøkelsesalarmer - tekniske feil



Forskjeller?
- ulike sløyfer, hjerte og ortopedi er helt avhengige av telefonen for å motta stans.	
	- noen bruker kanskje mer fordi de "må" - ortopedi
	
	
FORSLAG TIL ENDRING FRA DE ULIKE AVDELINGENE	

\section{Individ}
Hva er felles for alle?

Forskjeller?

- opplæring, spesielt kulturen den styrker.

- infeksjon ser ikke behov eller nytte for signaler på telefon.

- hjerte ikke negative til systemet, klager på mer tekniske feil og ikke systemet i seg selv. skjønner at det gir en trygghet.

- ortopedi synes det er et godt verktøy, men med forbedringspotensiale. skjønner at det gir en trygghet.

- innføring - infeksjon visste at det var en prøveperiode, kanskje ikke såå positive andre gangen.

- flere snakker om det med vane.

- Therese nevner tvang, at de må osv...

Alle er opptatt av at systemet må TILPASSES

\section{Oppgave}

Hva er felles for alle?

- planlagte oppgaver/tilfeldige oppgaver
- varierende hvor flinke folk er til å trykke seg inn, glemmer det, alle vet det og ser nytte. De gjør det stort sett når de vet at de skal være der en stund.

- alle er positive til selve oppgaven å gå inn til pasient (naturligvis). 


Forskjeller?
Selv om de tre avdelingene alle er sengeposter ved St. Olavs hospital er det noen generelle forskjeller vi ønsker å trekke frem. 
For det første er det stor forskjell på pasientgruppene på de forskjellige avdelingene. Både pasientenes alder, almenntilstand og gjennomsnittelig liggetid varierer mellom avdelingene, noe som gir forskjeller i arbeidsoppgaver og rutiner.  



