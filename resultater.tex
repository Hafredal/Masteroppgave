\chapter{Resultater}
\label{chp:resultater} 

I denne delen av oppgaven vil vi presentere funnene vi gjorde gjennom observasjonsrundene, og de påfølgende inervjuene. Vi vil se på hvordan systemet brukes på de forskjellige avdelingene i dag, og identifisere forskjeller i bruk. FITT-rammeverket presentert i kapittel \ref{sec:fitt-rammeverket} vil bli brukt som utgangspunkt for analysen og for strukturering av funnene.

\noindent
Hvordan er analysen gjort?

\noindent
Observasjonene og intervjuene ble gjennomført på tre forskjellige avdelinger, som vist i tabellene \ref{detaljer1}, \ref{detaljer2} og \ref{detaljerintervju}. Selv om de tre avdelingene alle er sengeposter ved St.Olavs hospital er det noen generelle forskjeller vi ønsker å trekke frem. For det første er det stor forskjell på pasientgruppene på de forskjellige avdelingene. Både pasientenes alder, almenntilstand og gjennomsnittelig liggetid varierer mellom avdelingene, noe som gir forskjeller i arbeidsoppgaver og rutiner. For det andre 

\section{Teknologi}

HER MÅ VI NEVNE NOE OM AT DET VAR INTERESSANT VED KODINGEN AV DATAENE, AT FÅ KODER OVERLAPPET MELLOM KATEGORIENE SIGNAL OG TELEFON - ERGO FÅ BRUKER TELEFONEN AKTIVT VED INNKOMMENDE SIGNAL. 

Hva er felles for alle?

- alle bruker telefonen til å ringe med

- alle klager på lyd, men i ulik grad, forstyrrende at det ringer i lomma, og at det går gjennom selv om du snakker i telefonen, nivå og melodi.

	- INFEKSJON - hjelper ikke hvor mye telefonen endres!!

- grensesnitt

- bruker paneler mer enn telefon 

- forsinkelse - telefonene skal være "system nr. 1", men blir nr. 2 pga forsinkelsen, "man blir litt immune".

- spøkelsesalarmer - tekniske feil

- noen bruker kanskje mer fordi de "må" - ortopedi

Forskjeller?

- infeksjon bruker ikke telefonen til å motta signaler.

- ulike sløyfer, hjerte og ortopedi er helt avhengige av telefonen for å motta stans.

\section{Individ}
Hva er felles for alle?

Forskjeller?

- opplæring, spesielt kulturen den styrker.

- infeksjon ser ikke behov eller nytte for signaler på telefon.

- hjerte ikke negative til systemet, klager på mer tekniske feil og ikke systemet i seg selv. skjønner at det gir en trygghet.

- ortopedi synes det er et godt verktøy, men med forbedringspotensiale. skjønner at det gir en trygghet.

- innføring - infeksjon visste at det var en prøveperiode, kanskje ikke såå positive andre gangen.

- flere snakker om det med vane.

\section{Oppgave}
Hva er felles for alle?

- planlagte oppgaver/tilfeldige oppgaver
- varierende hvor flinke folk er til å trykke seg inn, glemmer det, alle vet det og ser nytte. De gjør det stort sett når de vet at de skal være der en stund.

- alle er positive til selve oppgaven å gå inn til pasient (naturligvis). 



- 

Forskjeller?
 



