\chapter{Resultater}
\label{chp:resultater} 

Forskerne vil i dette kapittelet redegjøre for funn gjort under observasjoner og intervjuer. På tross av at funnene er komplekse og i stor grad henger tett sammen, er de her inndelt etter de tre objektene: teknologi, individ og oppgave. Dette er gjort for å lettere identifisere egenskapene ved de enkelte objektene, og i tråd med ISTA- og FITT-rammeverket (se kapittel \ref{sec:fitt-rammeverket}) vil samspillet mellom disse bli diskutert i kapittel \ref{chp:diskusjon}.

\begin{table}[H]\centering
    \begin{tabular}{ |l|l|l|l|l|l| }
    \hline
    Avdeling & Intervjuobjekt & Referanse \\ \hline
       A1 & Assisterende seksjonsleder & L-A1 \\ \hline
       A1 & Hjelpepleier & P1-A1 \\ \hline
       A1 & Sykepleier & P2-A1 \\ \hline
       A2 & Seksjonsleder & L-A2 \\ \hline
       A2 & Sykepleier & P1-A2 \\ \hline
       A2 & Sykepleier & P2-A2 \\ \hline
       A3 & Seksjonsleder & L-A3 \\ \hline
       A3 & Sykepleier & P1-A3 \\ \hline
       A3 & Sykepleier & P2-A3 \\ \hline
       - & IKT-rådgiver ved St. Olavs Hospital &  \\ \hline
    \end{tabular}
    \caption {Referanser for intervjuobjekter}
    \label{referanserintervju}
\end{table}

\section{Teknologi}
Teknologi beskrives av FITT-rammeverket som det tekniske systemet og de verktøy som brukes for å utføre oppgaver. Egenskaper ved teknologien kan blant annet være dens stabilitet, brukbarhet, infrastruktur og funksjonalitet. 

\noindent
Avdelingene har ulik teknisk utforming og det er derfor forskjeller på hvilke pasientsignaler som varsles hvor, og på hvilke enheter. Veggpanelene på avdeling A1 viser signaler fra alle tun, da hele avdelingen ligger på samme sløyfe. Ved avdeling A2 ligger det ene sengetunet på en annen sløyfe enn de to andre, og pasientsignaler fra det ene sengetunet varsles dermed ikke på veggpanelene til de to andre, og motsatt. Dette medfører at pleierne er avhengige av å bruke telefonene for å motta pasientsignaler på tvers av disse sløyfene, noe som er spesielt viktig på kvelds- og nattskift hvor det er færre pleiere på jobb. For avdeling A3 gjelder det samme, da de fire sengetunene er fordelt på to etasjer, hvor de to tunene i hver etasje ligger på samme sløyfe.

\noindent
IP-telefonen som benyttes som en del av pasientsignalsystemet har vært et sentralt aspekt ved forskningsarbeidet, og det ble avdekket tydelige ulikheter i bruk av denne. På alle tre avdelinger går sykepleierne med hver sin telefon, og bruker den for å ringe, og motta telefonanrop, men bare avdeling A2 og A3 bruker den for å motta pasientsignaler. Både seksjonsledere og pleiere ga under intervjuene uttrykk for at telefonen ofte brukes i situasjoner hvor de ønsker å komme i direkte kontakt med andre pleiere. L-A1 og P1-A1 påpeker at de bruker mindre tid på å lete etter hverandre nå som de går med telefon.   

\noindent
Generelt uttrykker flere at telefonen er et nyttig verktøy for kommunikasjon med andre, og at det er dette de bruker den mest til. Det trekkes derimot frem som et sentralt problem på alle avdelinger at pasientsignaler varsles i telefonsamtaler, da varslingen oppleves som svært forstyrrende, spesielt blant seksjonslederne og pleiere i roller hvor de ringer mye. P2-A2 forklarte det slik: \textit{ $"$... hvis jeg snakker i telefonen, [...] og så ringer [et pasientsignal], så bryter den gjennom, og det er forferdelig å snakke i telefonen$"$}. L-A2 understreker at det i hans tilfelle vil være uaktuelt å motta pasientsignaler på sin telefon, da han har mange telefonsamtaler i løpet av en dag. P1-A3 forklarte at de løser dette problemet med å heller bruke fasttelefonen på tunet, mens L-A2 fortalte at ansvarshavende sykepleier ofte går med to telefoner, hvor den ene mottar pasientsignaler og den andre telefonsamtaler. I samtale med IKT-rådgiver ved sykehuset ble forskerne gjort oppmerksomme på at systemet vil bli endret, slik at signaler ikke varsles under telefonsamtaler, men heller sendes direkte videre til neste mottaker.

\noindent
Ved innflytting i nye lokaler forsøkte avdeling A1 å bruke telefonen slik den er tenkt, til mottak av pasientsignaler, men etter en prøveperiode på to uker konkluderte de med at de ikke ønsket å benytte denne løsningen. Dette til tross for at de er pålagt slik bruk. Seksjonsleder og pleiere viste til flere årsaker for dette. Pleierne står ofte i stell, og har av hygieniske årsaker ikke alltid mulighet til å besvare signalene som mottas. L-A1 påpekte videre at de har en pasientgruppe som hyppig utløser pasientsignaler. Dette resulterte i mye støy beskrevet av L-A1 som \textit{$"$uutholdbar$"$}. Forskerne kan imidlertid ikke si å ha observert at pasientgruppen på avdeling A1 utløser flere signaler enn pasientene på de to andre avdelingene. Forskerne så heller ikke en sammenheng mellom mengden utløste signal, tid og avdeling. 

\noindent
Avdeling A1 opplevde også at pasienter ble forstyrret av støyen, og en av sykepleierne fortalte under observasjonene at noen pasienter mistet tillit til pleierne da det ringte i lommen deres uten at de besvarte anropet. Ved de to andre avdelingene brukes telefonen i større grad slik den er tenkt, men også der trekkes støy frem som en vesentlig utfordring. P2-A2 beskrev opplevelsen av varslinger på telefon slik: \textit{ $"$...det med telefon og sånn, det synes jeg [er] veldig urolig, urolig hverdag, bråkete. Særlig hvis du står i stressende, pressende situasjoner så ringer det, og ringer, og ringer, og ringer...$"$}. 

\noindent
Hun fortalte videre at pasienter vegrer seg for be om assistanse hvis de tror sykepleierne har mye å gjøre:

\noindent
\textit{... [pasientene sier] «oi, vi hører dere har det travelt». Men det trenger ikke være tilfellet. Så ringer dem ikke på. [...] Men vi trenger ikke ha det travelt selv om det ringer, for det kan jo ringe fra de andre tunene. Jeg får jo inn det på min telefon også. Det kan jo være en der borte som ligger på klokken hele dagen, og det vil jo ikke si at jeg har det travelt.}

\noindent
Til tross for at flere pleiere uttrykte misnøye med lydnivået på telefonene, poengterte de samtidig at det gir nødvendig informasjon, og en indikasjon på hvor mye det er å gjøre på avdelingen. Tidligere kunne lyden på telefonen skrues av, men dette skapte en risiko for at signaler ikke ble oppdaget. Lydnivået kan per i dag derfor ikke skrues lavere enn nivå to. Under observasjon ved avdeling A2 ble forskerne fortalt at noen pleiere valgte å teipe over høyttaleren på telefonen for å dempe lyden. Et annet eksempel ble gitt under intervjuet med P1-A2, som fortalte om en pleier som hadde pakket telefonen inn i bobleplast.

\noindent
Pleiere fra alle tre avdelinger påpekte at støy fra telefonen er et problem om natten i tilfeller hvor sykepleieren er inne hos en pasient når det utløses et signal. P2-A2 ser nytten av å gå med telefonen på nattevakt, da hun ønsker å være tilgjengelig for kolleger og pasienter som trenger hjelp. Samtidig uttrykte hun at det er utfordringer knyttet til dette: \textit{ $"$... så har jeg jo opplevd at pasienter våkner og ikke får sove igjen. Og det er jo en bakdel$"$}. Hun fortalte videre at pleiere på avdelingen derfor iblant legger igjen telefonen utenfor rommet for å unngå dette problemet. For å ikke være fullstendig utilgjengelig velger de fleste likevel å tilstedemarkere seg, da varslingen på rompanelene ikke oppleves som like forstyrrende som den på telefonen. For avdeling A2 og A3 vil det være svært alvorlig å være utilgjengelig på telefon, da disse avdelingene ikke mottar signaler fra alle tun på sengeposten på panelene. Dette gjelder også på dagtid, og er dermed et av de mest fremtredende argumentene for å motta pasientsignaler på telefon. Forskerne observerte imidlertid at pleierne i flere tilfeller ikke tilstedemarkerte seg på rom. Gjennom intervjuene ble det klart at det varierer hvorvidt dette er et bevisst valg, eller en forglemmelse. Pleierne ser derfor fordelen av å motta signaler på telefon, da dette oppleves som en sikkerhet i de tilfeller hvor de glemmer å markere seg. Dette gjelder også i tilfeller hvor de befinner seg på steder uten veggpaneler, eksempelvis på møterom eller utenfor avdelingen.

\noindent
Både pleiere og seksjonsledere fra avdelingene A1 og A3 trekker frem telefonens grensesnitt som lite brukervennlig, og beskriver den som tungvint å bruke. L-A1 uttrykte seg slik:

\noindent
\textit{ $"$... det jeg reagerer på når det gjelder IP-telefonen, er at man må sende en hel avdeling på kurs i flere timer for å forstå en telefon. [...] Det er ressurskrevende, og for komplisert. [...] [Den har funksjoner] som er veldig tungvinte, og det er veldig lite selvforklarende, veldig lite brukervennlig, og skiller seg vesentlig fra mobiltelefoner...$"$}.

\noindent
Et interessant funn gjort under observasjonene var at pleierne i relativt liten grad interagerer med telefonene, men heller ser på veggpanelene ved utløste pasientsignaler. Under intervjuene ble det avdekket flere årsaker til dette. For det første vil forsinkelsen i det trådløse systemet føre til at signalet varsles tidligere på panelene enn på telefonene. For det andre er panelene plassert slik at det ofte vil være enklere å se på disse enn å ta telefonen opp fra lommen, samtidig som det i andre tilfeller vil være problematisk å håndtere telefonen, eksempelvis av hygieniske årsaker, eller i situasjoner der det kan oppleves som forstyrrende for pasienten. Dette resulterer i at funksjonaliteten for å bekrefte eller avvise signalet ikke brukes, og dermed heller ikke arbeidslisten. Dette fører til unødvendig støy, da signalet varsles i 15 sekunder før det går videre til neste mottaker. Imidlertid er det flere som påpeker at panelene burde vært større, da det kan være vanskelig å lese fra avstand. Ved avdeling A1 er vaktromsapparatet på det ene sengetunet plassert bak en dør som vanligvis står åpen. Både L-A1 og pleiere fra avdeling A1 ønsker seg et panel i taket, slik de hadde før. Dermed kan de enkelt se hvor signalene er utløst fra uten å måtte gå bort til et veggpanel. P1-A1 kunne i tillegg tenke seg ulike varslingslyder for signaler fra de forskjellige tunene, slik at pleierne enkelt hører om signalet er utløst på deres tun.

\noindent
En annen tydelig forskjell mellom avdelingene er at avdeling A1, som eneste avdeling, kan utløse assistansesignal. Forskerne observerte at dette er en funksjon de bruker mye og er svært fornøyd med, noe pleierne også bekreftet under intervjuene. Denne funksjonaliteten eksisterte ved det gamle sykehuset, men ble ikke videreført i det nye. Avdelingen fikk likevel gjeninnført denne funksjonaliteten da de flyttet inn i nye lokaler høsten 2013, og pleierne beskriver den som noe de \textit{$"$kjempet$"$} for, og \textit{$"$forlangte$"$} å få. Under observasjonene fortalte også en av pleierne fra avdeling A2 at hun savnet assistanseknappen. Alternativet for avdelingene A2 og A3 er å utløse stansalarm, noe som ikke er ønskelig med mindre det er svært alvorlige situasjoner. Forskerne observerte at pleierne i noen tilfeller heller kom ut fra pasientrommet for å spørre om hjelp.

\noindent
Pleierne fra avdelingene A2 og A3 opplever at pasientsignalsystemet fungerer slik det er i dag, men påpeker tekniske feil og mangler. De opplever blant annet det de kaller \textit{$"$spøkelsesalarmer$"$}, signaler som ingen har utløst, eller som er utløst fra rom de ikke kjenner til. Forsinkelsen fra det faste til det trådløse systemet fører til at varslingen på telefonen fortsetter etter at en pleier har tilstedemarkert seg på et pasientrom, noe som skaper irritasjon blant pleierne. P1-A3 beskrev det slik: \textit{$"$...så fortsetter den her å pipe i inntil 1-1,5 minutt etterpå, og det er jo enormt irriterende. At når du allerede har utført jobben, så får du fortsatt beskjed om at jobben ikke er påstartet.$"$} Dette ble også observert av forskerne, og forsinkelsen førte i noen tilfeller til at pleiere kom for å besvare signalet selv etter at dette var gjort. L-A3 påpeker at dette kan føre til at pleierne blir \textit{$"$immune$"$} mot varslingene på telefonen, noe som også ble bekreftet av P2-A3:

\noindent
\textit{$"$... det er jo litt feilmeldinger og sånn på det da. [...] Det er jo ikke noe særlig. Det blir sånn ulv ulv nesten... $"$}

\section{Individ}
Et individ beskrives av FITT-rammeverket som enkeltbrukere eller brukergrupper av systemet, og er de som utfører alle oppgaver på arbeidsplassen. Egenskaper ved individet kan være IT-kunnskap, motivasjon, åpenhet for endring i arbeidsmåte og teamkultur.

\noindent
Observasjonene og intervjuene avdekket et bredt spekter av holdninger til systemet. På avdelingene A2 og A3 brukes pasientsignalsystemet i stor grad slik det er tenkt, og seksjonslederne og pleierne opplever at det er et nyttig verktøy. Avdeling A1 har derimot opplevd større motstand mot systemet, og anvender kun deler av det.

\noindent
I overgangen fra det gamle til det nye sykehuset forsøkte avdeling A1 på lik linje med andre avdelinger å benytte telefonen slik den er tenkt. Det ble derimot tidlig et problem at telefonene skapte mye støy og var lite brukervennlige, noe som skapte stor frustrasjon. P1-A1 fortalte under intervjuet at de ikke var forberedt på at flyttingen skulle medføre så store endringer i systemet. Han opplevde at de hadde et velfungerende system, og antok at dette ville bli videreført. Med innføringen av IP-telefoner, samt bortfall av assistanseknappen og panelet i taket, oppsto det derimot negative holdninger til det nye pasientsignalsystemet. Da avdelingen igjen flyttet inn i nye lokaler hadde de en prøveperiode på 14 dager, hvor de på nytt forsøkte å bruke systemet som tenkt. De fikk samtidig tilbake assistanseknappen de lenge hadde ønsket seg. P2-A1 fortalte at pleierne i denne perioden gjorde et forsøk på å få systemet til å fungere, men sa samtidig at de visste at det var en prøveperiode, og at de ikke brukte det optimalt da de ofte ikke godtok signalet på telefonen, men lot det gå videre til neste mottaker. Etter prøveperioden gikk avdelingen tilbake til å ikke motta pasientsignaler på telefon. Både L-A1 og pleierne på avdelingen uttrykte tydelig at de ikke ser nytten av slik bruk. De argumenterer med at de like gjerne kan se på panelene, og at de ofte står i situasjoner hvor de ikke har mulighet til å håndtere telefonen. L-A1 forklarte det slik: 

\noindent
\textit{$"$... en ting er problemene med systemet, det kan man alltids finne løsninger på. Men når det i tillegg ikke er gevinst med noe, da er det vanskelig å få gjennom noe som i utgangspunktet er et problem, det er et system med bare ulemper og ingen fordeler. Det er veldig vanskelig å få gjennom en sånn ting. Hvis vi hadde hatt fordeler med det, så kan vi leve med ulempene. Men å få gjennom noe med bare ulemper, det er vanskelig.$"$}

\noindent
Forskerne avdekket videre at endringer i lyd ikke vil endre avdelingens innstilling til å motta pasientsignaler på telefon. Dette ble tydelig understreket av P2-A1 som sa:

\noindent
\textit{$"$... jeg forstår ikke hvorfor vi skal ha det på telefonen. [...] Hvorfor du skal ned i en lomme for å se hvorfor det ringer liksom. Da må vi jo ha tilgjengelige skjermer, tilgjengelig i tak som sagt, tilgjengelig panel.$"$}

\noindent
P1-A1 ønsker å ha fullt fokus på pasientene, og beskrev sitt forhold til å gå med telefon på seg slik:

\noindent
\textit{$"$... altså når jeg går inn på et pasientrom så er det pasienten jeg har i fokus. Og da er det kun pasienten, og det som skjer utenfor det pasientrommet det blåser jeg i, for at det er det andre mennesker som tar seg av.$"$}

\noindent
Selv om de ikke bruker systemet slik det er tenkt fra produsentens side, opplever avdelingen at de har en velfungerende løsning. L-A1 påpekte imidlertid at det ikke er en god situasjon at de ikke bruker det slik de er pålagt.

\noindent
For avdeling A2 og A3 uttrykte pleierne en generelt mer positiv holdning til systemet, til tross for at holdningene varierte også her. Pleier P1-A2 beskriver det som et $"$kjempeverktøy$"$, mens P2-A2 opplever det til tider som stressende å motta pasientsignaler på telefon. P2-A3 så derimot ikke på dette som et problem: 

\noindent
\textit{Det er veldig greit at du kobler deg på rommene som du har, og at det ringer først til deg. Og at jeg kan avvise det. Det synes jeg er greit.}

\noindent
Som tidligere påpekt påvirker avdelingenes tekniske utforming hvordan sykepleierne bruker, og logger seg på systemet. På alle tre avdelinger fordeler pleierne på hvert sengetun ansvaret for pasientrommene mellom seg. Systemet er tenkt slik at pleierne skal logge seg på som primær på sine rom og som disp på andre, og dermed motta pasientsignaler fra disse på sin telefon. Det ble likevel avdekket avvik fra dette, da noen kun logger seg på som primær, og andre kun setter seg som disp på tunet. På kvelds- og nattskift er det få pleiere på jobb, og det er vanlig at de kun setter seg som disp. For å motta pasientsignaler fra andre tun logger pleierne på A2 seg på som disponible på disse. På A3 gjøres dette kun på nattskift. 

\noindent
Gjennom intervjuene ble forskerne gjort oppmerksomme på at avdelingene selv står for opplæring av nyansatte. Seksjonslederne og pleierne er åpne om at dette fører til at holdninger og rutiner videreføres, som beskrevet av P1-A2:

\noindent
\textit{$"$... det er sikkert individuelle forskjeller [...] hvor nøye man er, også er det sikkert litt i forhold til opplæring. Er det dårlige vaner på en avdeling, og man har opplæring så blir det kanskje til at man lærer det videre litt ubevisst. Som ny er man jo veldig var på hva slags holdninger og sånn de har de man går med, er man litt sløv og bruker det feil så er det kanskje det man lærer bort også.$"$}

\noindent
Selv om pleierne på avdelingene A2 og A3 stort sett er fornøyde med systemet ble det påpekt utfordringer. P1-A2 fortalte at det kan være en risiko at de manuelt må sette seg som disp for å motta signaler fra andre tun, da det kan være nyansatte som ikke er klar over dette. Pleierne gjør kontinuerlige vurderinger av hvilke oppgaver de skal utføre når, og hvordan de skal håndtere innkommende pasientsignaler. Til tross for at pleierne både under observasjonene og intervjuene ga uttrykk for at de har et felles ansvar for pasientene, fortalte P2-A2 at rollene primær og disp medfører en utfordring:

\noindent
\textit{$"$... det som jeg ser på som er det negative det er vel på en måte at vi setter oss opp på $"$våre pasienter$"$, så hvis jeg er opptatt med en av mine pasienter og det ringer på på de to andre mine pasienter så blir ikke klokkene tatt. De bare avslutter eller kjører telefonen videre, mange ganger, og det synes jeg er feil.$"$}

\noindent
På alle tre avdelinger er det variasjon i hvor lenge pasientene ligger inne, og dermed hvor godt pleierne kjenner dem. De fleste gir likevel uttrykk for at de normalt har tilstrekkelig kunnskap om pasientenes tilstand til å kunne vurdere signalenes hastegrad. Under observasjonene fortalte flere pleiere at pasientene har ulik terskel for å utløse signaler. Mens noen kun gjør det ved alvorlige hendelser, utløser andre signaler oftere. Forskerne observerte også at sykepleierne iblant gjør avtaler med pasientene om å utløse signal ved spesifikke hendelser, og da vet hva signalet gjelder. Til tross for at sykepleierne ga uttrykk for at de besvarer pasientsignalene raskt uansett pasient, observerte forskerne ved flere tilfeller at signaler ringte opp til flere minutter før de ble besvart.  
  
\noindent
L-A2: \textit{$"$... den kulturen med å ta klokker da, den er veldig forskjellig fra sengepost til sengepost, og jeg tror at der det er mest rolig, der sykepleierne antar at det ikke haster når en pasient ringer, så er det nok en kultur for at man kan la det ringe lenge. Og hvis man lar det ringe lenge blir det veldig mye støy – for alle. Og da tror jeg motivasjonen for å bruke systemet blir ganske lav. Nettopp av den grunn.$"$} 

\section{Oppgave}
En oppgave beskrives av FITT-rammeverket som helheten av oppgaver og arbeidsprosesser som må utføres av brukeren med den gitte teknologi. Egenskaper ved oppgaven kan blant annet være organisering av oppgavene, gjensidig avhengighet mellom oppgavene og oppgavens grad av kompleksitet.

\noindent
Felles for alle tre avdelinger er at sykepleierne utfører både planlagte og tilfeldige oppgaver. Medisinering, stell og matservering er eksempler på førstnevnte, og disse skjer normalt til faste tider. Et eksempel på sistnevnte er pasientsignaler, som ofte utløses uten forvarsel.  

\noindent
Flere oppgaver innebærer bruk av systemet. Pleierne fortalte under intervjuene at de ved starten av hvert skift logger seg på telefonen, og de anser dette som en vane. Det var imidlertid ulike synspunkt på hvorvidt det å logge seg på rom er en rutine. Forskerne observerte ved noen tilfeller at pleiere fra tidligere skift fremdeles var logget på som ansvarlige for pasientrom. Pleierne forklarte at de iblant glemmer å logge seg på rom, spesielt hvis det er mye å gjøre. som beskrevet av P2-A2:

\noindent
\textit{... vi er en ganske hektisk avdeling, [så om morgenen] så kan det jo eksplodere her, og da er det kanskje fåtallet av oss... De fleste har vel logget på telefonen, men vi har jo ikke logget oss på systemet, og da [om det eksploderer og vi er i jobb alle mann] tenker vi ikke over det, før kanskje langt utpå formiddagen, at oi, her er det noe som har gått oss hus forbi, så kan vi se at det er nesten ingen som har logget seg på systemet.}

\noindent
P1-A3 fortalte derimot: $"$... det er noe alle gjør. Det er inne, det er rutine nå. Det er ikke noe problem.$"$ 

\noindent
For å motta pasientsignaler på rompanelet, og for å formidle sin tilstedeværelse til kolleger, skal sykepleierne tilstedemarkere seg på pasientrom. I tilfeller hvor en sykepleier går inn til en pasient ved utløst pasientsignal, trykker pleierne alltid på tilstedeknappen, da dette avstiller signalet. Hvis det derimot ikke er utløst et signal ble det under observasjonene avdekket store variasjoner i hvorvidt tilstedeknappen benyttes. Pleierne forklarte at dette gjøres oftere i tilfeller hvor de vet at oppgaven på pasientrommet tar tid, mens de ved kortere besøk heller velger å la døren stå på gløtt. Dette ble bekreftet av P2-A3 under intervjuet: 

\noindent
\textit{$"$... noen er vel litt sløve med å logge seg inn på rom da. Å trykke på grønnlyset og sånt. Kanskje litt flinkere til å gjøre det når vi vet at vi skal stelle og sånt. At vi blir der en stund. Men hvis vi bare skal inn å levere medisiner eller noe sånt så, blir det ikke brukt noe grønnlys.$"$}

\noindent
I flere tilfeller observerte forskerne at pleierne glemte å trykke på rødknappen for å fjerne tilstedemarkeringen når de forlot rommet. Dette medfører at signaler varsles inne på pasientrom uten at det er sykepleiere tilstede.

\noindent
På avdeling A1 har det etter innflytting i nye lokaler vært fokus på at pleierne må tilstedemarkere seg, da de ellers ikke vil bli varslet om utløste pasientsignaler og stansalarmer. Forskerne la imidlertid ikke merke til at dette ble gjort i større grad her enn på de andre avdelingene.  

\section{Oppsummering}
Den tydeligste forskjellen mellom avdelingene er den negative holdningen A1 har til bruk av telefonen for mottak av pasientsignal, som har resultert i at de ikke bruker denne funksjonen. De mest fremtredende årsakene til dette er støy, 











