\section{Redundans}
\label{sec:redundans}

Informasjon som gjentar allerede etabler kunnskap uten å tilføre noe nytt kalles redundans. Ordet betyr noe som er overflødig, og brukes gjerne om reserveinnrettninger som kan overta en oppgave om noe annet svikter \citep{Rosness}.  \citet{Cabitza05} skiller mellom tre typer redundans. 

\begin{itemize}
\item Redundans av funksjon innebærer at flere individer har overlappende funksjoner eller ferdigheter, slik at flere kan utføre samme oppgave. En slik overlapping kan også gi organisasjonen økt robusthet og kapasitet.
\item Redundans av innsats oppstår når allerede utførte oppgaver blir utført på ny. Dette kan i noen tilfeller brukes som en strategi for å forbedre effektivitet eller sikkerhet \citep{Rygh}
\item Redundans av data betyr at lik eller lignende informasjon finnes på flere steder, eksempelvis både digitalt og på papir eller på flere digitale enheter. Dette brukes gjerne i systemer hvor en vil forsterke pålitelighet og toleranse for feil. Når det anbefales å ta backup av dine bilder på PC, for så å lagre disse på en egen ekstern harddisk er dette eksempel på etablering av redundans av data. \citep{Rygh}.
\end{itemize}

\citet{Rosness} legger til en type han kaller organisatorisk redundans, som skapes når personer rådfører seg med hverandre, sjekker hverandre og korrigerer hverandre. Et eksempel her kan være et fly med to piloter som kan overvåke hverandre og stille kritiske spørsmål. 

\noindent
Redundans oppfattes ofte som ineffektivt og kostbart \citep{Rygh}, men \citet{Cabitza05} påpeker at en økning i redundans av data kan føre til en reduksjon i redundans av innsats, og dermed mer effektiv koordinasjon av arbeid. Redundans kan dermed også være positivt, og kan i tillegg føre til økt sikkerhet og pålitelighet \citep{Rygh}.

\noindent
\citet{Rosness} understreker at det er forskjell på teknisk og organisatorisk redundans. I teknisk redundans er det gjerne tilstrekkelig å duplisere identiske komponenter. I organisatorisk redundans, for å unngå to personer med samme bakgrunn og erfaring som vil begå de samme feilene er det ofte en forutsettning med diversitet, personer med forskjellig bakgrunn, eller som har litt ulike roller.
