\section{Redundans}
\label{sec:redundans}

Informasjon som gjentar allerede etablert kunnskap uten å tilføre noe nytt kalles redundans. Ordet betyr noe som er overflødig, og brukes gjerne om reserveinnretninger som kan overta en oppgave om noe annet svikter \citep{Rosness}. \citet{Cabitza} skiller mellom tre typer redundans: 

\begin{itemize}
\item Redundans av funksjon; innebærer at flere individer har overlappende funksjoner eller ferdigheter, slik at flere kan utføre samme oppgave. Et eksempel på dette er at flere pleiere kan behandle samme pasient. En slik overlapping kan gi organisasjonen økt robusthet og kapasitet.
\item Redundans av innsats; oppstår når allerede utførte oppgaver blir utført på ny. Dette kan i noen tilfeller brukes som en strategi for å forbedre sikkerhet \citep{Rygh13}
\item Redundans av data; betyr at lik eller lignende informasjon finnes på flere steder.  Dette brukes gjerne i systemer hvor en vil forsterke pålitelighet og toleranse for feil, eksempelvis ved at pasientsignaler varsles både gjennom det faste og det trådløse systemet. 
\end{itemize}
 
\noindent
Redundans oppfattes ofte som ineffektivt og kostbart \citep{Rygh13}, men \citet{Cabitza} påpeker at en økning i redundans av data kan føre til en reduksjon i redundans av innsats, og dermed mer effektiv koordinasjon av arbeid. Redundans kan dermed også være positivt, og kan i tillegg føre til økt sikkerhet og pålitelighet \citep{Rygh13}.
