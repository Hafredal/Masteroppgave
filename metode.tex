\chapter{Metode}
\label{chp:metode} 

Arbeidet med denne oppgaven har vært tredelt. Forskerne fikk i sitt arbeid med fordypningsprosjektet, høsten 2013, god kjennskap til hvordan pasientsignalsystemet var \textit{tenkt} brukt. Da denne masteroppgaven allerede var godkjent av NSD \citep{NSD} og REK \citep{REK}, hadde forskerne større frihet til å gå ut i feltet. For å få en større forståelse av sykepleiernes \textit{reelle} arbeidspraksis, ble det derfor utført observasjoner ved tre ulike avdelinger ved St. Olavs Hospital. I løpet av denne observasjonsperioden ble det avdekket tydelige variasjoner i arbeidspraksis, og forskningsområdet ble dermed avgrenset til å omhandle disse. Det ble deretter utført nye observasjoner med spisset fokus. På bakgrunn av erfaringene gjort under observasjonsperioden ble det utarbeidet intervjuguider, og avslutningsvis ble det gjennomført intervjuer med seksjonsledere og pleiere fra de observerte avdelingene. I dette kapittelet vil valg av forskningsmetoder bli belyst. Resultatene av disse vil bli presentert i kapittel \ref{chp:resultater}, og deretter diskutert i kapittel \ref{chp:diskusjon}.


\section{Hensikt og foskningsspørsmål}
Motivasjonen for oppgaven har vært å kartlege sykepleiernes anvendlese av systemet, og identifisere forskjeller i bruk. Videre ønsket vi å svare på hva som kan være årsaker til at disse forskjellende har oppstått. 

\noindent
Da vi startet arbeidet med denne oppgaven var tanken at vi skulle se på hvordan systemet kunne endres for å bedre møte sykepleiernes behov. Etter første observasjonsrunde så vi imidlertid at en mer interessant vinkling ville være å se på hvordan systemet blir brukt forskjellig i, og mellom de forskjellige avdelingene, og hva de bakenforliggende årsakende til dette kan være. Dette resulterte i to forskningsspørsmål:

\begin{enumerate}
\item Hvordan brukes pasientsignalsystemet ved St.Olavs hospital forskjellig i, og mellom ulike avdelinger? 
\item Hvilke faktorer kan være årsak til disse forskjellene?
\end{enumerate}

\noindent
For å besvare disse spørsmålene har vi gjennomført både observasjoner og semistrukturerte intervjuer ved tre avdelinger ved St.Olavs hospital. Dataene fra disse innsamlingene ble så analysert og valg av relevent teori ble gjort i tråd med stegvis-deduktiv induktiv metode.
