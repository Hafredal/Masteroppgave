\chapter{Metode}
\label{chp:metode} 

Arbeidet med denne oppgaven har vært tredelt. Forskerne fikk i sitt arbeid med fordypningsprosjektet, høsten 2013, god kjennskap til hvordan pasientsignalsystemet var \textit{tenkt} brukt. Da denne masteroppgaven allerede var godkjent av NSD \citep{NSD} og REK \citep{REK}, hadde forskerne større frihet til å gå ut i feltet. For å få en større forståelse av sykepleiernes \textit{reelle} arbeidspraksis, ble det derfor utført observasjoner ved tre ulike avdelinger ved St. Olavs Hospital. I løpet av denne observasjonsperioden ble det avdekket tydelige variasjoner i arbeidspraksis, og forskningsområdet ble dermed avgrenset til å omhandle disse. Det ble deretter utført nye observasjoner med spisset fokus. På bakgrunn av erfaringene gjort under observasjonsperioden ble det utarbeidet intervjuguider, og avslutningsvis ble det gjennomført intervjuer med seksjonsledere og pleiere fra de observerte avdelingene. I dette kapittelet vil valg av forskningsmetoder bli belyst. Resultatene av disse vil bli presentert i kapittel \ref{chp:resultater}, og deretter diskutert i kapittel \ref{chp:diskusjon}.





