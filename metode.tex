\chapter{Metode}
\label{chp:metode} 

Det finnes flere måter å klassifisere og karakterisere ulike former for forskning, men en av de vanligste er å skille mellom \textit{kvalitative} og \textit{kvantitative} forskningsmetoder \citep{Myers13, Tjora}. Kvalitative metoder er velegnet dersom hensikten er å forstå sosiale og kulturelle aspekter ved mennesker og organisasjoner, hvis forskningen er utforskende, eller man ønsker å studere et spesielt emne i dybden \citep{Myers13}. En slik tilnærming vil i større grad avdekke i hvilken \textit{kontekst} en handling eller avgjørelse blir utført, og kan derfor hjelpe forskere å forstå hvorfor informantene handler som de gjør. Med et sosioteknisk utgangspunkt, var det naturlig å benytte forskningsmetoder som genererte kvalitative data i arbeidet med denne studien.

\noindent
Forskerne fikk i sitt arbeid med fordypningsprosjektet, \citep{Sund13}, god kjennskap til hvordan pasientsignalsystemet var \textit{tenkt} brukt. Da denne masteroppgaven var godkjent av NSD\footnote{Norsk samfunnsvitenskapelig datatjeneste \citep{NSD}.} og REK\footnote{Regionale komiteer for medisinsk og helsefaglig forskningsetikk \citep{REK}.}, hadde forskerne større frihet til å gå ut i feltet. For å få en større forståelse av sykepleiernes \textit{reelle} arbeidspraksis, ble det derfor utført observasjoner ved tre ulike avdelinger ved St. Olavs Hospital. I løpet av denne observasjonsperioden ble det avdekket tydelige variasjoner i arbeidspraksis, og forskningsområdet ble dermed avgrenset til å omhandle disse. Det ble deretter utført nye observasjoner med spisset fokus. På bakgrunn av erfaringene gjort under observasjonsperiodene ble det utarbeidet intervjuguider, og avslutningsvis ble det gjennomført intervjuer med seksjonsledere og pleiere fra de observerte avdelingene. Dette utgjorde en triangulering av forskningsmetoder, som styrket resultatenes kvalitet.

\noindent
I dette kapittelet vil valg av forskningsmetoder bli belyst. Resultatene av disse vil bli presentert i kapittel \ref{chp:resultater}, og deretter diskutert i kapittel \ref{chp:diskusjon}. 

\section{Hensikt og forskningsspørsmål}
Med bakgrunn i forskernes prosjektoppgave \citep{Sund13}, var utgangspunktet for dette arbeidet å videre undersøke hvordan systemet kan endres for å bedre møte sykepleiernes behov. Det ble derimot tidlig avdekket tydelige ulikheter i sykepleiernes anvendelse av systemet, og forskerne valgte å vinkle forskningsspørsmålene annerledes. Motivasjonen for forskningsarbeidet ble dermed å kartlegge sykepleiernes anvendelse av pasientsignalsystemet, og å identifisere ulikheter i bruk. Videre ønsket forskerne å svare på hva som kan være årsaker til disse forskjellene. Dette resulterte i to forskningsspørsmål:

\begin{enumerate}
\item Hvordan brukes pasientsignalsystemet ved St.Olavs Hospital forskjellig i, og mellom ulike avdelinger? 
\item Hvilke faktorer kan være årsak til disse forskjellene?
\end{enumerate}

\noindent
For å besvare disse spørsmålene gjennomførte forskerne observasjoner og intervjuer ved tre avdelinger ved St.Olavs Hospital. Dataene fra innsamlingen ble deretter analysert, og relevant teori ble valgt i tråd med stegvis-deduktiv induktiv metode. Forskningsmetodene som ble anvendt belyses videre i dette kapittelet.