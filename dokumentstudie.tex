\section{Dokumentstudie}
\label{sec:dokumentstudie}
Ifølge \citet{Tjora} er dokumentene som brukes i et dokumentstudie i utgangspunktet skrevet for andre formål enn forskning. Slike dokumenter fungerer gjerne som bakgrunns- eller tilleggsdata, og kan være casespesifikke eller generelle. Dokumentene kan være eneste kilde til empiri eller kun fungere som tilleggsdata. 

\noindent
Dokumenter har i dette forskningsarbeidet blitt brukt både som bakgrunns- og tilleggsdata. Eksempler på førstnevnte har vært offentlige dokumenter som omhandler utbyggingen av det nye sykehuset, eksempelvis utforming av fysisk arkitektur \citep{Aslaksen, Sintef-sengetun} og teknisk infrastruktur \citep{TU}, noe som ga forskerne en dypere forståelse av utbyggingens omfang. Dokumenter som er brukt som tilleggsdata omfatter brukerveiledninger til pasientsignalsystemet, samt strategidokumenter og informasjon hentet fra St. Olavs Hospitals hjemmesider \citep{BrukermanualforPasientsignalogPasientsignalapplikasjon, BrukerveiledningforPasientsignal, BrukerveiledningforTradlostelefon, styring13, stolavs}. 

\noindent
Opplæringsmaterialet ble studert for å få en tydelig forståelse av hvordan systemet er tenkt brukt, samtidig som det var av relevans for forskningsområdet å identifisere hvilket opplæringmateriell som er tilgjengelig for sykepleierne.