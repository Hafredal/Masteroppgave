\subsection{Dokumentstudie}
\label{sec:dokumentstudie}
Ifølge \citet{Tjora} er dokumentene som brukes i et dokumentstudie i utgangspunktet skrevet for andre formål enn forskning. Slike dokumenter fungerer gjerne som bakgrunns- eller tilleggsdata, og kan være casespesifikke eller generelle. Dokumentene kan være eneste kilde til empiri, eller kun fungere som tilleggsdata. 

\noindent
Dokumenter har i dette forskningsarbeidet blitt brukt både som bakgrunns- og tilleggsdata. Eksempler på førstnevnte har vært offentlige dokumenter som omhandler utbyggingen av det nye sykehuset, eksempelvis utforming av fysisk arkitektur \citep{Aslaksen, Sintef-sengetun} og infrastruktur \citep{TU}. Dette ga forskerne en dypere forståelse av utbyggingens omfang, og hvordan sykehuset er tenkt å virke. Dokumenter som er brukt som tilleggsdata omfatter brukerveiledninger til pasientsignalsystemet, samt strategidokumenter og informasjonssider tilhørende St. Olavs Hospital. 

\noindent
Opplæringsdokumenter som er studert er hentet fra sykehusets hjemmesider, og inkluderer:
\begin{itemize}
\item Brukerveiledning for pasientsignal \citep{BrukerveiledningforPasientsignal}
\item Brukermanual for pasientsignal \& pasientsignalapplikasjon \citep{BrukermanualforPasientsignalogPasientsignalapplikasjon}
\item Brukermanual for trådløs telefon \citep{BrukerveiledningforTradlostelefon}
\end{itemize}

\noindent
Opplæringsmaterialet ble studert for å få en tydelig forståelse av hvordan systemet er tenkt brukt, samtidig som det var av relevans for forskningsområdet å undersøke hvilken opplæring sykepleierne har tilgjengelig i bruk av systemet.
