\subsection{Intervju}
\label{sec:intervju}

Intervju er den mest utbredte datagenereringsmetoden innen kvalitativ forskning, med den hensikt å studere informantenes \textit{subjektive} meninger, holdninger og erfaringer. Innen SDI-modellen har intervjuet som mål å generere refleksjon blant deltagerne \citep{Tjora}.  


Strukturerte intervjuer minner om muntlige spørreundersøkelser, hvor intervjueren spør forhåndsbestemte, identiske spørsmål ved hvert intervju. Semi-strukturerte intervjuer tillater  derimot større frihet, eksempelvis ved at spørsmål legges til, eller stilles i en annen rekkefølge enn tenkt. 



\subsubsection{Intervjuguide}
Observasjonsperioden avdekket tydelige forskjeller i sykepleiernes arbeidspraksis, både internt og mellom de ulike avdelingene. Intervjuenes hensikt var dermed å avdekke faktorer som kunne forklare disse forskjellene fra informantenes perspektiv. 

\subsubsection{Intervjuobjekter og utførelse}



Det ble gjennomført semi-strukturerte intervjuer med seksjonsledere og pleiere fra hver observerte avdeling, se tabell \ref{detaljerintervju} for detaljer. Det ble utarbeidet to intervjuguider, en for seksjonsledere og en for pleiere, se tillegg \ref{chp:appendix_intervjuguide_seksjonsledere} og \ref{chp:appendix_intervjuguide_pleiere}.



 På forhånd ble intervjuenes tidsramme estimert til 30 minutter, men intervjuenes lengde ble tilpasset hver informant, avhengig av hvor mye de ønsket å fortelle. Intervjuenes varighet varierte mellom 15-20 minutter for pleierne, og 30-40 for seksjonslederne. 


\begin{table}[H]\centering
    \begin{tabular}{ |l|l|l|l|l|l| }
    \hline
    Intervju & Avdeling & Intervjuobjekt \\ \hline
       I1 & A2 & Seksjonsleder \\ \hline
       I2 & A2 & Sykepleier \\ \hline
       I3 & A2 & Sykepleier \\ \hline
       I4 & A1 & Assisterende seksjonsleder \\ \hline
       I5 & A1 & Hjelpepleier \\ \hline
       I6 & A3 & Seksjonsleder \\ \hline
       I7 & A3 & Sykepleier \\ \hline
       I8 & A3 & Sykepleier \\ \hline
    \end{tabular}
    \caption {Detaljer for intervjuer}
    \label{detaljerintervju}
\end{table}
 






- rolle
- tid og sted
- lydopptak
