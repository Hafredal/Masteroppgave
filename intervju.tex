\subsection{Intervju}
\label{sec:intervju}
 
Intervju er den mest utbredte datagenereringsmetoden innen kvalitativ forskning, og har som hensikt å forstå informantens \textit{subjektive} meninger, holdninger og erfaringer \citep{Tjora}. \textit{Kvalitative intervjuer} beskriver en rekke fremgangsmåter som strekker seg fra å være ustrukturerte og åpne, til strukturerte og direkte \citep{Smith12}. Sistnevnte minner om muntlige spørreundersøkelser, hvor intervjueren spør forhåndsbestemte, identiske spørsmål ved hvert intervju. Denne intervjuformen egner seg best i situasjoner hvor man har god kjennskap til fenomenet som studeres, og har tilgang til et stort antall informanter. Semistrukturerte intervjuer tillater større frihet og fleksibilitet, eksempelvis ved at spørsmål legges til, eller stilles i en annen rekkefølge enn tenkt \citep{Tjora}.
 
\noindent
Observasjonsperioden avdekket tydelige forskjeller i sykepleiernes arbeidspraksis, både internt og mellom ulike avdelinger. Intervjuenes hensikt var dermed å avdekke faktorer som kunne forklare disse forskjellene fra informantenes perspektiv. Diktafon ble brukt under intervjuene, med informantenes samtykke, se tillegg \ref{chp:appendix_informasjon_intervju} for informasjonsskriv. Lydfilene ble i etterkant transkribert for videre analyse.
 
\subsubsection{Intervjuguide}
Basert på funnene avdekket under observasjonsperioden, ble det utarbeidet intervjuguider, hvor målet var å utdype, sammenligne og eventuelt avdekke sprik mellom forskernes og informantenes oppfatninger. Forskerne valgte en semistrukturert tilnærming, hvor temaene diskutert under intervjuene ble holdt innenfor rammen av forskningsområdet, samtidig som informantene fikk frihet til å reflektere over, og dele den informasjon de anså som relevant.
I semistrukturerte intervjuer kan spørsmålene være formulerte eller stikkordsbaserte, men med åpne svar som utdypes etter intervjuerens eller informantens skjønn \citep{Schensul99}. Forskerne valgte å formulere fullstendige spørsmål, med stikkord som underpunkter. Det ble utarbeidet to intervjuguider, en for seksjonsledere og en for pleiere, se tillegg \ref{chp:appendix_intervjuguide_seksjonsledere} og \ref{chp:appendix_intervjuguide_pleiere}. Disse ble utformet etter de tre fasene beskrevet av \citet{Tjora} - oppvarming, refleksjon og avrunding. Intervjuet ble delt i fire deler med ulike temaer, hvor spørsmålene innen hvert tema også fulgte disse tre fasene.
 
\subsubsection{Intervjuobjekter og utførelse}
Det ble gjennomført intervjuer med seksjonsleder og tre pleiere fra hver observerte avdeling, se tabell \ref{detaljerintervju} for detaljer. Intervjuene ble utført på avdelingene, og informantene ble valgt blant de pleierne som var på jobb av seksjonsleder, uten at forskerne hadde satt kriterier til pleiernes egenskaper eller erfaringer. Arbeidserfaring ble avdekket under intervjuene, og forskerne anså det som tilstrekkelig at pleierne jobbet på den aktuelle avdelingen. \citet{Tjora} påpeker problemet med at andre som ikke deltar i undersøkelsen kan ha andre synspunkter og erfaringer enn de som deltar. Forskerne var bevisste på dette, og brukte observasjonsdataene som støtte ved analysen.
 
\noindent
På forhånd ble intervjuenes tidsramme estimert til 30 minutter, men varigheten ble tilpasset hver informant, avhengig av hvor mye de ønsket å fortelle. Intervjuenes varighet varierte mellom 15-20 minutter for pleiere, og 30-40 for seksjonsledere.
 
\begin{table}[H]\centering
    \begin{tabular}{ |l|l|l|l|l|l| }
    \hline
    Intervju & Avdeling & Intervjuobjekt \\ \hline
       I1 & A2 & Seksjonsleder \\ \hline
       I2 & A2 & Sykepleier \\ \hline
       I3 & A2 & Sykepleier \\ \hline
       I4 & A1 & Assisterende seksjonsleder \\ \hline
       I5 & A1 & Hjelpepleier \\ \hline
       I6 & A1 & Sykepleier \\ \hline
       I7 & A3 & Seksjonsleder \\ \hline
       I8 & A3 & Sykepleier \\ \hline
       I9 & A3 & Sykepleier \\ \hline
    \end{tabular}
    \caption {Detaljer for intervjuer}
    \label{detaljerintervju}
\end{table}
