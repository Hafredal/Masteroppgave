\subsection{Utfordringer med CSCW}
\label{chp:utfordringerMedCSCW}

\noindent
Ifølge Berg (1999) vil mange CSCW-systemer komme til kort i forhold til forventningene til deres suksess. Dette kommer gjerne til syne ved at de tiltenkte brukerene unngår å bruke systemet, eller at de stadig lager midlertidige løsninger (workarounds).

\noindent
Workarounds defineres av \citet{Kobayashi05} som \emph{"informal temporary practices for handling exceptions to normal workflow"}. Direkte oversatt til norsk betyr det "å jobbe rundt", eller å finne midlertidige løsninger.
Workarounds kan være nødvendig når det oppstår akutte situasjoner hvor man ikke har nødvendige ressurser tilgjengelig, eller de kan oppstå som følge av sperrer i et system. Disse sperrene kan være tilsiktede, eller utilsiktede. \citet{Vogelsmeier08} beskriver workarounds som førstegrads problemløsing i den forstand at man lager mekanismer for å jobbe rundt problemer, uten å forsøke å løse den underliggende årsaken til at problemet oppsto. Dersom workarounds oppstår som konsekvens av at systemet er for rigid i forhold til sykepleierenes arbeidsmønster slik at systemet ikke støtter opp om arbeidet på en tilfredstillende måte, er dette svært uheldig. Dette kan i verste fall føre til livstruende situasjoner.


\noindent
Et typisk CSCW-system vil vanligvis bli brukt av et bredt spekter av brukere, med forskjellig bakgrunn, erfaringer og forhold til bruk av informasjonssystemer generelt, noe som gjør utviklingen av et slikt system svært kompleks. Desverre er det ikke uvanlig at beslutningstakere tar avgjørelser basert på hva slags funksjoner som vil være fordelaktig for brukere som dem selv, og dermed overser hva andre brukere kan ha nytte av. Funksjonalitet som lettet arbeidet for én gruppe brukere, kan gi merarbeid til en annen gruppe brukere. Det kan også være vanskelig å lære fra tidligere feil, da CSCW-systemer er svært komplekse og unike for hvert enkelt tilfelle, noe som vanskeliggjør evaluering i ettertid. Det er utfordrende å gjenskape miljø og forhold som er essensielle i den virkelige konteksten hvor et CSCW-system skal implementeres, i et laboratorium. Feltobservasjoner kan også gi et feilaktig inntrykk, da det kan være variasjoner avhengig av gruppesammensetning og miljømessige faktorer.

\noindent
Det å utvikle et CSCW-system for helseomsorgen vil dermed kunne være en utfordrende prosess. Det er en konflikt mellom det flytende samarbeidet og de tilsynelatende uforutsette arbeidsoppgavene til sykepleiere, og den formelle, standardiserte og relativt stive funksjonaliteten til et informasjonsystem. Derfor er en av forutsetningene for et suksessfullt system i et slikt miljø, å ikke forsøke å erstatte denne 'rotetheten' med en rasjonalitet og strømlinjeform som ofte er vanlig for slike systemer. Verktøy som kun har forutbestemte sekvensielle trinn, eller som kun tillater gitte typer input vil derfor ikke fungere sammen med måten sykepleierene arbeider på, og som en følge av dette ikke overleve \cite{Berg99}.