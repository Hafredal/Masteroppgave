\chapter{Analyse av funn}
\label{chp:analyseavfunn}
Forskerne vil i dette kapittelet presentere og analysere funn gjort under observasjoner og intervjuer. Informanter, observasjoner og intervjuer er referert til i henhold til tabellene \ref{referanserobservasjoner} og \ref{referanserintervju}. I teksten vil alle sykepleiere refereres til som $"$hun$"$ for å redusere risikoen for gjenkjennelse av personer. Begrepet $"$signaler$"$ omfatter både pasientsignaler og hasteanrop.

\begin{table}[H]\centering
    \begin{tabular}{ |l|l|l|l|l|l| }
    \hline
   \textbf{Observasjon} & \textbf{Avdeling} & \textbf{Tidspunkt} & \textbf{Observatør} & \textbf{Referanse}\\ \hline
       O1 & A1 & 08:00 - 10:00 & Veronica & O1-A1\\ \hline
       O2 & A2 & 14:30 - 16:30 & Monika & O2-A2 \\ \hline
      O3 & A1 & 08:00 - 10:00 & Monika & O3-A1 \\ \hline
       O4 & A2 & 14:30 - 16:30 & Veronica & O4-A2 \\ \hline
         O5 & A2 & 09:30 - 11:30 & Monika & O5-A2 \\ \hline
       O6 & A2 & 09:30 - 11:30 & Veronica & O6-A2 \\ \hline
      O7 & A1 & 15:00 - 17:00 & Monika & O7-A1 \\ \hline
       O8 & A3 & 08:00 - 10:00 & Veronica & O8-A3 \\ \hline
       O9 & A1 & 15:00 - 17:00 & Veronica & O9-A1 \\ \hline
       O10 & A3 & 15:30 - 17:30 & Monika & O10-A3 \\ \hline
      O11 & A3 & 08:00 - 10:00 & Monika & O11-A3 \\ \hline
       O12 & A3 & 15:30 - 17:30 & Veronica & O12-A3 \\ \hline
       O13 & A3 & 12:30 - 14:00 & begge & O13-A3 \\ \hline
       O14 & A3 & 17:30 - 19:00 & begge & O14-A3 \\ \hline
      O15 & A1 & 12:30 - 14:00 & begge & O15-A1 \\ \hline
       O16 & A1 & 16:30 - 18:00 & begge & O16-A1 \\ \hline
         O17 & A2 & 12:30 - 14:00 & begge & O17-A2 \\ \hline
       O18 & A2 & 16:30 - 18:00 & begge & O18-A2 \\ \hline
    \end{tabular}
    \caption {Referanser for observasjoner}
    \label{referanserobservasjoner}
\end{table}

\begin{table}[H]\centering
    \begin{tabular}{ |l|l|l|l|l|l| }
    \hline
    \textbf{Avdeling} & \textbf{Intervjuobjekt} & \textbf{Referanse} \\ \hline
       A1 & Assisterende seksjonsleder & L-A1 \\ \hline
       A1 & Hjelpepleier & P1-A1 \\ \hline
       A1 & Sykepleier & P2-A1 \\ \hline
       A2 & Seksjonsleder & L-A2 \\ \hline
       A2 & Sykepleier & P1-A2 \\ \hline
       A2 & Sykepleier & P2-A2 \\ \hline
       A3 & Seksjonsleder & L-A3 \\ \hline
       A3 & Sykepleier & P1-A3 \\ \hline
       A3 & Sykepleier & P2-A3 \\ \hline
    \end{tabular}
    \caption {Referanser for intervjuobjekter}
    \label{referanserintervju}
\end{table}

\section{Teknisk utforming}
Å levere signaler gjennom to systemer, det faste og det trådløse, er et eksempel på redundans av data (jf. \ref{sec:redundans}). Da alle sengetunene ved avdeling A1 ligger på samme sløyfe varsler veggpanelene alle signaler fra alle tun (jf. \ref{sec:system}). Det kan derfor argumenteres for at det finnes redundans av data i det faste systemet alene. Siden pleierne velger å ikke motta signalene på telefon anser de trolig dette som tilstrekkelig. For avdelingene A2 og A3 eksisterer det derimot ikke full redundans av data i det faste systemet. Dette medfører at pleierne er avhengige av å bruke telefonene for å motta signaler på tvers av disse sløyfene. Dette er spesielt viktig på kvelds- og nattskift hvor det er færre pleiere på jobb. Pleierne ved A2 og A3 opplever derfor stor nytte av å motta signaler på telefon da dette gir slik redundans (jf. \ref{sec:kognitive_elementer}).

\section{Avbrytelser og støy}
Mange oppgaver krever pleiernes fulle oppmerksomhet, og avbrytelser i form av signaler kan være uheldig dersom deres kognitive kapasitet overskrides. Dette kan hemme oppmerksomheten og føre til feil og ineffektivitet \citep{Ebright10, Parker00}.

\subsubsection{Bruk av telefon som kommunikasjonsmiddel}
IP-telefonen som benyttes som en del av pasientsignalsystemet har vært et sentralt aspekt ved forskningsarbeidet, og det ble avdekket tydelige ulikheter i bruk av denne. På alle tre avdelinger går sykepleierne med hver sin telefon og bruker denne for å ringe, og motta telefonanrop, men bare avdeling A2 og A3 bruker den for å motta signaler. Både seksjonsledere og pleiere ga under intervjuene uttrykk for at telefonen ofte brukes i situasjoner hvor de ønsker å komme i direkte kontakt med andre pleiere. L-A1 og P1-A1 påpeker også at de bruker mindre tid på å lete etter hverandre nå som alle går med telefon. Generelt uttrykker flere at telefonen er et nyttig verktøy for kommunikasjon med andre, og at det er dette den brukes mest til. Pleierne anser telefonen i slike tilfeller som et nyttig verktøy og har derfor akseptert og integrert den i sin arbeidshverdag. Videre kan man si at teknologien har påvirket det sosiale systemet da pleierne nå bruker mindre tid på å kommunisere ansikt-til-ansikt (jf. \ref{sec:tilpasning}).

\noindent
Det trekkes derimot frem som et sentralt problem på alle avdelinger at pasientsignaler varsles i telefonsamtaler da varslingen oppleves som svært forstyrrende, spesielt blant seksjonslederne og pleiere i roller hvor de ringer mye. Systemet er designet slik at signaler skal varsles uansett, og dette er dermed en tilsiktet funksjon. P2-A2 forklarte det slik: \textit{ $"$... hvis jeg snakker i telefonen, [...] og så ringer [et pasientsignal], så bryter den gjennom, og det er forferdelig å snakke i telefonen$"$}. Disse avbrytelsene medfører negative effekter som frustrasjon og forsinkelse i pleiernes arbeid (jf. \ref{sec:dualitet}). L-A2 understreker at det i hennes tilfelle vil være uaktuelt å motta signaler på sin telefon da hun har mange telefonsamtaler i løpet av en dag. P1-A3 forklarte at de løser dette problemet med å heller bruke fasttelefonen på tunet, mens L-A2 fortalte at ansvarshavende sykepleier ofte går med to telefoner, hvor den ene mottar signaler og den andre telefonsamtaler. Det oppstår dermed workarounds hvor pleierne finner løsninger på det de opplever som en sperre i systemet (jf. \ref{sec:tilpasning}). Disse tilpasningene av teknologien kan i tråd med \citet{Coiera07} sees på som implisitte signaler om en svakhet i systemet og et behov for endring. Det kan samtidig argumenteres for at denne varslingen har en positiv effekt da det sikrer at signaler oppfattes, og det ville utgjort en risiko dersom alle aktuelle pleiere snakket i telefonen uten å bli varslet (jf. \ref{sec:dualitet}). I samtale med IKT-rådgiver ved sykehuset ble forskerne gjort oppmerksomme på at systemet vil bli endret slik at signaler ikke lenger varsles under telefonsamtaler, men heller sendes direkte videre til neste mottaker. Det kan dermed antas at det vurderes som svært lite sannsynlig at alle pleiere står i telefonsamtaler samtidig. Denne endringen var ikke gjort da forskningsarbeidet ble utført, men skulle iverksettes i løpet av kort tid. Dette er en form for avbruddshåndtering som forebygger avbrytelser ved å blokkere innkommende signaler (jf. \ref{sec:håndtering}. Her er det tydelig at pleiernes lokale tilpasninger har ført til ulik bruk og tvunget frem en endring i teknologien (jf. \ref{sec:tilpasning}).

\subsubsection{Bruk av telefon for mottak av signaler}
Seksjonsleder og pleiere ved avdeling A1 viste til flere årsaker til at de ikke ønsker å benytte telefonen for mottak av signaler. Pleierne står ofte i stell og har av hygieniske årsaker ikke alltid mulighet til å interagere med telefonen. En manglende tilpasning mellom teknologi og oppgave er svært fremtredende i slike situasjoner (jf. \ref{sec:tilpasning}). L-A1 påpekte videre at de har en pasientgruppe som hyppig utløser pasientsignaler. Dette resulterte i mye støy beskrevet av L-A1 som \textit{$"$uutholdbar$"$}. Forskerne kan ikke si å ha observert at pasientgruppen på avdeling A1 utløser flere signaler enn pasientene på de to andre avdelingene. Forskerne så heller ikke noen sammenheng mellom antall utløste signal, tid og avdeling, uten at det ble foretatt kvantitative mål av dette.

\noindent
Ved de to andre avdelingene brukes telefonen i større grad slik den er tenkt, men også der trekkes støy frem som en vesentlig utfordring. P2-A2 beskrev opplevelsen av varslinger på telefon slik: \textit{ $"$...det med telefon og sånn, det synes jeg [er] veldig urolig, urolig hverdag, bråkete. Særlig hvis du står i stressende, pressende situasjoner så ringer det, og ringer, og ringer, og ringer...$"$}. 

\noindent
Hun fortalte videre at pasienter vegrer seg for be om assistanse hvis de tror sykepleierne har mye å gjøre:

\noindent
\textit{$"$... [pasientene sier] «oi, vi hører dere har det travelt». Men det trenger ikke være tilfellet. Så ringer dem ikke på. [...] Men vi trenger ikke ha det travelt selv om det ringer, for det kan jo ringe fra de andre tunene. Jeg får jo inn det på min telefon også. Det kan jo være en der borte som ligger på klokken hele dagen, og det vil jo ikke si at jeg har det travelt.$"$}

\noindent
Pleierne på avdeling A1 opplevde også at pasienter ble forstyrret av støyen. En av sykepleierne fortalte under observasjonene at noen pasienter mistet tillit til pleierne da det ringte i lommen deres uten at de besvarte anropet. Data presentert av \citet{Rygh13} viser derimot at pasientene ikke blir like forstyrret av varslingen av pasientsignaler som pleierne gjerne tror, og at pasientene heller ikke opplever at pleierne forstyrres. Både \citet{Rygh13} og egne funn tyder derimot på at pasientene likevel vegrer seg for å utløse pasientsignaler dersom de tror at pleierne har mye å gjøre, og at de dermed indirekte påvirkes av varslingene.

\noindent
Til tross for at flere pleiere uttrykte misnøye med lydnivået på telefonene poengterte de samtidig at det gir nødvendig informasjon, og en indikasjon på hvor mye det er å gjøre på avdelingen. Dualiteten ved disse avbruddene gjør det utfordrende å tilpasse systemet til sykepleiernes oppgaver, da de både gir uttrykk for at de ønsker å bli varslet samtidig som de ønsker å ha fullt fokus på pasientene (jf. \ref{sec:dualitet}). Tidligere kunne lyden på telefonen skrues av, men dette skapte en risiko for at signaler ikke ble oppdaget. Lydnivået kan derfor per i dag ikke skrues lavere enn nivå to. Under observasjon ved avdeling A2 ble forskerne fortalt at noen pleiere valgte å teipe over høyttaleren på telefonen for å dempe lyden. Et annet eksempel ble gitt under intervjuet med P1-A2 som fortalte om en pleier som hadde pakket telefonen inn i bobleplast. Slike workarounds vitner om at tilpasningen mellom teknologi og individ ikke er optimal (jf. \ref{sec:tilpasning}).

\noindent
Pleiere fra alle tre avdelinger påpekte at støy fra telefonen er et problem om natten i tilfeller hvor sykepleieren er inne hos en pasient når det utløses et signal. Dette er i tråd med \citet{klemets13} som viser at sykepleiernes håndtering av pasientsignaler avhenger av kontekst. P2-A2 ser nytten av å gå med telefonen på nattevakt da hun ønsker å være tilgjengelig for kolleger og pasienter som trenger hjelp. Samtidig uttrykte hun at det er utfordringer knyttet til dette: \textit{ $"$... så har jeg jo opplevd at pasienter våkner og ikke får sove igjen. Og det er jo en bakdel$"$}. Hun fortalte videre at pleiere på avdelingen derfor iblant legger igjen telefonen utenfor rommet for å unngå dette problemet. Dette er et tydelig eksempel på en uønsket konsekvens som gir en arbeidspraksis som i verste fall kan være en risiko for pasientsikkerhet dersom signaler ikke oppfattes (jf. \ref{sec:tilpasning}). En teknisk løsning på støyproblemet kunne vært at telefonen automatisk blir satt til lydløs når pleieren går inn på pasientrommet, altså en form for modifisert varsling (jf. \ref{sec:håndtering}). Også her oppstår imidlertid en risiko for at signaler ikke oppfattes, noe som kan ha negativ innvirkning på pasientsikkerheten. Til tross for at sykepleierne uttrykker forskjellige behov avhengig av tid på døgnet er de opptatt av at pasientsikkerheten må ivaretas. Det oppstår dermed en avveining mellom hva som er viktigst, å fullføre en oppgave uten forstyrrelser, eller å bli varslet om signaler. Pleierne ga uttrykk for at de alltid vil varsles om hasteanrop, men slik systemet fungerer i dag varsles disse på lik linje med pasientsignaler. Det er dermed grunn til å anta at det ville vært hensiktsmessig å i større grad differensiere mellom signalene, slik at kun varslingen av pasientsignaler kan modifiseres.

\noindent
Dualiteten ved avbruddene gjør tilpasningen svært vanskelig da sykepleiernes behov stadig endres avhengig av kontekst og arbeidsoppgave, som videre fører til at pleierne bruker systemet ulikt (jf. \ref{sec:tilpasning}). For å ikke være fullstendig utilgjengelig ga pleierne uttrykk for at de likevel tilstedemarkerer seg, da varslingen på rompanelene ikke oppleves som like forstyrrende. For avdelingene A2 og A3 vil det være svært alvorlig å være utilgjengelig på telefon, da de ikke mottar signaler fra alle tun på panelene. Dette gjelder også på dagtid, og er dermed et av de mest fremtredende argumentene for at disse avdelingene skal motta pasientsignaler på telefon. 

\section{Interaksjon med telefon ved utløst pasientsignal}
Både pleiere og seksjonsledere fra avdelingene A1 og A3 trekker frem telefonens grensesnitt som lite brukervennlig og beskriver den som tungvint å bruke. Det er dermed tydelig at telefonens $"$ease of use$"$ er svært mangelfull (jf. \ref{sec:kognitive_elementer}). Manglende alfabetisk telefonliste påpekes av flere som en kilde til frustrasjon. L-A1 uttrykte seg slik:

\noindent
\textit{ $"$... det jeg reagerer på når det gjelder IP-telefonen, er at man må sende en hel avdeling på kurs i flere timer for å forstå en telefon. [...] Det er ressurskrevende, og for komplisert. [...] [Den har funksjoner] som er veldig tungvinte, og det er veldig lite selvforklarende, veldig lite brukervennlig, og skiller seg vesentlig fra mobiltelefoner...$"$}.

\noindent
Et interessant funn gjort under observasjonene var at pleierne i relativt liten grad interagerer med telefonene, men heller ser på veggpanelene ved utløste pasientsignaler. Under intervjuene ble det avdekket flere årsaker til dette. For det første vil forsinkelsen i det trådløse systemet føre til at signalet varsles tidligere på panelene enn på telefonene. Til tross for at \citet{Sletten09} hevder at denne forsinkelsen har minimal betydning for sengetunene viser derimot egne funn at denne kan være kritisk, spesielt ved utløste hasteanrop da pleierne ønsker å motta disse umiddelbart. For det andre er panelene plassert slik at det ofte vil være enklere å se på disse enn å ta telefonen opp fra lommen samtidig som det i noen tilfeller vil være problematisk å håndtere telefonen, eksempelvis av hygieniske årsaker eller der det kan oppleves som forstyrrende for pasienten. Dette resulterer i at funksjonaliteten for å bekrefte eller avvise signalet ikke brukes, noe som fører til unødvendig støy siden signalet varsles i 15 sekunder før det går videre til neste mottaker. 

\noindent
Tidligere arbeid har sett på ulike former for avbruddshåndtering (jf. \ref{sec:håndtering}). \citet{Sletten09} foreslår å la pleierne sette telefonene på $"$pause$"$, som er en form for forebygging, mens \citet{Selseth12} foreslår å la pasientene sende en tekstmelding, som er en form for forhåndsvisning. I forskernes egen oppgave \citet{Sund13} foreslås det å distribuere informasjon om pleiernes tilgjengelighet, som er en form for fraråding. Da funnene viser at pleierne i så liten grad interagerer med telefonene er det derfor grunn til å anta at disse funksjonene vil være lite hensiktsmessige.

\noindent
Flere påpeker at panelene burde vært større da de kan være vanskelige å lese fra avstand. Ved avdeling A1 er vaktromsapparatet på det ene sengetunet plassert bak en dør som vanligvis står åpen. For å enklere kunne se hvor signalene er utløst fra ønsker både L-A1 og pleiere fra avdeling A1 seg et panel i taket slik de hadde før. P1-A1 kunne i tillegg tenke seg ulike varslingslyder for signaler fra de forskjellige tunene slik at pleierne enkelt hører om signalet er utløst på deres tun. Disse ønskene antyder at teknologien ikke er tilpasset den fysiske settingen hvor den er tatt i bruk (jf. \ref{sec:tilpasning}). Det er overraskende at pleierne ikke bruker telefonene som kilde til informasjon i større grad da dette kunne løse disse problemene, spesielt ved avdeling A1 hvor disse endringene etterspørres av flere. At pleierne ga uttrykk for de har behov som ikke er dekket kan bety at de ikke har forstått det nye systemets egenskaper (jf. \ref{sec:kognitive_elementer}). Det kan også bety at pleierne på tross av systemets egenskaper likevel ikke aksepterer å bruke det da de opplevde $"$ease of use$"$ som svært mangelfull. Fra utviklernes side kan panelene ha blitt laget små fordi de forventet mer bruk av telefonene, og ikke forutså at sykepleierne ikke alltid ville ha mulighet til å interagere med denne og derfor ville ha behov for større panel. I motsetning til avdeling A3 opplevde  A1 i tillegg bruk av telefonen for mottak av pasientsignaler som lite nyttig, og dette er en sentral forskjell som tydelig har ført til ulikheter i bruk.

\section{Assistansesignal}
Et interessant funn er sykepleiernes holdning til å utløse hasteanrop. Ifølge sykehusets opplæringsmateriell skal hasteanrop utløses ved behov for assistanse. Observasjonene avdekket derimot for det første at sykepleierne kaller dette signalet for en $"$stansalarm$"$, og for det andre at de helst utløser denne kun i nødsituasjoner. Pleierne forklarte at alle i slike tilfeller slipper det de har i hendene og løper for å bistå pleieren som har utløst signalet, noe som ble observert under observasjon O18-A2. Forskerne observerte også at pleierne i noen tilfeller heller kom ut fra pasientrom for å spørre om hjelp enn å utløse et hasteanrop. Det har dermed oppstått en workaround fordi fortolkningen av systemet skiller seg fra tenkt bruk (jf. \ref{sec:tilpasning}). Da de ved det gamle sykehuset hadde en egen assistanseknapp kan det også argumenteres for at bortfallet av assistanseknappen etterlot et behov som sykepleierne nå dekker gjennom workarounds. 

\noindent
En tydelig forskjell mellom avdelingene er at pleierne på avdeling A1 fortsatt har mulighet til å utløse et slikt assistansesignal. Forskerne observerte at dette er en funksjon de bruker mye og er svært fornøyd med, noe pleierne også bekreftet under intervjuene. 

\section{Holdninger til systemet}
Observasjonene og intervjuene avdekket et bredt spekter av holdninger til systemet. På avdelingene A2 og A3 brukes pasientsignalsystemet i stor grad slik det er tenkt, og seksjonslederne og pleierne opplever stort sett at det er et nyttig verktøy. Avdeling A1 har derimot opplevd større motstand mot systemet, og anvender kun deler av det.

\subsubsection{Feil og forsinkelse}
Pleierne fra avdelingene A2 og A3 opplever at pasientsignalsystemet fungerer slik det er i dag, men påpeker tekniske feil og mangler. De opplever blant annet \textit{$"$spøkelsesalarmer$"$}, signaler som ingen har utløst, eller som er utløst fra rom de ikke kjenner til. Dette fører til svært unødvendig støy og avbrytelser i et allerede avbruddsdrevet miljø, og L-A3 påpeker at dette kan føre til at pleierne blir \textit{$"$immune$"$} mot varslingene på telefonen, noe som også ble bekreftet av P2-A3:

\noindent
\textit{$"$... det er jo litt feilmeldinger og sånn på det da. [...] Det er jo ikke noe særlig. Det blir sånn ulv ulv nesten... $"$}

\noindent 
Forsinkelsen fra det faste til det trådløse systemet fører til at varslingen på telefonen fortsetter etter at en pleier har tilstedemarkert seg på et pasientrom, noe som skaper irritasjon blant pleierne. P1-A3 beskrev det slik: \textit{$"$...så fortsetter den her å pipe i inntil 1-1,5 minutt etterpå, og det er jo enormt irriterende. At når du allerede har utført jobben, så får du fortsatt beskjed om at jobben ikke er påstartet.$"$} Dette ble også observert av forskerne, og forsinkelsen førte i noen tilfeller til at pleiere kom for å besvare signaler selv etter at dette var gjort. Dette er et eksempel på redundans av innsats, og er en unødvendig avbrytelse i arbeidet til de pleierne som blir overflødige. Redundans av data gir i slike tilfeller ikke en reduksjon av innsats slik teorien tilsier, men gir en sikkerhet for at signaler besvares (jf. \ref {sec:redundans}). 

\subsubsection{Innflytting i nytt sykehus}
P1-A1 fortalte under intervjuet at de ikke var forberedt på at flyttingen skulle medføre så store endringer i systemet. Hun opplevde at de hadde et velfungerende system og antok at dette ville bli videreført. Med innføringen av IP-telefoner samt bortfall av assistanseknappen og panelet i taket, oppsto det dermed negative holdninger til det nye pasientsignalsystemet da avdelingen flyttet første gang. Da de igjen flyttet inn i nye lokaler forsøkte de på nytt å bruke systemet som tenkt. Pleierne ga derimot ikke inntrykk av at de gikk inn i denne prøveperioden med en positiv holdning, og en av pleierne fortalte under observasjon O15-A1 at dette var noe de gjorde \textit{$"$for å ha det på papiret$"$}. 

\noindent
De fikk ved andre flytting tilbake assistanseknappen de lenge hadde ønsket seg. Pleierne beskriver denne som noe de \textit{$"$kjempet for$"$} og \textit{$"$forlangte$"$} å få tilbake. P2-A1 fortalte at pleierne i prøveperioden gjorde et forsøk på å få systemet til å fungere, men sa samtidig at de visste at det var en $"$prøveperiode$"$, og at de ikke brukte det optimalt da de ofte ikke godtok signalet på telefonen, men lot det gå videre til neste mottaker. Etter prøveperioden gikk avdelingen tilbake til å ikke motta pasientsignaler på telefon. Avdelingens ikke-bruk av telefonen for mottak av signaler er et tydelig tegn på motstand (jf. \ref{sec:motstand}). Både L-A1 og flere pleiere på avdelingen uttrykte tydelig at de ikke ser nytten av slik bruk. De argumenterer med at de like gjerne kan se på panelene og at de ofte står i situasjoner hvor de ikke har mulighet til å håndtere telefonen. Dette er i tråd med \citet{Jacobsen12} som trekker frem faglig uenighet rundt nødvendigheten av endringen eller valg av løsning som en mulig årsak til motstand (jf. \ref{sec:motstand}). At pleierne tydelig ytret sine opposisjonerende meninger og nektet å bruke systemet slik det er tenkt kan beskrives som det \citet{Lapointe05} betegner som aktiv motstand. At pleierne ikke er villige til å endre sine arbeidsmåter er en form for passiv motstand. Avdelingens ikke-bruk er dermed aktiv, motivert, overveid og strukturert i tråd med \citet{Satchell09}. L-A1 forklarte sin holdning til systemet slik: 

\noindent
\textit{$"$... en ting er problemene med systemet, det kan man alltids finne løsninger på. Men når det i tillegg ikke er gevinst med noe, da er det vanskelig å få gjennom noe som i utgangspunktet er et problem, det er et system med bare ulemper og ingen fordeler. Det er veldig vanskelig å få gjennom en sånn ting. Hvis vi hadde hatt fordeler med det, så kan vi leve med ulempene. Men å få gjennom noe med bare ulemper, det er vanskelig.$"$}

\noindent
Forskerne avdekket videre at endringer i lyd ikke vil endre avdelingens innstilling til å motta signaler på telefon. Dette ble tydelig understreket av P2-A1 som sa:

\noindent
\textit{$"$... jeg forstår ikke hvorfor vi skal ha det på telefonen. [...] Hvorfor du skal ned i en lomme for å se hvorfor det ringer liksom. Da må vi jo ha tilgjengelige skjermer, tilgjengelig i tak som sagt, tilgjengelig panel.$"$}

\noindent
Samtidig er et slikt system avhengig av et tilstrekkelig antall brukere \citep{Ackerman00}. Under observasjon O7-A1 var det en av pleierne som antydet at det kanskje ikke var så negativt da alle var logget på med telefonen for mottak av signaler. Det vil derimot ikke ha noen hensikt for denne pleieren å ta i bruk systemet alene. Det kreves dermed en endring i både avdelingens strukturelle elementer og pleiernes mentale modell for at de skal se nytten av slik bruk og endre sin arbeidspraksis (jf. \ref{sec:implementering}). Selv om de ikke bruker systemet slik det er tenkt fra produsentens side opplever avdelingen at de har en velfungerende løsning slik de bruker systemet i dag. L-A1 påpekte imidlertid at det ikke er en god situasjon at de ikke bruker det slik de er pålagt. 

\noindent
Til tross for at systemet brukes i større grad slik det er tenkt ved de to andre avdelingene, er det også her variasjoner i holdninger til systemet. Pleier P1-A2 beskriver det som et $"$kjempeverktøy$"$, mens P2-A2 opplever det til tider som stressende å motta pasientsignaler på telefon. P2-A3 så derimot ikke på dette som et problem: 

\noindent
\textit{$"$Det er veldig greit at du kobler deg på rommene som du har, og at det ringer først til deg. Og at jeg kan avvise det. Det synes jeg er greit.$"$}

\noindent
Dette er i tråd med \citet{Jacobsen12} som sier at fravær av motstand ikke nødvendigvis betyr at alle er enige i løsningen. Det kan også tyde på at det eksisterer en form for apatisk motstand ved disse avdelingene (jf. \ref{sec:motstand}). Videre påpeker \citet{Berg99} at en av de største utfordringene ved utvikling av CSCW-systemer er det brede spekteret av brukere som ofte fører til individuelle holdninger. Ved å se på motstanden i avdeling A1 som noe positivt, en kritisk innvending til behovet for endring og valg av løsning, kan man avdekke problemer som ikke utelukkende eksisterer her \citep{Jacobsen12}.

\section{Ansvarsfordeling}
Under observasjon O1-A1 delte sykepleierne pasientene i to grupper, A og B. Pleierne avtalte å være to stykker sammen på de rommene de visste at stell av pasientene ville kreve mer. Også under O9-A1 delte pleierne pasientene i gruppe A og B, og de serverte mat til $"$sine$"$ pasienter. Under O16-A1 fortalte en av pleierne at de har ansvar for tre rom hver, og dersom de begge er ledige besvares pasientsignalet av den som har primæransvar, ellers deler de på å besvare signalene. Også ved avdeling A2 fordeler sykepleierne primæransvar for pasientene, men pleierne uttrykte noe delte meninger om hvorvidt de besvarer signaler fra andre pasienter. Under observasjon O6-A2 fortalte en av pleierne at \textit{$"$alle hjelper alle, vi er et team$"$}. Under observasjon O4-A2 fortalte derimot en av pleierne at de forsøker å ha en policy på at de skal hjelpe hverandre, men at noen kun svarer på signaler fra egne pasienter. Ingen av pleierne ga likevel uttrykk for at de selv unngår å svare på pasientsignaler fra pasienter de ikke har primæransvar for. At noen i større grad etterstreber å dele på det totale ansvaret kan tyde på at det eksisterer ulike verdier blant pleierne ved avdelingen (jf. \ref{sec:kognitive_elementer}). Ved avdeling A3 gir pleierne uttrykk for at de normalt fordeler primæransvar for pasientene, som innebærer oppgaver som blant annet stell, medisinering og matservering. I likhet med de andre avdelingene er det primæransvarlig som hovedsakelig besvarer pasientsignal fra sine pasienter, men pleierne understreket at dersom denne er opptatt besvares signalet av andre. Forskerne observerte ved alle tre avdelinger at pleierne på dagskift i større grad besvarte signaler fra pasienter de hadde primæransvar for, mens dette oftere varierte på kveldsskift hvor de var færre pleiere på jobb. Det ble ikke observert at en pleier forlot et pasientrom for å besvare et annet pasientsignal, noe som kan tyde på at pleierne stoler på at andre besvarer signalene. Forskerne observerte dermed ingen tydelig pleiemodell hos de tre avdelingene da alle har en arbeidspraksis som bærer preg av både primærsykepleie og teamsykepleie. Avdelingene er organisert nærmere det som kan betegnes som en modulær sykepleiemodell, da pleierne har ansvar for den totale omsorgen og distribuerer oppgaver innenfor sengetunet (jf. \ref{sec:strukturelle_elementer}). 

\noindent
Systemet er tenkt slik at pleierne skal logge seg på i bemanningsplanen som primæransvarlig for rom de har ansvar for, og som disp på andre, for å motta signaler på sin telefon. Det ble likevel avdekket avvik fra dette da noen kun logger seg på som primær, og andre kun setter seg som disp på hele tunet. På kvelds- og nattskift er det få pleiere på jobb og det er vanlig at de kun logger seg på som disp for hele tunet. For å motta signaler fra andre tun logger pleierne på A2 seg alltid på som disp på disse. På A3 gjøres dette kun på nattskift. Dersom hensikten er at pleierne skal være pålogget med primæransvar for pasientene kan det anses som en workaround at pleierne velger å gjøre noe annet som bedre passer deres arbeidspraksis (jf. \ref{sec:tilpasning}). Det kan også argumenteres for at det faste systemet er for rigid i forhold til sykepleiernes arbeidspraksis. I tråd med \citep{Ackerman00} oppstår det nye normer for bruk som bidrar til å gjøre systemet mer fleksibelt. Det er dermed en klar sammenheng mellom sløyfenes utforming og sykepleiernes bruk av systemet.

\noindent
På alle tre avdelinger er det variasjon i hvor lenge pasientene ligger inne, og dermed hvor godt pleierne kjenner dem. De fleste gir likevel uttrykk for at de normalt har tilstrekkelig kunnskap om pasientenes tilstand til å kunne besvare alle pasientsignaler og vurdere deres hastegrad. Det eksisterer dermed redundans av funksjon (jf. \ref{sec:redundans}). Til tross for at sykepleierne ga uttrykk for at de besvarer pasientsignalene raskt, observerte forskerne ved flere tilfeller at signaler ringte opp til flere minutter før de ble besvart.

\noindent
L-A2: \textit{$"$... den kulturen med å ta klokker da, den er veldig forskjellig fra sengepost til sengepost, og jeg tror at der det er mest rolig, der sykepleierne antar at det ikke haster når en pasient ringer, så er det nok en kultur for at man kan la det ringe lenge. Og hvis man lar det ringe lenge blir det veldig mye støy – for alle. Og da tror jeg motivasjonen for å bruke systemet blir ganske lav. Nettopp av den grunn.$"$}

\noindent
Til tross for at pleierne under både observasjonene og intervjuene ga uttrykk for at de har et felles ansvar for pasientene, fortalte P2-A2 at rollene primær og disp medfører en utfordring:

\noindent
\textit{$"$... det som jeg ser på som er det negative det er vel på en måte at vi setter oss opp på $"$våre pasienter$"$, så hvis jeg er opptatt med en av mine pasienter og det ringer på på de to andre mine pasienter så blir ikke klokkene tatt. De bare avslutter eller kjører telefonen videre, mange ganger, og det synes jeg er feil.$"$}.  
  
\noindent
Forskerne har forsøkt å kategorisere pleiernes holdninger til å besvare pasientsignaler i tre grupper. (1) De som kun ønsker å besvare signaler fra pasienter de har primæransvar for. (2) De som  først og fremst ønsker å besvare signaler fra egne pasienter, men også besvarer andre dersom primæransvarlig er opptatt. (3) De som besvarer alle signaler, men ikke ønsker å bli forstyrret dersom de er opptatt, for eksempel hos en annen pasient. Slik systemet fungerer i dag er det best tilpasset pleierne i gruppe (2). Pleiere i gruppe (1) ønsker ikke å bli varslet om signaler fra andres pasienter, og opplever derfor disse signalene som støy. Pleierne fra gruppe (3) opplever varslinger som støy i visse kontekster.

\noindent
Til tross for at det ikke foreligger kvantitative data på dette opplevde forskerne en tendens til at signalene varsles lenger på avdeling A1 før de blir besvart. Dette kan forklares ved at de ofte utfører oppgaver som på grunn av hygienerutiner er tidkrevende å påbegynne og avslutte. 

\noindent
Det er tydelig at avdelingenes strukturelle elementer påvirker pleiernes adopsjon og bruk av systemet. Avvik fra tenkt bruk kan forklares ved manglende samsvar mellom avdelingenes arbeidspraksis og systemets design (jf. \ref{sec:strukturelle_elementer} og \ref{sec:tilpasning}). 

\section{Opplæring}
Som påpekt av \citet{Venkatesh99} er opplæring essensielt for at brukere skal forstå og akseptere ny teknologi. Før innflytting i nye lokaler fikk pleierne opplæring i pasientsignalsystemet. Det er derimot variasjoner i hvordan pleierne opplevde denne. P2-A2 opplevde at den var mangelfull, mens for P2-A3 var hovedproblemet at opplæringen ble gitt for lang tid i forveien, slik at mye var glemt ved innflytting. Pleierne ga ikke inntrykk av å ha skriftlige retningslinjer på hvordan systemet faktisk skal brukes, men pleierne ved avdelingene A2 og A3 mener likevel de bruker det slik de skal. 

\noindent
I opplæringsmateriellet som foreligger brukes blant annet begrepene $"$signal$"$, $"$anrop$"$ og $"$alarm$"$ for å beskrive et pasientsignal, samtidig som det ble observert at sykepleierne ofte bruker begrepet $"$klokke$"$ om pasientsignalene. Da opplæringsmateriell bør ha som hensikt å fremstille teknologi som enkel å bruke er det grunn til å anta at det her virker mot sin hensikt. Mangel på tydelige retningslinjer tyder på at opplæringsmateriellet benyttes i liten grad. 

\noindent
Ulike fortolkninger og tilpasninger fører til avvik fra tenkt bruk som former nye normer for hvordan systemet brukes. Et eksempel på dette er pleiernes fortolkning av hasteanropet. Gjennom intervjuene ble forskerne gjort oppmerksomme på at avdelingene selv står for opplæring av nyansatte. Seksjonslederne og pleierne er åpne om at dette fører til at holdninger og rutiner videreføres, som beskrevet av P1-A2:

\noindent
\textit{$"$... det er sikkert individuelle forskjeller [...] hvor nøye man er, også er det sikkert litt i forhold til opplæring. Er det dårlige vaner på en avdeling, og man har opplæring så blir det kanskje til at man lærer det videre litt ubevisst. Som ny er man jo veldig var på hva slags holdninger og sånn de har de man går med, er man litt sløv og bruker det feil så er det kanskje det man lærer bort også.$"$}

\noindent
P1-A2 fortalte at det kan være en risiko dersom det er funksjoner og rutiner de nyansatte ikke kjenner til, eksempelvis at de manuelt må sette seg som disp for å motta signaler fra andre tun. Da opplæring trekkes frem som essensielt for at brukere skal oppleve et system som enkelt og nyttig å bruke, er det grunn til å anta at opplæringen, spesielt ved avdeling A1, har vært mangelfull (jf. \ref{sec:kognitive_elementer}). 

\section{Planlagte og tilfeldige oppgaver}
Felles for alle tre avdelinger er at sykepleierne utfører både planlagte og tilfeldige oppgaver. Medisinering, stell og matservering er eksempler på førstnevnte, og disse skjer normalt til faste tider. Et eksempel på sistnevnte er håndtering av pasientsignaler, som ofte utløses uten forvarsel.  

\noindent
For å motta signaler på rompanelet, og for å formidle sin tilstedeværelse til kolleger, skal sykepleierne tilstedemarkere seg på pasientrom. Under observasjonene ble det derimot avdekket store variasjoner i hvorvidt tilstedeknappen benyttes. Pleierne forklarte at de tilstedemarkerer seg oftere i tilfeller hvor de vet at oppgaven på pasientrommet tar tid, mens de ved kortere besøk heller velger å la døren stå på gløtt. Dette ble bekreftet av P2-A3 under intervjuet: 

\noindent
\textit{$"$... noen er vel litt sløve med å logge seg inn på rom da. Å trykke på grønnlyset og sånt. Kanskje litt flinkere til å gjøre det når vi vet at vi skal stelle og sånt. At vi blir der en stund. Men hvis vi bare skal inn å levere medisiner eller noe sånt så, blir det ikke brukt noe grønnlys.$"$}

\noindent
Forskerne observerte derimot at pleierne ikke alltid tilstedemarkerte seg ved lengre besøk, og heller ikke at de alltid lot døren stå oppe ved korte besøk. Når de ikke tilstedemarkerer seg mister pleierne redundansen av data som rompanelet gir. 

\noindent
På avdeling A1 har det etter innflytting i nye lokaler vært fokus på at pleierne må tilstedemarkere seg, da de ellers ikke vil bli varslet om utløste signaler. Forskerne la imidlertid ikke merke til at dette ble gjort i større grad her enn på de andre avdelingene. 

\noindent
Videre var det ulike synspunkter på hvorvidt det å logge seg på i bemanningsplanen er en rutine. Forskerne observerte ved noen tilfeller at pleiere fra tidligere skift fremdeles var logget på som ansvarlige for pasientrom. Pleierne forklarte at de iblant glemmer å logge seg på rom, spesielt hvis det er mye å gjøre, som beskrevet av P2-A2:

\noindent
\textit{$"$... vi er en ganske hektisk avdeling, [om morgenen] så kan det jo eksplodere her, og da er det kanskje fåtallet av oss... De fleste har vel logget på telefonen, men vi har jo ikke logget oss på systemet, og da [om det eksploderer og vi er i jobb alle mann] tenker vi ikke over det, før kanskje langt utpå formiddagen, at oi, her er det noe som har gått oss hus forbi, så kan vi se at det er nesten ingen som har logget seg på systemet.$"$}

\noindent
P1-A3 fortalte derimot: \textit{$"$... det er noe alle gjør. Det er inne, det er rutine nå. Det er ikke noe problem.$"$} 

\noindent
Forskerne har ikke avdekket tydelige årsaker til at pleierne ikke alltid tilstedemarkerer seg på rom eller logger seg på i bemanningsplanen, men pleierne ga uttrykk for at dette ofte skyldes en forglemmelse. Når det gjelder å tilstedemarkere seg på rom er både sykepleiernes holdninger og praksis inkonsistent. Da sykepleierne uttrykte at de kun ser fordeler ved å tilstedemarkere seg på rom kan det tyde på at manglende $"$ease of use$"$ fører til at dette ikke er fullstendig integrert i pleiernes arbeidspraksis (jf. \ref{sec:kognitive_elementer}). Et forslag til en teknisk løsning på dette kan være å automatisk tilstedemarkere pleierne når de går inn på et pasientrom. 

