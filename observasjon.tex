\subsection{Observasjon}
\label{section:observasjon} 

For finne ut hva mennesker gjør i praksis, fremfor hva de sier de gjør eller burde gjort, vil de fleste etnografiske undersøkelser involvere observasjon \citep{Oates, Blomberg93, Tjora}. Som beskrevet av \citet{Tjora} gir observasjon \textit{$"$... tilgang til sosiale situasjoner som de involverte i situasjonen ikke selv først har tolket$"$}.

\noindent
Ved hurtig etnografisk forskning må forskerne identifisere forskningsområdet og konkrete spørsmål som feltarbeidet skal svare på \citep{Millen00}. I den første observasjonsperioden ble forskningsområdet avgrenset til å omhandle sykepleiernes håndtering av innkommende pasientsignaler, og det ble spurt spørsmål vedrørende deres handlinger, vurderinger og kontekst, se tillegg X for fullstendig observasjonsplan.

\noindent
Observasjonene ble utført på tre ulike avdelinger ved St. Olavs Hospital. Disse avdelingene ble i stor grad gitt av vår veileder, da han hadde gjort observasjoner der tidligere med god erfaring. Vi anså det ikke som nødvendig å kontakte andre, da alle avdelingene hadde implementert pasientsignalsystemet, samtidig som vi vurderte de som ulike nok til at det kunne være mulig å potensielt identifisere ulik arbeidspraksis. Avdelingslederne fungerte som våre \textit{feltguider}, og i samråd med disse ble tid og sted forsøkt optimalisert slik at det var sannsynlig å observere aktiviteter relevante for vårt forskningsområde. Dersom hensikten er å studere aktiviteter på en gitt lokasjon, bør observasjoner gjøres på ulike tider av dagen \citep{Blomberg93}. Observatørene gjorde to observasjoner hver per avdeling, en på formiddag og en på ettermiddag, for å undersøke hvorvidt bruken av systemet og aktivitetsnivået varierte.

\begin{table}[H]\centering
    \begin{tabular}{ |l|l|l|l|l| }
    \hline
    Avdeling & Observasjon & Skift & Notasjon & Observatør \\ \hline
 	\multirow{4}{*}{A1} & O1 & Dag & A1-O1 & Veronica \\
 	& O2 & Kveld & A1-O2 & Veronica \\
	& O3 & Dag & A1-O3 & Monika \\
 	& O4 & Kveld & A1-O4 & Monika \\ \hline
   	\multirow{4}{*}{A2} & O5 & Dag & A2-O5 & Veronica \\
 	& O6 & Kveld & A2-O6 & Veronica \\
	& O7 & Dag & A2-O7 & Monika \\
 	& O8 & Kveld & A2-O8 & Monika \\ \hline
 	\multirow{4}{*}{A3} & O9 & Dag & A3-O9 & Veronica \\
 	& O10 & Kveld & A3-O10 & Veronica \\
	& O11 & Dag & A3-O11 & Monika \\
 	& O12 & Kveld & A3-O12 & Monika \\ 
 	\hline
    \end{tabular}
    \caption {Detaljer for første observasjonsperiode}
\end{table}

\paragraph{Observatørenes rolle}





- deltagende observasjon
- åpen observasjon

- varierte mellom deltagende observatør og fullstendig observatør =>interaktiv observatør(Tjora)




Utførelse

Utfordringer
