\subsection{Observasjon}
\label{section:observasjon}
 
Observasjonsstudier vil ofte være en naturlig del av etnografiske undersøkelser, da de som påpekt av \citet{Tjora} \textit{$"$gir tilgang til sosiale situasjoner som de involverte i situasjonen ikke selv først har tolket$"$}. Dette antyder at observasjon i større grad enn intervjuer, avdekker hva en gruppe mennesker \textit{faktisk} gjør, og ikke hva de \textit{sier} de gjør \citep{Oates, Blomberg93, Tjora}. Observasjonsdata kan samtidig være et godt supplement til intervjudata, og de to forskningsmetodene utfyller ofte hverandre. Også fra et pragmatisk perspektiv vil observasjon være hensiktsmessig, da man unngår å bruke tiden til de man forsker på ved å trekke dem ut fra sitt arbeid \citep{Tjora}. Da det var ønskelig å få et reellt innblikk i sykepleiernes arbeidspraksis, var det naturlig å velge observasjon som metode for datagenerering.
 
\noindent
\subsubsection{Avgrensing av forskningsområdet}
Før man gjør observasjonsstudier må en foreløpig problemstilling være på plass, og spesielt ved hurtig etnografisk forskning bør forskningsområdet avgrenses tilstrekkelig før man går inn i feltet. Forskerne bør derfor først identifisere forskningsområdet og konkrete spørsmål som feltarbeidet skal svare på \citep{Tjora, Millen00}. I den første observasjonsperioden ble forskningsområdet avgrenset til å omhandle sykepleiernes arbeidspraksis ved bruk av pasientsignalsystemet, og de ble spurt spørsmål vedrørende sine handlinger og vurderinger, se tillegg \ref{chp:appendix_observasjonsplan1} for fullstendig observasjonsplan. Det ble i løpet av denne perioden avdekket funn som førte til at forskningsområdet ble spisset i andre observasjonsperiode. Forskningsspørsmålene ble derfor formulert til å omhandle forskjeller i bruk av pasientsignalsystemet, både internt og mellom de ulike avdelingene, se tillegg \ref{chp:appendix_observasjonsplan2} for fullstendig observasjonsplan. Analysen av observasjonsdataene ga grunnlag for intervjuene som deretter ble utført. Intervju som metode vil bli belyst i kapittel \ref{sec:intervju}.

\subsubsection{Forskernes rolle}
Det finnes flere tilnærminger til utførelsen av observasjonsstudier, og for å kunne gjøre relevante observasjoner påpeker \citet{Tjora} viktigheten av at forskerne velger en rolle som er legitim i den settingen som observeres. Informantene var sykepleierne på vakt, og av etiske og pragmatiske hensyn ble disse informert om at de ble observert. Å gjøre skjult observasjon, hvor informantene ikke vet at de blir forsket på, var aldri et alternativ da forskerne måtte ha deres samtykke. Dette kan derimot ha ført til at sykepleierne handlet unaturlig fordi de visste at de ble forsket på, et fenomen \citet{Tjora} kaller \textit{forskningseffekt}. Forskerne har vært bevisste på at dette kan ha påvirket resultatene. Det var ikke til å unngå at utenforstående, eksempelvis pårørende, pasienter, leger eller andre sykepleiere, interagerte med både pleierne på tunet og observatørene. Der  det var naturlig ble disse kort forklart hvorfor observatørene var tilstede. Hendelser som involverte andre som ikke hadde samtykket til observasjonen, ble ikke notert.
 
\noindent
\citet{Greg} skiller mellom \textit{direkte} og \textit{deltagende observasjon}. Førstnevnte er i likhet med det \citet{Oates} betegner som \textit{systematisk observasjon}, en kvantitativ teknikk. Her observerer og noterer forskeren \textit{synlige} kvantitative, forhåndsdefinerte mål om hendelser, som hyppighet, varighet og antall. Denne type observasjon krever ingen interaksjon mellom observatør og de som observeres.
Deltagende observasjon er derimot en relativt ustrukturert, kvalitativ tilnærming, hvor analysen i større grad er tolkende. I tilfeller hvor forskerne har begrenset kunnskap om det sosiale miljøet på forhånd, kan deltagende observasjon være et godt utgangspunkt for senere, mer strukturert datainnsamling. Gjennom deltagende observasjon kan forskerne fange regler, normer og rutiner som informantene tar for gitt eller ikke har et bevisst forhold til.
Det var ikke ønskelig å gå inn i feltet med forhåndsdefinerte mål, da feltarbeidet var utforskende innenfor rammen av forskningsområdet, og kvantitative mål kunne begrense datainnsamlingen. Observatørene noterte derfor fritt ned det de observerte underveis, og analysen og resultatene ble basert på disse feltnotatene. Innledningsvis noterte observatørene en detaljert gjengivelse av det de så, men etter hvert ble observasjonene mer systematiske i den forstand at forskerne kun noterte detaljer vedrørende de hendelser de anså som relevante for forskningsområdet. I etterkant av hver observasjon transkriberte forskerne sine notater. På denne måten hadde de hendelsene friskt i minnet og kunne legge til flere detaljer der det var behov, for senere analyse. Det viste seg å være svært nyttig med flere observasjoner per avdeling, da forskerne kunne spørre informantene om hendelser fra tidligere observasjoner for å få en rikere innsikt i hva som faktisk hadde skjedd.
 
\noindent
Som understreket av \citet{Myers13}, vil det være nærmest umulig å forstå situasjoner og handlinger i en organisasjon, uten å snakke med noen om det. Da forskerne ønsket å studere både hvordan sykepleierne handlet og hvilke vurderinger de la til grunn for sine handlinger, var det naturlig å velge en \textit{interaktiv} rolle \citep{Tjora}.  Som interaktive observatører var forskerne hovedsaklig passive, og deltok ikke i de observertes aktiviteter. Samtidig tok forskerne en aktiv rolle i situasjoner hvor det var nødvendig å spørre spørsmål for å få større innsikt.
 
\subsubsection{Tid og sted}
Observasjonene ble utført på sengetun ved tre ulike avdelinger ved St. Olavs Hospital. Disse avdelingene ble valgt i samråd med forskernes veileder, da han hadde gjort observasjoner der tidligere med god erfaring. Forskerne anså det ikke som nødvendig å kontakte flere, da alle avdelingene hadde implementert pasientsignalsystemet, samtidig som de ble vurdert ulike nok, både i fysisk utforming og pleieoppgaver, til at det kunne være mulig å potensielt identifisere ulik arbeidspraksis.
 
\noindent
Som påpekt av \citet{Millen00} kan optimalisering av tidspunkt for observasjonsperioden øke sannsynligheten for at interessante hendelser oppstår. I samråd med seksjonslederne ble tid og sted forsøkt optimalisert slik at det var sannsynlig å observere aktiviteter relevante for forskningsspørsmålene. Alle seksjonslederne sa derimot på forhånd at det var vanskelig å forutsi hyppigheten av utløste pasientsignaler. \citet{Blomberg93} understreker at dersom hensikten er å studere aktiviteter på en gitt lokasjon, bør observasjoner gjøres på ulike tider av dagen. I tråd med dette gjorde forskerne i første observasjonsperiode to observasjoner hver per avdeling, en på formiddag og en på ettermiddag, for å undersøke hvorvidt bruken av systemet og aktivitetsnivået varierte. Årsaken til at forskerne valgte å observere hver for seg, kan forklares ved svakhetene \citet{Oates} trekker frem ved deltagende observasjon: 
\begin{itemize}
\item Observatøren må være tilstede når relevante hendelser oppstår, det som skjer uten forskerens tilstedeværelse blir ikke registrert.
\item Deltagende design blir iblant kritisert for manglende pålitelighet, da forskningen i stor grad avhenger av forskeren selv, og er vanskelig å gjenta av andre.
\item Det kan være vanskelig å generalisere funn, da disse kan være unike for en gitt situasjon.
\end{itemize}
\noindent
Flere observasjoner ga dermed en rikere og mer pålitelig datainnsamling, og forskerne kunne diskutere funnene som ble gjort for å avklare om disse var generaliserbare.

\noindent
Observasjonene hadde en varighet på to timer, og ble på dagskiftet utført mellom 8:00 - 11:30, og på kveldskiftet mellom 14:30 - 17:30, se tabell \ref{detaljer1} for detaljer.
Innledningsvis ønsket forskerne å være tilstede ved vaktskiftet for å få et innblikk i hvordan informasjon ble formidlet fra de som gikk av vakt til de som gikk på vakt, for å kunne vurdere om denne informasjonen senere påvirket sykepleiernes aktiviteter. Deretter ble tidspunktene justert slik at forskerne kunne observere aktiviteter som skjedde en stund etter vaktskiftet.
 
\begin{table}[H]\centering
    \begin{tabular}{ |l|l|l|l|l|l| }
    \hline
    Observasjon & Avdeling & Tidspunkt & Observatør \\ \hline
       O1 & A1 & 08:00 - 10:00 & Veronica \\ \hline
       O2 & A2 & 14:30 - 16:30 & Monika \\ \hline
      O3 & A1 & 08:00 - 10:00 & Monika \\ \hline
       O4 & A2 & 14:30 - 16:30 & Veronica \\ \hline
         O5 & A2 & 09:30 - 11:30 & Monika \\ \hline
       O6 & A2 & 09:30 - 11:30 & Veronica \\ \hline
      O7 & A1 & 15:00 - 17:00 & Monika \\ \hline
       O8 & A3 & 08:00 - 10:00 & Veronica \\ \hline
       O9 & A1 & 15:00 - 17:00 & Veronica \\ \hline
       O10 & A3 & 15:30 - 17:30 & Monika \\ \hline
      O11 & A3 & 08:00 - 10:00 & Monika \\ \hline
       O12 & A3 & 15:30 - 17:30 & Veronica \\ \hline
    \end{tabular}
    \caption {Detaljer for første observasjonsperiode}
    \label{detaljer1}
\end{table}
 
\noindent
I andre observasjonsperiode ble observasjonene utført med begge forskerne tilstede, både på dag- og kveldsskift, se tabell \ref{detaljer2} for detaljer. Dette la til rette for at forskerne kunne observere flere situasjoner, i tillegg til at de fikk en rikere forståelse av de samme hendelsene.  Som påpekt av \citet{Millen00} kan tilstedeværelsen av flere forskere i større grad forstyrre den naturlige settingen. Forskerne forsøkte etter beste evne å unngå dette ved å dele seg i situasjoner hvor de var veldig synlige, eksempelvis ved at de observerte ved hvert sitt sengetun.
 
\begin{table}[H]\centering
    \begin{tabular}{ |l|l|l|l|l|l| }
    \hline
    Observasjon & Avdeling & Tidspunkt \\ \hline
       O13 & A3 & 12:30 - 14:00 \\ \hline
       O14 & A3 & 17:30 - 19:00 \\ \hline
      O15 & A1 & 12:30 - 14:00 \\ \hline
       O16 & A1 & 16:30 - 18:00 \\ \hline
         O17 & A2 & 12:30 - 14:00 \\ \hline
       O18 & A2 & 16:30 - 18:00 \\ \hline
    \end{tabular}
    \caption {Detaljer for andre observasjonsperiode}
    \label{detaljer2}
\end{table}
 
\noindent
SETT INN BILDE AV TUNENE, skriv noe om plassering


 