\subsection{Observasjon}
\label{section:observasjon} 

For finne ut hva mennesker gjør i praksis, fremfor hva de sier de gjør eller burde gjort, vil de fleste etnografiske undersøkelser involvere observasjon som metode for datagenerering \citep{Oates, Blomberg93, Tjora}. Fra et pragmatisk perspektiv kan observasjon være hensiktsmessig, da man unngår å bruke tiden til de man forsker på ved å trekke de ut av deres arbeid \citep{Tjora}. 

\noindent
Ved hurtig etnografisk feltarbeid må forskerne identifisere forskningsområdet og konkrete spørsmål som feltarbeidet skal svare på \citep{Millen00}. I den første observasjonsperioden ble forskningsområdet avgrenset til å omhandle sykepleiernes håndtering av innkommende pasientsignaler, og de ble spurt spørsmål vedrørende deres handlinger, vurderinger og kontekst, se tillegg X for fullstendig observasjonsplan.

\noindent
Observasjonene ble utført på sengetun ved tre ulike avdelinger, ved St. Olavs Hospital. Disse avdelingene ble valgt i samråd med vår veileder, da han hadde gjort observasjoner der tidligere med god erfaring. Vi anså det ikke som nødvendig å kontakte andre, da alle avdelingene hadde implementert pasientsignalsystemet, samtidig som vi vurderte de som ulike nok, både i fysisk utforming og pleieoppgaver, til at det kunne være mulig å potensielt identifisere ulik arbeidspraksis.

\noindent
Som påpekt av \citet{Millen00} kan optimalisering av tidspunkt for observasjonsperioden øke sannsynligheten for at interessante hendelser oppstår. Avdelingslederne fungerte som våre \textit{feltguider}, og i samråd med disse ble tid og sted forsøkt optimalisert slik at det var sannsynlig å observere aktiviteter relevante for vårt forskningsområde. Alle avdelingslederne sa derimot på forhånd at det var vanskelig å forutsi hyppigheten av utløste pasientsignal fra dag til dag, og skift til skift. \citet{Blomberg93} understreker at dersom hensikten er å studere aktiviteter på en gitt lokasjon, bør observasjoner gjøres på ulike tider av dagen. I tråd med dette gjorde observatørene to observasjoner hver per avdeling, en på formiddag og en på ettermiddag, for å undersøke hvorvidt bruken av systemet og aktivitetsnivået varierte. Alle observasjonene hadde en varighet på to timer, og ble på dagskiftet utført mellom 8:00 - 11:30, og på kveldskiftet mellom 14:30 - 17:30, se tabell \ref{detaljer} for detaljer. 
Innledningsvis ønsket forskerne å være tilstede ved vaktskiftet for å få et innblikk i hvordan informasjon ble formidlet fra de som gikk av vakt til de som gikk på vakt, for å kunne vurdere om denne informasjonen senere påvirket sykepleiernes aktiviteter. Deretter ble tidspunktene justert slik at forskerne kunne observere aktiviteter som skjedde en stund etter vaktskiftet.

\begin{table}[H]\centering
    \begin{tabular}{ |l|l|l|l|l|l| }
    \hline
    Avdeling & Observasjon & Tidspunkt & Observatør \\ \hline
 	A1 & O1 & 08:00 - 10:00 & Veronica \\ \hline
 	A2 & O2 & 14:30 - 16:30 & Monika \\ \hline
	A1 & O3 & 08:00 - 10:00 & Monika \\ \hline
 	A2 & O4 & 14:30 - 16:30 & Veronica \\ \hline
   	A2 & O5 & 09:30 - 11:30 & Monika \\ \hline
 	A2 & O6 & 09:30 - 11:30 & Veronica \\ \hline
	A1 & O7 & 15:00 - 17:00 & Monika \\ \hline
 	A3 & O8 & 08:00 - 10:00 & Veronica \\ \hline
 	A1 & O9 & 15:00 - 17:00 & Veronica \\ \hline
 	A3 & O10 & 15:30 - 17:30 & Monika \\ \hline
	A3 & O11 & 08:00 - 10:00 & Monika \\ \hline
 	A3 & O12 & 15:30 - 17:30 & Veronica \\ \hline
    \end{tabular}
    \caption {Detaljer for første observasjonsperiode}
    \label{detaljer}
\end{table}

\noindent
Det finnes flere tilnærminger til utførelsen av observasjon. \citet{Oates} skiller mellom \textit{åpen} og \textit{skjult observasjon}, hvor åpen observasjon innebærer at informantene vet at de blir observert, mens skjult observasjon innebærer det motsatte. Videre skiller \citet{Greg} mellom \textit{direkte observasjon} og \textit{deltagende observasjon}, hvor direkte observasjon, i likhet med det \citet{Oates} betegner som \textit{systematisk observasjon}, er en kvantitativ teknikk. Her observerer og noterer forskeren kvantitative, forhåndsdefinerte mål om hendelser, som hyppighet, varighet og antall. Deltagende observasjon er derimot en relativt ustrukturert, kvalitativ tilnærming, hvor analysen i større grad er tolkende. For å kunne gjøre relevante observasjoner, påpeker \citet{Tjora} viktigheten av å velge en observasjonsrolle som er legitim i den settingen som  observeres. \citet{Gold58} definerer fire ulike observasjonsroller : \textit{fullstendig deltager}, \textit{observerende deltager}, \textit{deltagende observatør} og \textit{fullstendig observatør}. Ifølge \cite{Gold58} er begge de fullstendige rollene forbeholdt skjult observasjon. \citet{Oates} åpner derimot for at rollen som fullstendig observatør også er mulig innen åpen observasjon. Ved utførelsen av observasjonene ble det naturlig å velge rollen som \textit{interaktiv observatør} \citep{Tjora}. Dette kan hevdes å være en krysning mellom rollene deltagende observatør og fullstendig observatør. 


 
ETISKE HENSYN

\citet{Tjora} understreker at man må være åpen i møte med feltet, og fleksibel og åpen for å justere observasjonene etter hva som er minst forstyrrende. 



- varierte mellom deltagende observatør og fullstendig observatør =>interaktiv observatør(Tjora)

