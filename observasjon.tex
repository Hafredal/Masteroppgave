\subsection{Observasjon}
\label{section:observasjon} 

For finne ut hva mennesker gjør i praksis, fremfor hva de sier de gjør eller burde gjort, vil de fleste etnografiske undersøkelser involvere observasjon \citep{Oates, Blomberg93, Tjora}. Som beskrevet av \citet{Tjora} gir observasjon \textit{... tilgang til sosiale situasjoner som de involverte i situasjonen ikke selv først har tolket}.

\noindent
I tråd med feltmetodene innen hurtig etnografi ble forskningsområdet avgrenset til å omhandle sykepleiernes håndtering av innkommende pasientsignaler.


Observasjonene ble utført på tre ulike avdelinger ved St. Olavs Hospital. Disse avdelingene ble i stor grad valgt av vår veileder, da han hadde gjort observasjoner der tidligere med god erfaring. Vi anså det ikke som nødvendig å kontakte andre, da de alle hadde implementert pasientsignalsystemet, og vi vurderte de som ulike nok til at det kunne være mulig å potensielt identifisere ulik arbeidspraksis.

I samråd med veileder og professor ble ikke forskningsområdet avgrenset mer enn at fokuset ble rettet mot sykepleiernes bruk av systemet, spesielt ved håndtering av innkommende pasientsignaler.


Forskerne gjorde to observasjoner hver per avdeling, en på formiddag og en på ettermiddag,for å undersøke hvorvidt bruken av systemet og aktivitetsnivået varierte.



\subsection{Rolle}
- deltagende observasjon
- åpen observasjon

- varierte mellom deltagende observatør og fullstendig observatør =>interaktiv observatør(Tjora)


FOKUS


\subsection{Utførelse}

\subsection{Utfordringer}
