\section{Motstand, Ikke-Bruk og Medvirkning}
\label{sec:motstand}

Tidligere studier har vist at motstand er et mangfoldig og vedvarende fenomen, og kan enkelt forklares som alt ansatte gjør som ledelsen ikke vil de skal gjøre, og alt de unnlater å gjøre som ledelsen ønsker de skal gjøre \citep{Timmons03}. 

\noindent
Når vi ser på motstand mot en endring er det viktig å identifisere og forstå hva som er årsaken til motstanden \citep{Lapointe05}. \citet{Timmons03} påpeker årsakene til motstand oppstår i grensesnittet mellom systemet og eksisterende arbeidsmetoder, noe som stemmer godt overens med en sosioteknisk tankegang. Typisk vil ansatte motsette seg endringer de tror vil føre til for eksempel tap av status eller inntekt. Individer vil ofte motsette seg implementering av systemer de tror vil gi forskjellsbehandlig og urettferdighet, mens grupper vil motsette seg endringer de tror vil føre til tap av makt \citep{Lapointe05}. \citet{Jacobsen12} trekker frem blandt annet faglig uenighet rundt nødvendigheten av endringen eller valg av løsning, frykt for det ukjente og usikkerheten endringen medfører samt ekstraarbeid som mulige årsaker til motstand.

\noindent
Motstand synliggjøres hovedsaklig gjennom motstandernes adferd, og \citep{Lapointe05} klassifiserer motstand i fire nivåer basert på dette.

\begin{itemize}
\item Apati inkluderer passitivitet, mangel på interesse og distanse fra endringen.
\item Passiv motstand inkluderer forsinkelser, unskyldninger og å holde på tidligere gjøremåter
\item Aktiv motstand tar blandt annet i bruk ytringer av opposisjonerende meninger, dannelse av koalisjoner og delevis eller total nekt til bruk av systemet
\item Aggressiv motstand kan innebære intern strid, trusler, streik, boikott og sabotasje, og søker å være forstyrrende eller destruktiv
\end{itemize}

\noindent
Det vanlige fokuset i forskning på menneske-maskin interaksjon er brukere av systemet, og i tilfeller hvor ikke-brukere har vært av interesse er disse gjerne identifiert som potensielle brukere. 
\citet{Satchell09} understreker at selv om noen ikke-brukere er postensielle brukere, gjelder ikke dette alle. Blandt annet trekker de frem misnøye med systemet, uttrykt gjennom kun delevis bruk av dette, og aktiv motstand som former for ikke-bruk.

\noindent
\citet{Lapointe05} konkluderer i sin studie med at svak håndtering av motstand tidlig i endringsprosessen i lengden vil provosere frem en eskalering av denne. De hevder videre at motstanden tidlig i en implementeringsfase vil være rettet mot systemet i seg selv og dets funksjoner, og at det er viktig å gjøre nødvendige tilpassninger i denne perioden. Dersom dette ikke gjøres vil håndteringen av motstand bli en krevende oppgave. \citet{Jacobsen12} hevder at indre motivasjon og involvering av de ansatte slik at de føler seg som medeiere i endringsprosessen, er avgjørende for å motivere de ansatte for endringen, og dermed redusere motstand. Bred deltagelse gir den enkelte ansatte opplevelsen av at den er med på å forme sin egen fremtid, noe som vil skape aksept og forståelse for usikkerheten som er assosiert med endringen.

\noindent
Motstand oppfattes ofte som utelukkende negativt, og de som motsetter seg en endring sees ofte på som gammeldagse \citep{Jacobsen12}. Brukerene blir av mange delt inn i de gode og de dårlige brukerne, henholdsvis de som adopterer og bruker systemet slik det var tenkt, og de som ikke omfavner systemet \citep{Satchell09}. Dette skjuler det faktum at motstand i mange tilfeller bør sees som noe positivt, i form av kritiske invendinger til behovet for endring og valg av løsning. Fravær av motstand \emph{kan} bety at alle er enige og går helhjertet inn for den nye løsningen, men det kan også bety at de ansatte er uinteressert i hvordan det går med organisasjonen \citep{Jacobsen12}. på samme måte er ikke-bruk ikke fravær av noe eller et tomrom, men ofte heller aktivt, meningsfullt, motivert, overveid, stukturert og prodktivt \citep{Satchell09}

\subsection{Medvirkning}
\label{sec:medvirkning}
Jacobsen (2012) hevder at indre motivasjon og involvering av de ansatte slik at de føler seg som medeiere i endringsprosessen, er avgjørende for å få de ansatte motivert for endringen, og dermed redusere motstand.
Empiriske studier kan imidlertid ikke bevise at det alltid er en sammenheng mellom involvering av, og medvirkning fra de ansatte og et informasjonsystems suksess. Dette tyder på at medvirkning hverken er tilstrekkelig eller ytterst nødvendig for å redusere motstanden nok til å garantere en suksessfull implementering og bruk av det nye systemet \citep{Cavaye95}. 