\section{Motstand}
\label{sec:motstand}

Tidligere studier har vist at motstand er et mangfoldig og vedvarende fenomen. At motstand mot IT-systemer er vanlig kan vi se av hyppigheten av identifisering av slike situasjoner innen helseinformatikk.  Motstand kan enkelt forklares som alt ansatte gjør som ledelsen ikke vil de skal gjøre, og alt de unnlater å gjøre som ledelsen ønsker de skal gjøre \citep{Timmons03}.

\noindent
Når vi ser på motstand mot en endring er det viktig å identifisere og forstå hva som er årsaken til motstanden \citep{Lapointe05}. Det er gjerne ikke endringene i seg selv, men truslene de berørte tror endringen vil medføre som er åsak til motstand. Typisk vil ansatte motsette seg endringer de tror vil føre til for eksempel tap av status, tap av inntekt eller tap av makt. Individer vil ofte motsette seg implementering av systemer de tror vil gi forskjellsbehandlig og urettferdighet, mens grupper vil motsette seg set dersom de tror det kan føre til tap av makt \citep{Lapointe05}. \citet{Jacobsen12} trekker frem blandt annet faglig uenighet rundt nødvendigheten av endringen, frykt for det ukjente og usikkerheten endringen medfører og  ekstraarbeid som mulige årsaker til motstand. I \citep{Timmons03} leser vi at motstand må forstås i form av tilpassningen mellom systemet og eksisterende arbeidsmetoder, noe som stemmer godt overens med en sosioteknisk tankegang.

\noindent
Vi kan klassifisere motstand i fire nivåer som synliggjøres gjennom motstanderenes adferd, hoveddimensjonen ved motstnand. (1) apati, som inkluderer passitivitet, mangel på interesse og distanse fra endringen, (2) passiv motstand, inkludert forsinkelser, unskyldninger, og holde på tidligere gjøremåter (3) aktiv motstand, tar blandt annet i bruk ytringer av opposisjonerende meninger og dannelse av koalisjoner, og (4) aggressiv motstand, som kan innebære intern strid, trusler, streik, boikott og sabotasje. Slik aggressiv motstand søker å være forstyrrende eller destruktiv \citep{Lapointe05}. De peker også på at i tidlige stadier en endringsprosess er den individuelles adferd i forhold til endringen i stor grad uavhengig av andres, mens de på senere stadier ofte samles.

\noindent
\citet{Lapointe05} konkluderer i sin studie med at svak håndtering av motstand tidlig i endringsprosessen i lengden vil provosere frem en eskalering av denne. De hevder videre at motstanden tidlig i en implementeringsfase vil være rettet mot systemet i seg selv og dets funksjoner, og at det er viktig å gjøre nødvendige tilpassninger i denne perioden. Dersom dette ikke gjøres vil håndteringen av motstand bli en krevende oppgave.

\noindent
Mostand oppfattes ofte som utelukkende negativt, og de som motsetter seg en endring sees ofte på som gammeldagse. Dette skjuler det faktum at motstand i mange tilfeller bør sees som noe positivt, i form av kritiske invendinger til behovet for endring og valg av løsning. Mangel på motstand \emph{kan} bety at alle er enige og går helhjertet inn for den nye løsningen, men det kan også bety at ingen bryr seg og at de ansatte er uinteressert i hvordan det går med organisasjonen \citep{Jacobsen12}.