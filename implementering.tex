\section{Implementering og adopsjon av nye IKT-systemer}
\label{sec:implementering}
Å investere i ny informasjonsteknologi vil kun lede til økt produktivitet dersom teknologien aksepteres og brukes \citep{Venkatesh99}. Ifølge \citet{Orlikowski92} har kognitive og strukturelle elementer betydelige implikasjoner for adopsjon, forståelse og tidlig bruk av et nytt CSCW-system. 

\subsection{Kognitive elementer}
\label{sec:kognitive_elementer}
Kognitive elementer betegner individers mentale modeller, som innebærer hvordan de tenker om verden, blant annet sin organisasjon, sitt arbeid og teknologi. Mens \citet{Berg99} og \citet{Ackermann00} understreker at brukere ofte har ulik kunnskap og bakgrunn, påpeker \citet{Orlikowski92} at brukere med felles faglig bakgrunn, arbeidserfaring og regelmessig interaksjon ofte fører til at en gruppe deler felles antagelser og verdier. Dersom ny teknologi skiller seg fra den som er brukt tidligere kreves en endring i brukernes mentale modell for at de skal forstå hvordan de skal interagere med den nye teknologien på en effektiv måte. Hvordan brukere endrer sine mentale modeller for å tilpasse seg ny teknologi avhenger av typen og mengden informasjon de mottar, og hvilken opplæring de får. Hvis brukergruppen har en dårlig eller feilaktig forståelse av de unike og nye egenskapene til en ny teknologi vil de kunne motsette seg å bruke den, eller velge å ikke integrere den i sitt arbeid. Informasjon og opplæring er derfor sentralt for å få brukerne til å forstå teknologiens muligheter og nytte \citep{Orlikowski92}. I tråd med TAM\footnote{Technology Acceptance Model} vil et individs ønske om å bruke et system avhenge av to faktorer: (1) opplevd nytte, definert som i hvilken grad en person mener at bruk av et system vil øke sin jobbytelse, og (2) oppfattet $"$ease of use$"$, definert som i hvilken grad et system oppleves som enkelt å bruke \citep{Venkatesh00}. Opplæring er derfor essensielt for at brukere skal oppfatte ny teknologi som enkel å bruke, som igjen fører til at teknologien aksepteres og brukes videre \citep{Venkatesh99}.

\subsection{Strukturelle elementer}
\label{sec:strukturelle_elementer}
Strukturelle elementer betegner belønningssystemer, retningslinjer, arbeidspraksis og normer som formes av menneskene i organisasjonen. I en organisasjon kan individer og grupper ha ulike mål, ulik kunnskap og bakgrunn \citep{Ackermann00}. Dersom teknologien ikke er tilpasset organisasjonens strukturelle elementer, vil den sannsynligvis ikke skape effektiv samhandling uten at disse endres. \citet{Orlikowski92} viser et eksempel fra et stort tjenestefirma $"$Alpha Solutions$"$ (pseudonym), som har implementert applikasjonen $"$Notes$"$. Notes støtter kommunikasjon, koordinasjon og samarbeid innen organisasjonen, og inkluderer blant annet e-post og mulighet for diskusjonsforum og lignende. En av utfordringene ved implementeringen av Notes var at firmaet hadde en konkurransepreget kultur, hvor ansatte i stor grad jobbet selvstendig og ikke ønsket å dele informasjon. Denne individualismen støttet ikke samarbeid og deling av kunnskap, og sto dermed i kontrast til de underliggende forutsetningene for CSCW-systemet. 


\noindent
For sykepleiere kan et strukturelt element være hvordan de organiserer ansvar og oppgaver. Dette kan gjøres på ulike måter og \citet{Rygh13} presenterer primærsykepleie og teamsykepleie som modeller for dette. Førstnevnte deler oppgavene inn i kategorier som medisinering og sårstell. Hver sykepleier får deretter sine oppgaver som de har ansvar for den aktuelle vakten. På denne måten vil sykepleierne som team tilsammen utføre alle oppgaver. Teamledere koordinerer omsorgen utført av teamet og har, i samarbeid med avdelingsleder, ansvar for pasientbehandling og kommunikasjon med annet helsepersonell. I primærsykepleie tildeles ansvar for pasientene til individuelle sykepleiere. Primærsykepleier har ansvar for koordinering av omsorg og pleie for et lite antall paseinter fra innleggelse til utskrivelse, uten å måtte gjøre alle oppgaver knyttet til dette. 

\noindent
Forskning viser til at innføring av primærsykepleie i stor grad kan gjøre pleierne mer autonome i sitt arbeid, og også mer pasientorientert. Nyere forskning viser imidlertid at avdelinger ofte er organisert på måter som tar i bruk egenskaper fra de forskjellige modellene. Et eksempel på dette er modulær sykepleie som er organisert rundt relativt små geografiske grupperinger av pasienter, og hvor pleiepersonellet har ansvar for den totale omsorgen, og distribuerer oppgaver innenfor teamet \citep{Rygh13}.

\subsection{Tilpasning og interaksjon mellom elementene i et sosioteknisk system}
\label{sec:tilpasning}
\citet{Harrison} påpeker at implementering av ny IT\footnote{Informasjonsteknologi} i helsesektoren ofte medfører uforutsette og uønskede konsekvenser, noe som kan undergrave gitte praksiser og i verste fall være en risiko for pasientsikkerheten. Ofte vil helsepersonell skylde på teknologien for at slike konsekvenser og implementeringsfeil oppstår, til tross for at det i flere tilfeller er det sosiotekniske samspillet som har feilet. En vanlig feiltolkning er at problemer som ligger i samspillet mellom bruker og oppgave blir tillagt teknologien. Et eksempel på dette er dersom innføringen av et nytt IT-system gir brukerne flere dokumentasjonssoppgaver. Ofte vil det nye systemet få skylden for dette, mens problemet ikke nødvendigvis ligger i teknologien i seg selv, men i brukernes misnøye med oppgaven \citep{FITT}.

\noindent
\citet{Harrison} deler det sosiotekniske systemet i ny teknologi, arbeidsmønstre, helsepersonell og organisasjon. Samspillet mellom disse elementene innebærer komplekse interaksjoner, og kan føre til svært ulik bruk selv av identiske systemer. Eksempler på slike interaksjoner er:

\begin{itemize}
\item Ny IT endrer det eksisterende sosiale systemet, deriblant arbeidsmønstre, kommunikasjon og relasjon mellom helsepersonell. Teknologien kan eksempelvis føre til at helsepersonell må bruke mer tid på dokumentasjon enn tidligere, eller en reduksjon i kommunikasjonen ansikt-til-ansikt.
\item Manglende tilpasning av IT til den fysiske settingen hvor den skal tas i bruk kan medføre bruk og workarounds som har negative effekter på sikkerhet, kvalitet og effektivitet. Teknologien kan eksempelvis være plassert slik at den er problematisk å aksessere eller flytte.
\item Fortolkninger og tilpasninger av ny IT vil ofte føre til praksiser som skiller seg fra tenkt bruk. Disse avvikene oppstår fordi det opprinnelige designet ikke reflekterer de eksisterende arbeidsmønstre og sosiale relasjoner, og workarounds oppstår som resultat av dette. 
\item Som et resultat av at brukere gjør lokale tilpasninger som skiller seg fra tenkt bruk kan utviklere og ledere være tvunget til å gjøre endringer i teknologien.
\end{itemize} 

\noindent
For å lettere kunne analysere og identifisere de sosiotekniske faktorene som påvirker innføringen av nye IT-applikasjoner og -systemer, deler \citet{FITT} settingen hvor sysstemet skal brukes inn i de tre objektene individ, oppgave og teknologi. Hensikten er å se på samspillet mellom objektene (og deres egenskaper) og hvordan de passer sammen, ikke objektene i seg selv, for dermed å kunne peke på hvilke faktorer det er som hindrer optimal utnyttelse av den teknologien. De tre objektene innehar ulike egenskaper som kan være tilstede i større eller mindre grad. For et individ kan disse blant annet være IT-kunnskap, motivasjon, åpenhet for endringer i arbeidsmåte og team-kultur. For en oppgave kan det være organisering av oppgavene som skal gjennomføres og oppgavenes grad av kompleksitet. Teknologi kan ha egenskaper som stabilitet og brukbarhet, funksjonalitet og tilgjengelig teknisk infrastruktur. Dersom objektene mangler, eller har lite utviklede egenskaper som er nødvendige for et optimalt samspill, kan man direkte påvirke disse. Et eksempel er manglende IT-kunnskap blant brukerne (individ), hvor opplæring av ansatte kan øke tilstedeværelsen av denne egenskapen. Siden dette er tiltak som påvirker objektene, har det en indirekte påvirkning på samspillet mellom dem. 

\subsection{Workarounds}
\label{sec:workarounds}
Menneskelig aktivitet er fleksibel, nyansert og kontekstualisert, og det er vanskelig å designe løsninger som støtter dette da tekniske systemer ofte er rigide og lite fleksible. Brukere skaper derfor kontinuerlig normer for bruk av et system som bidrar til å gjøre systemet mer fleksibelt. Dermed kan systemer brukes på måter som utviklerne ikke har forutsett \citep{Ackermann00}. Ved å få teknologi til å oppføre seg på måter, og i settinger den ikke i utgangspunktet var ment for oppstår det som kalles $"$workarounds$"$. Slike tilpasninger kan sees på som implisitte signaler om behov som ikke er dekket, og at å studere disse er en effektiv måte å identifisere styrker og svakheter ved eksisterende systemer og prosesser, og dermed forstå hva som må endres \citep{Coiera07}. 

\noindent
Workarounds defineres av \citet{Kobayashi05} som \emph{"informal temporary practices for handling exceptions to normal workflow$"$}. Oversatt til norsk betyr det $"$uformelle, midlertidige løsninger for å håndtere avvik fra normal arbeidsflyt$"$. Slike midlertidige løsninger kan være nødvendig når det oppstår akutte situasjoner hvor man ikke har nødvendige ressurser tilgjengelig, eller de kan oppstå som følge av sperrer i et system. Disse sperrene kan være tilsiktede, eller utilsiktede. \citet{Vogelsmeier08} beskriver workarounds som førstegrads problemløsing i den forstand at man lager mekanismer for å jobbe rundt problemer, uten å forsøke å løse den underliggende årsaken til at problemet oppsto. Dersom workarounds oppstår som konsekvens av at systemet er for rigid i forhold til sykepleierenes arbeidsmønster slik at systemet ikke støtter opp om arbeidet på en tilfredstillende måte, er dette svært uheldig. Dette kan i verste fall føre til livstruende situasjoner.

\noindent
Det å utvikle et CSCW-system for helseomsorgen kan dermed være en utfordrende prosess. Det er konflikt mellom det flytende samarbeidet og de tilsynelatende uforutsette arbeidsoppgavene til sykepleiere, og den formelle, standardiserte og relativt stive funksjonaliteten til et informasjonsystem. En av forutsetningene for et suksessfullt system i et slikt miljø er derfor å ikke forsøke å erstatte denne $"$rotetheten$"$ med en rasjonalitet og strømlinjeform som ofte er vanlig for slike systemer. Verktøy som kun har forutbestemte sekvensielle trinn, eller som kun tillater gitte typer input vil derfor ikke fungere sammen med sykepleiernes arbeidsmåter, og som en følge av dette ikke overleve \citep{Berg99}.
Ifølge \citet{Berg99} vil en sosioteknisk tilnærming være kritisk til systemer som forsøker å ta avsand fra den nødvendig rotete og $"$ad hoc$"$ måten helsearbeidere jobber på, gjennom IT-systemer med stor grad av standardisering og rasjonalisering av oppgave og arbeidsflyt.

\noindent
Det å utvikle gode CSCW-systemer kan være svært utfordrende. I tillegg til å måtte passe en organisasjons kognitive og strukturelle elementer, vil mangfoldet av brukere innebære individuelle erfaringer og holdninger til bruk av et informasjonssystem. Videre kan det være vanskelig å lære fra tidligere feil da slike systemer er svært komplekse og unike for hvert enkelt tilfelle, noe som vanskeliggjør evaluering i ettertid. Det er i tillegg utfordrende å gjenskape miljø og forhold som er essensielle i den virkelige konteksten hvor systemet skal implementeres i et laboratorium \citep{Berg99}.

\subsection{Motstand}
\label{sec:motstand}
Tidligere studier har vist at motstand er et mangfoldig og vedvarende fenomen, og kan enkelt forklares som alt ansatte gjør som ledelsen ikke vil de skal gjøre, og alt de unnlater å gjøre som ledelsen ønsker de skal gjøre \citep{Timmons03}.
 
\noindent
Når vi ser på motstand mot endring er det viktig å identifisere og forstå hva som er årsaken til motstanden \citep{Lapointe05}. \citet{Timmons03} påpeker at årsakene til motstand oppstår i grensesnittet mellom systemet og eksisterende arbeidsmetoder, som også er i tråd med en sosioteknisk tankegang. \citet{Jacobsen12} trekker blant annet frem faglig uenighet rundt nødvendigheten av endringen eller valg av løsning, frykt for det ukjente og usikkerheten endringen medfører og ekstraarbeid som mulige årsaker til motstand.
 
\noindent
Motstand synliggjøres hovedsakelig gjennom motstandernes adferd, og \citet{Lapointe05} klassifiserer motstand i fire nivåer basert på dette.
 
\begin{itemize}
\item Apati inkluderer passivitet, mangel på interesse og distanse fra endringen.
\item Passiv motstand inkluderer forsinkelser, unnskyldninger og lite villighet for endring i arbeidsmåter.
\item Aktiv motstand tar blant annet i bruk ytringer av opposisjonerende meninger, dannelse av koalisjoner og delvis eller total nekt av bruk av systemet.
\item Aggressiv motstand kan innebære intern strid, trusler, streik, boikott og sabotasje, og søker å være forstyrrende eller destruktiv.
\end{itemize}
 
\noindent
Det vanlige fokuset i forskning på menneske-maskin interaksjon er brukere av systemet, og i tilfeller hvor ikke-brukere har vært av interesse, er disse gjerne identifisert som potensielle brukere. \citet{Satchell09} understreker at dette ikke gjelder alle, og de trekker frem misnøye med systemet, uttrykt gjennom kun delvis bruk av dette, og aktiv motstand som former for ikke-bruk.
 
\noindent
Motstand oppfattes ofte som utelukkende negativt, og de som motsetter seg en endring sees ofte på som gammeldags \citep{Jacobsen12}. Brukerne blir av mange delt inn i gode og  dårlige brukere, henholdsvis de som adopterer og bruker systemet slik det var tenkt, og de som ikke omfavner systemet \citep{Satchell09}. Dette skjuler det faktum at motstand i mange tilfeller bør sees som noe positivt, i form av kritiske innvendinger til behovet for endring og valg av løsning. Fravær av motstand \textit{kan} bety at alle er enige og går helhjertet inn for den nye løsningen, men det kan også bety at de ansatte er uinteressert i hvordan det går med organisasjonen \citep{Jacobsen12}. På samme måte er ikke ikke-bruk fravær av noe eller et tomrom, men ofte heller aktivt, meningsfullt, motivert, overveid, strukturert og produktivt \citep{Satchell09}
 
