\section{Implementering og adopsjon av nye IKT-systemer}
\label{sec:implementering}
Å investere i ny informasjonsteknologi vil kun lede til økt produktivitet dersom teknologien aksepteres og brukes \citep{Venkatesh99}. Ifølge \citet{Orlikowski92} har kognitive og strukturelle elementer betydelige implikasjoner for adopsjon, forståelse og tidlig bruk av et nytt CSCW-system. Strukturelle elementer betegner belønningssystemer, retningslinjer, arbeidspraksis og normer som formes av menneskene i organisasjonen. I en organisasjon kan individer og grupper ha ulike mål, ulik kunnskap og bakgrunn \citep{Ackermann00}. Dersom teknologien ikke er tilpasset organisasjonens strukturelle elementer, vil den sannsynligvis ikke skape effektiv samhandling uten at disse endres. Forfatteren viser et eksempel fra et stort tjenestefirma $"$Alpha Solutions$"$ (pseudonym), som har implementert applikasjonen $"$Notes$"$. Notes støtter kommunikasjon, koordinasjon og samarbeid innen organisasjonen, og inkluderer blant annet e-post og mulighet for diskusjonsforum og lignende. En av utfordringene ved implementeringen av Notes var at firmaet hadde en konkurransepreget kultur, hvor ansatte i stor grad jobbet selvstendig og ikke ønsket å dele informasjon. Denne individualismen støttet ikke samarbeid og deling av kunnskap, og sto dermed i kontrast til de underliggende forutsetningene for CSCW-systemet.

\noindent
Sykepleiere kan organisere ansvar og oppgaver på ulike måter, og \citet{Rygh13} presenterer primærsykepleie og teamsykepleie som modeller for dette. Førstnevnte deler oppgavene inn i kategorier som medisinering og sårstell. Hver sykepleier får deretter sine oppgaver som de har ansvar for den aktuelle vakten. På denne måten vil sykepleierne som team tilsammen utføre alle oppgaver. Teamledere koordinerer omsorgen utført av teamet og har, i samarbeid med avdelingsleder, ansvar for pasientbehandling og kommunikasjon med annet helsepersonell. I primærsykepleie tildeles ansvar for pasientene til individuelle sykepleiere. Primærsykepleier har ansvar for koordinering av omsorg og pleie for et lite antall paseinter fra innleggelse til utskrivelse, uten å måtte gjøre alle oppgaver knyttet til dette. 

\noindent
Forskning viser til at innføring av primærsykepleie i stor grad kan gjøre pleierne mer autonome i sitt arbeid, og også mer pasientorientert. Nyere forskning viser imidlertid at avdelinger ofte er organisert på måter som tar i bruk egenskaper fra de forskjellige modellene. Et eksempel på dette er modulær sykepleie som er organisert rundt relativt små geografiske grupperinger av pasienter, og hvor pleiepersonellet har ansvar for den totale omsorgen, og distribuerer oppgaver innenfor teamet \citep{Rygh13}.

\noindent
Kognitive elementer betegner brukernes \textit{mentale modeller}, det vil si hvordan de tenker om verden, blant annet sin organisasjon, sitt arbeid og teknologi. Mens \citet{Berg99} og \citet{Ackermann00} understreker at brukere ofte har ulik kunnskap og bakgrunn, påpeker \citep{Orlikowski92} at brukere med felles faglig bakgrunn, arbeidserfaring og regelmessig interaksjon ofte fører til at en gruppe deler felles antagelser og verdier . Hvis brukergruppen har en dårlig eller feilaktig forståelse av de unike og nye egenskapene til en ny teknologi, vil de kunne motsette seg å bruke den, eller velge å ikke integrere den i sitt arbeid. Videre påpeker \citet{Ackermann00} at brukere ikke bare tilpasser seg nye systemer, men at de også tilpasser systemet til sitt bruk. Dermed kan systemer brukes på måter som utviklerne  ikke har forutsett. Informasjon og opplæring er sentralt for å få brukerne til å forstå teknologiens muligheter og nytte. Dette støttes av \citet{Venkatesh99} som trekker frem opplæring som essensielt for at brukere skal oppfatte ny teknologi som enkel å bruke, som igjen fører til at teknologien aksepteres og brukes videre. Videre vektlegges viktigheten av å ha en motivator som under opplæringen bidrar til å øke brukernes motivasjon.
 
\subsection{Motstand}
\label{sec:motstand}
Tidligere studier har vist at motstand er et mangfoldig og vedvarende fenomen, og kan enkelt forklares som alt ansatte gjør som ledelsen ikke vil de skal gjøre, og alt de unnlater å gjøre som ledelsen ønsker de skal gjøre \citep{Timmons03}.
 
\noindent
Når vi ser på motstand mot endring er det viktig å identifisere og forstå hva som er årsaken til motstanden \citep{Lapointe05}. \citet{Timmons03} påpeker at årsakene til motstand oppstår i grensesnittet mellom systemet og eksisterende arbeidsmetoder, som også er i tråd med en sosioteknisk tankegang. \citet{Jacobsen12} trekker blant annet frem faglig uenighet rundt nødvendigheten av endringen eller valg av løsning, frykt for det ukjente og usikkerheten endringen medfører og ekstraarbeid som mulige årsaker til motstand.
 
\noindent
Motstand synliggjøres hovedsakelig gjennom motstandernes adferd, og \citet{Lapointe05} klassifiserer motstand i fire nivåer basert på dette.
 
\begin{itemize}
\item Apati inkluderer passivitet, mangel på interesse og distanse fra endringen.
\item Passiv motstand inkluderer forsinkelser, unnskyldninger og lite villighet for endring i arbeidsmåter.
\item Aktiv motstand tar blant annet i bruk ytringer av opposisjonerende meninger, dannelse av koalisjoner og delvis eller total nekt av bruk av systemet.
\item Aggressiv motstand kan innebære intern strid, trusler, streik, boikott og sabotasje, og søker å være forstyrrende eller destruktiv.
\end{itemize}
 
\noindent
Det vanlige fokuset i forskning på menneske-maskin interaksjon er brukere av systemet, og i tilfeller hvor ikke-brukere har vært av interesse, er disse gjerne identifisert som potensielle brukere. \citet{Satchell09} understreker at dette ikke gjelder alle, og de trekker frem misnøye med systemet, uttrykt gjennom kun delvis bruk av dette, og aktiv motstand som former for ikke-bruk.
 
\noindent
Motstand oppfattes ofte som utelukkende negativt, og de som motsetter seg en endring sees ofte på som gammeldags \citep{Jacobsen12}. Brukerne blir av mange delt inn i gode og  dårlige brukere, henholdsvis de som adopterer og bruker systemet slik det var tenkt, og de som ikke omfavner systemet \citep{Satchell09}. Dette skjuler det faktum at motstand i mange tilfeller bør sees som noe positivt, i form av kritiske innvendinger til behovet for endring og valg av løsning. Fravær av motstand \textit{kan} bety at alle er enige og går helhjertet inn for den nye løsningen, men det kan også bety at de ansatte er uinteressert i hvordan det går med organisasjonen \citep{Jacobsen12}. På samme måte er ikke ikke-bruk fravær av noe eller et tomrom, men ofte heller aktivt, meningsfullt, motivert, overveid, strukturert og produktivt \citep{Satchell09}
 
