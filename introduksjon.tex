\chapter{Introduksjon}
\label{chp:introduksjon}

\textit{St. Olavs Hospital var per 1. januar 2014 universitetssykehus for 702 869 innbyggere i Midt-Norge, og lokalsykehus i fylket, med vel 300 000 innbyggere. I 2012 hadde sykehuset 131 547 innlagte pasienter og gjennomførte 554 083 polikliniske konsultasjoner \citep{stolavs}. Sykehuset hadde i 2013 som mål å legge til rette for innovasjon, og å øke $"$... implementering av nye produkter, [...] og løsninger som bidrar til økt kvalitet, effektivitet... $"$ \citep{styring13}.}

\noindent
Ved utbyggingen av nye St. Olavs Hospital ble den leverte IKT-løsningen betegnet som Norges dyreste og mest kompliserte IKT-prosjekt. Infrastrukturen som ble implementert inneholdt blant annet et fast og et trådløst pasientsignalsystem \citep{TU}. Pasientsignaler utløses av pasienter ved behov for assistanse, og leveres til sykepleiere gjennom varsling på telefon og/eller veggpaneler. Denne studien omhandler sykepleiernes bruk av dette systemet.  

\section{Bakgrunn}
Forskerne så i sin prosjektoppgave \citep{Sund13} på hvordan sykepleiere bruker informasjon om kollegers aktiviteter og tilgjengelighet, og hvordan slik informasjon kan distribueres på en hensiktsmessig måte. I tillegg så forskerne på hvordan systemet kan endres for å begrense negative effekter ved avbrytelser. \citet{KlemetsRedundancy} undersøker ansvarsfordeling og redundans av kunnskap blant sykepleiere, og hvordan dette kan støttes av et trådløst system. De peker på at konsekvensen av å innføre et slikt system i et allerede avbruddsdrevet miljø, er en økning i antall forstyrrende avbrytelser. De foreslår mulige endringer for å minimere disse forstyrrelsene, blant annet å gi sykepleierne mulighet til å $"$gå av systemet$"$, og dermed ikke motta signaler. \citet{klemets13} ser på sykepleieres avgjørelsesprosess ved innkommende pasientsignal, og konkluderer med at deres håndtering påvirkes av hvilken situasjon de befinner seg i, og deres relasjon til pasienten som har utløst signalet. Da det i flere tilfeller er utfordrende å bruke telefonen som kilde til informasjon, foreslås det blant annet å vise informasjonen på en lettere tilgjengelig personlig enhet, samt å benytte pasientterminalen for å besvare signalet.

\noindent
Masteroppgaven \citep{Sletten09} er skrevet kort tid etter at det trådløse pasientsignalsystemet ble innført ved St.Olavs Hospital. Allerede her etterlyses en funksjon for å sette telefonen på $"$pause$"$ i situasjoner hvor sykepleierne ikke ønsker å bli forsyrret. \citet{Rygh13} ser på forholdet pleiere og pasienter har til det trådløse pasientsignalsystemet og hvilke muligheter som ligger i denne teknologien med tanke på kontinuitet i pleie . Også her foreslås funksjonalitet for å sette telefonene i pausemodus for å redusere antall forstyrrelser. Både \citep{Rygh13} og \citep{Selseth12} foreslår funksjonalitet for å differensiere pasientsignaler med utgangspunkt i pasientens behov. \citet{Selseth12} ser også på muligheten for at pasienter kan sende en slags tekstmelding til sykepleier om sine behov.

\section{Hensikt og forskningsspørsmål}
Med bakgrunn i forskernes prosjektoppgave \citep{Sund13}, var utgangspunktet for dette arbeidet å videre undersøke hvordan systemet kan endres for å bedre møte sykepleiernes behov. Det ble derimot tidlig avdekket tydelige ulikheter i sykepleiernes anvendelse av systemet, og forskerne valgte å vinkle forskningsspørsmålene annerledes. Motivasjonen for forskningsarbeidet ble dermed å kartlegge sykepleiernes anvendelse av pasientsignalsystemet, og å identifisere ulikheter i bruk. Videre ønsket forskerne å svare på hva som kan være årsaker til disse forskjellene. Dette resulterte i to forskningsspørsmål:

\begin{enumerate}
\item Hvordan brukes pasientsignalsystemet ved St.Olavs Hospital forskjellig i, og mellom ulike avdelinger? 
\item Hvilke faktorer kan være årsak til disse forskjellene?
\end{enumerate}

\noindent
For å besvare disse spørsmålene gjennomførte forskerne observasjoner og intervjuer ved tre avdelinger ved St.Olavs Hospital. Dataene fra innsamlingen ble deretter analysert, og relevant teori ble valgt i tråd med stegvis-deduktiv induktiv metode. Forskningsmetodene som ble anvendt blir belyst i kapittel \ref{chp:metode}.

\section{Avgrensning}
Denne oppgaven er avgrenset til å omhandle sykepleieres bruk av pasientsignalsystemet. Forskerne har valgt å ikke fokusere på underliggende tekniske arkitekturer eller økonomiske aspekter. Forskerne har ikke vært i direkte kontakt med pasienter, og har dermed ikke generert egne data som kan si noe om hvordan sykepleiernes bruk av pasientsignalsystemet påvirker pasientene. 

\section{Disposisjon}
Oppgaven er strukturert som følger:
\begin{itemize}
\item Kapittel \ref{chp:case} gir et innblikk i dagens situasjon og en mer detaljert forklaring av pasientsignalsystemet.
\item Kapittel \ref{chp:teori} utdyper teori og tidligere arbeid av relevans til forskningsområdet. I tillegg presenteres rammeverket benyttet for analyse av funnene.
\item Kapittel \ref{chp:metode} belyser oppgavens hensikt og de metoder som er brukt for datainnsamling og analyse.
\item Kapittel \ref{chp:resultater} oppsummerer avdekkede funn.
\item Kapittel \ref{chp:diskusjon} drøfter resultatene fra kapittel \ref{chp:resultater} i lys av relevant teori fra kapittel \ref{chp:teori}.
\item Kapittel \ref{chp:konklusjon} oppsummerer og konkluderer oppgaven, og foreslår mulig fremtidig arbeid
\end{itemize}

