\section{Sosioteknisk tilnærming}
\label{section:sosioteknisk}

Ikke alle informasjons- og kommunikasjonsteknologiske (IKT) systemer som introduseres i helseomsorgen er suksessfulle, og ifølge \citet{FITT} er det estimert at opp til 70\% av alle slike prosjekter feiler \citep{Coiera07}. Tradisjonelt sett er teknologien blitt designet først og menneskene deretter tilpasset denne \citep{Appelbaum97}.
I dag er imidlertid teknologien i stor grad integrert i det sosiale domenet, noe en må ta hensyn til ved design av tekniske systemer. Mennesker, vektøy og teknologi danner tilsammen det som kalles et sosioteknisk system \citep{Coiera04}.
En sosioteknisk tilnærming til IKT-systemer forsøker å forstå hvordan mellommenneskelige aspekter og tekniske systemer påvirker hverandre \citep{Coiera04}, og hvordan interaksjonen mellom mennesker begrenser eller former interaksjonen mellom mennesker og teknologi \citep{Coiera07}. \citet{Ackerman00} presenterer begrepet \textit{”sosioteknisk gap”}, og beskriver dette som det store skillet mellom hva som må støttes sosialt, og hva som kan støttes teknisk. Han argumenterer for at den sentrale utfordringen innen CSCW er å utforske, forstå og minke dette gapet. \citet{Coiera04} poengterer at oppførselen til et sosioteknisk system fremkommer gjennom interaksjonen mellom komponentene, og flere komponenter vil dermed gjøre det vanskeligere å forutse utfallet av tilsynelatende enkle endringer.

\noindent
Som beskrevet vil sosiale og tekniske systemer påvirke hverandre \citep{Coiera04}. Eksempler på at teknologien påvirker sosiale systemer blir synlig ved at innføring av teknologi i en ny setting ikke bare påvirke brukerne den er ment for, men også menneskene brukerne omgås. Eksempler på det motsatte, at sosiale systemer påvirker teknologien, er tydelig i tendensen vi ser i at mennesker i stadig større grad behandler datamaskiner og kommunikasjonsmedier som om det var mennesker, og en naturlig del av et sosialt system. I tillegg påpeker \citet{Coiera07} at brukernes forhold til teknologien i et slikt system vil bli tydelig påvirket av hva som skjer i det sosiale domenet. For det første vil villigheten til å bruke systemet avhenge av holdningen andre mennesker har ovenfor systemet. For det andre vil avgjørelsen om hvor mye kognitiv kapasitet vi til enhver tid tildeler teknologien være bestemt av hvem vi samhandler med på det sosiale planet på samme tid.

\noindent
Et annet perspektiv som trekkes frem av \citet{Coiera07} er mennerskers evne til å utvikle \emph{$"$workarounds$"$}, hvor en får teknologien til å oppføre seg på måter, og i settinger den ikke i utgangspunktet var ment for. Videre påpeker han at disse tilpasningene kan sees på som  implisitte signaler om et behov som ikke er dekket, og at å studere disse er en effektiv måte å identifisere styrker og svakheter ved eksisterende systemer og prosesser, og dermed forstå hva som må endres. 

\noindent
Workarounds defineres av \citet{Kobayashi05} som \emph{"informal temporary practices for handling exceptions to normal workflow$"$}. Oversatt til norsk betyr det $"$uformelle, midlertidige løsninger for å håndtere avvik fra normal arbeidsflyt$"$.
Workarounds kan være nødvendig når det oppstår akutte situasjoner hvor man ikke har nødvendige ressurser tilgjengelig, eller de kan oppstå som følge av sperrer i et system. Disse sperrene kan være tilsiktede, eller utilsiktede. \citet{Vogelsmeier08} beskriver workarounds som førstegrads problemløsing i den forstand at man lager mekanismer for å jobbe rundt problemer, uten å forsøke å løse den underliggende årsaken til at problemet oppsto. Dersom workarounds oppstår som konsekvens av at systemet er for rigid i forhold til sykepleierenes arbeidsmønster slik at systemet ikke støtter opp om arbeidet på en tilfredstillende måte, er dette svært uheldig. Dette kan i verste fall føre til livstruende situasjoner.

\noindent
En sosioteknisk tilnærming er altså ikke en liste med kriterier som må oppfylles for å sikre et suksessfullt system, men en metode som understreker at empirisk, kvalitativ innsikt og forståelse for den allerede eksisterende settingen teknologien skal passe inn i, bør være utgangspunktet for utvikling av IKT-systemer \citep{Berg99}. En slik tilnærming, med den ekstra dimensjonen som oppstår når andre mennesker konkurrerer om oppmerksomheten til et individ samtidig som det samhandler med teknologien, må ikke forveksles med teori om menneske-maskin interaksjon, som studerer hvordan individer jobber, og prossesserer informasjon når de ikke forstyrres av andre \citep{Coiera07}.

\noindent
Både \citet{Coiera07} og \citet{Berg99} trekker frem systemer i helseomsorgen som spesielt egnet for sosioteknisk analyse på grunn av sin svært komplekse natur. I følge \citet{Berg99} vil en sosioteknisk tilnærming være kritisk til systemer som forsøker å ta avsand fra den nødvendig rotete og $"$ad hoc$"$ måten helsearbeidere jobber på, gjennom IT-systemer med stor grad av standardisering og rasjonalisering av oppgave og arbeidsflyt.






