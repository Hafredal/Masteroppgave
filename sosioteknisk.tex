\section{Sosioteknisk tilnærming}
\label{sec:sosioteknisk}
Tradisjonelt sett er teknologi blitt designet først og menneskene deretter tilpasset denne \citep{Appelbaum97}. I dag må det ved design av tekniske systemer tas hensyn til at teknologien i stor grad er integrert i det sosiale domenet. En sosioteknisk tilnærming til IKT-systemer forsøker å forstå hvordan mellommenneskelige aspekter og tekniske systemer påvirker hverandre \citep{Coiera04}, og hvordan interaksjonen mellom mennesker begrenser eller former interaksjonen mellom mennesker og teknologi \citep{Coiera07}. \citet{Ackerman00} presenterer begrepet \textit{”sosioteknisk gap”}, og beskriver dette som det store skillet mellom de sosiale krav som stilles til teknologien og hva som lar seg løse teknisk. Han argumenterer for at den sentrale utfordringen innen CSCW er å utforske, forstå og redusere dette gapet. 

\noindent
Som beskrevet vil sosiale og tekniske systemer påvirke hverandre \citep{Coiera04}. Eksempler på at teknologien påvirker sosiale systemer blir synlig ved at innføring av teknologi i en ny setting ikke bare påvirker brukerne den er ment for, men også menneskene brukerne omgås. Eksempler på at sosiale systemer påvirker teknologien er tydelig i tendensen til at mennesker i stadig større grad behandler datamaskiner og kommunikasjonsmedier som om de var en naturlig del av et sosialt system. I tillegg påpeker \citet{Coiera07} at brukernes forhold til teknologien i et slikt system vil bli tydelig påvirket av hva som skjer i det sosiale domenet. For det første vil villigheten til å bruke systemet avhenge av holdningen andre mennesker har ovenfor systemet. For det andre vil avgjørelsen om hvor mye kognitiv kapasitet vi til enhver tid tildeler teknologien være bestemt av hvem vi samhandler med på det sosiale planet på samme tid.

\noindent
Tekniske systemer i helseomsorgen trekkes frem som spesielt egnet for sosioteknisk analyse på grunn av sin svært komplekse natur \citep{Coiera07, Berg99}.
En slik tilnærming, med den ekstra dimensjonen som oppstår når andre mennesker konkurrerer om oppmerksomheten til et individ samtidig som det samhandler med teknologien, må ikke forveksles med teori om menneske-maskin interaksjon, som studerer hvordan individer jobber og prosesserer informasjon når de ikke forstyrres av andre \citep{Coiera07}.
