\section{Sosio-teknisk tilnærming}
\label{section:sosioteknisk}

Ikke alle informasjons- og kommunikasjonsteknologiske (IKT) systemer som introduseres i helseomsorgen er suksessfulle, og i følge \citet{FITT} er det estimert at opp til 70\% av alle slike prosjekter feiler \citep{Coiera07}. Tradisjonelt sett er teknologien blitt designet først og menneskene deretter tilpasset denne \citep{Appelbaum97}.
Teknologien i dag er deromot i stor grad intigrert i det sosiale domenet noe vi må ta hensyn til i design av tekniske systemer. Mennesker, vektøy og teknologi danner tilsammen det vi kaller et sosioteknisk system \citep{Coiera04}.
En sosioteknisk tilnærming til IKT-systemer forsøker å forstå hvordan mellommenneskelige aspekter og tekniske systemer påvirker hverandre \citep{Coiera04}, og hvordan interaksjonen mellom mennesker begrenser eller former interaksjonen mellom mennesker og teknologi \citep{Coiera07}. \citet{Coiera04} poengterer at oppførselen til et sosioteknisk system fremkommer gjennom interaksjonen mellom komponentene, og flere komponenter vil dermed gjøre det vanskeligere å forutse utfallet av tilsynelatende enkle endringer.

\noindent
Som sagt vil sosiale og tekniske systemer påvirke hverandre \citep{Coiera04}. Eksempler på at teknologien påvirker sosliale systemer ser vi blandt annet i at introduksjon av teknologi i en ny setting ikke bare påvirke brukerne den er ment for, men også menneskene brukerne omgås.
Eksempler på det motsatte, at sosiale systemer påvirker teknologien, er tydelig i tendensen vi ser til at mennesker i stadig større grad behandler datamaskiner og kommunikasjonsmedier som om det var mennesker, og en naturlig del av et sosialt system. I tillegg påpeker \citet{Coiera07} at brukerenes forhold til teknologien i et slikt system vil bli tydelig påvirket av hva som skjer i det sosiale domenet. For det første vil villigheten til å bruke systemet avhenge av holdningen andre mennesker har ovenfor systemet. For det andre vil avgjørelsen om hvor mye kogvitiv kapasitet vi til en hver tid tildeler teknologien være bestemt av hvem vi samhandler med på det sosiale planet på samme tid.

\noindent
Et annet perspektiv som trekkes frem av \citet{Coiera07} er mennerskers evne til å utvikle \emph{$"$workarounds$"$}, hvor en får teknologien til å oppføre seg på måter, og i settinger den ikke i utgangspunktet var ment. Videre påpeker han at disse tilpassningene kan sees på som  implisitte signaler om et behov som ikke er dekket, og at å studere dem er en effektiv måte å identifisere styrker og svakheter ved eksisterende systemer og prosesser, og dermed forstå hva som må endres. 

\noindent
En sosioteknisk tilnærming er altså ikke en liste med kriterier som må oppfylles for å sikre et suksessfullt system, men en metode som understreker at empirisk, kvalitativ innsikt og forståelse for den allerede eksisterende settingen teknologien skal passe inn i, bør være utgangspunktet for utvikling av IKT-systemer \citep{Berg99}. En slik tilnærming, med den ekstra dimensjonen som oppstår når andre mennesker konkurrerer om oppmerksomheten til et individ samtidig som det samhandler med teknologien, må ikke forveksles med teori om menneske-maskin interaksjon, som studerer hvordan individer jobber, og prossesserer informasjon når de ikke forstyrres av andre \citep{Coiera07}.

\noindent
Både \citet{Coiera07} og \citet{Berg99} trekker frem systemer i helseomsorgen som spesielt egnet for sosioteknisk analyse på grunn av sin svært komplekse natur. I følge \citet{Berg99} vil en sosioteknisk tilnærming være kritisk til systemer som forsøker å ta avsand fra den nødvendig rotete og $"$ad hoc$"$ måten helsearbeidere jobber på, gjennom IT-systemer med stor grad av standardisering og rasjonalisering av oppgave og arbeidsflyt.











\noindent
[DE TRE KARAKTERISTIKKENE TIL BERG99 (2.1, 2.2, 2.3). FORTSETTER PÅ DETTE OM VI SER AT DET ER VERDT Å HA MED.]
\citet{Berg99} trekker frem tre karakteristikker ved en sosioteknisk tilnærming til IT-systemer i helseomsorgen. For det første, i sosioteknisk tilnærming er arbeidspraksiser konseptualisert som nettverk av mennesker, verktøy, rutiner, dokumenter osv. Grunnet den tette sammenknyttningen av elementene i dette nettverket vil introduksjonen av et nytt element (som et IT-system) ofte gi betydlige følger for arbeidspraksisen. 






