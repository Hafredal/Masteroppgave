\section{Sosio-teknisk tilnærming}
\label{section:sosioteknisk}

Ikke alle informasjons- og kommunikasjonsteknologiske (IKT) systemer som introduseres i helseomsorgen er suksessfulle, og i følge \citet{FITT} er det estimert at opp til 70\% av alle slike prosjekter feiler.
Sosioteknisk systemanalyse gir oss et solid rammeverk for å forstå grunnene til at dette skjer, og fokuserer på hvordan interaksjonen mellom mennesker begrenser eller former interaksjonen mellom mennesker og teknologi \citep{Coiera07}. 

\noindent
En sosioteknisk tilnærming til IKT-systemer forsøker å forstå hvordan mellommenneskelige aspekter og  tekniske systemer påvirker hverandre. I et sosioteknisk system vil brukerenes forhold til teknologien bli tydelig påvirket av hva som skjer i det sosiale domenet. Et eksempel er at villigheten til å bruke systemet vil avhenge av innstillingen andre mennesker har til systemet. På samme måte vil avgjørelsen på hvor mye kogvitiv kapasitet vi til en hver tid tildeler teknologien være bestemt av hvem vi samhandler med på det sosiale nivået på samme tid. Dette kan hjelpe oss å forstå hvordan en felles optimalisering av disse aspektene kan oppnås \citep{Coiera07}.

\noindent
Et annet perspektiv som trekkes frem av \citet{Coiera07} er mennerskers evne til å utvikle \emph{$"$workarounds$"$}, hvor en får teknologien til å oppføre seg på måter, og i settinger den ikke var ment. Videre påpeker han at disse tilpassningene kan sees på som  implisitte signaler om et behov som ikke er dekket, og at å studere dem er en effektiv måte å identifisere styrker og svakheter ved eksisterende systemer og prosesser, og dermed hva som må endres. 

\noindent
En sosioteknisk tilnærming er altså ikke en liste med kriterier som må oppfylles for å sikre et suksessfullt system, men heller en metode som understreker at empirisk, kvalitativ innsikt og forståelse for den allerede eksisterende settingen det skal passe inn i, bør være utgangspunktet for utvikling av slike systemer \citep{Berg99}. En slik tilnærming, med den ekstra dimensjonen som oppstår når andre mennesker konkurrerer om oppmerksomheten til et individ samtidig som det samhandler med teknologien, må ikke forveksles med teori om menneske-maskin interaksjon, som studerer hvordan individer jobber og prossesserer informasjon når de er uforstyrret \citep{Coiera07}.

\noindent
Både \citet{Coiera07} og \citet{Berg99} trekker frem helseomsorg-systemer som spesielt egnet for sosioteknisk analyse på grunn av sin svært komplekse natur. I følge \citet{Berg99} vil en sosioteknisk tilnærming her være kritisk til systemer som forsøker å ta avsand fra den nødvendig rotete og $"$ad hoc$"$ måten helsearbeidere jobber på gjennom IT-systemer med stor grad av standardisering og rasjonalisering av oppgave og arbeidsflyt.

\noindent
[DE TRE KARAKTERISTIKKENE TIL BERG99 (2.1, 2.2, 2,3). FORTSETTER PÅ DETTE OM VI SER AT DET ER VERDT Å HA MED.]
\citet{Berg99} trekker frem tre karakteristikker ved en sosioteknisk tilnærming til IT-systemer i helseomsorgen. For det første, i sosioteknisk tilnærming er arbeidspraksiser konseptualisert som nettverk av mennesker, verktøy, rutiner, dokumenter osv. Grunnet den tette sammenknyttningen av elementene i dette nettverket vil introduksjonen av et nytt element (som et IT-system) ofte gi betydlige følger for arbeidspraksisen. 






