\renewcommand{\abstractname}{Forord}
\begin{abstract}
\noindent
Denne masterbesvarelsen er skrevet ved Institutt for Telematikk (ITEM) ved Norges Teknisk-Naturvitenskapelige Universitet (NTNU), våren 2014. Forfatterne har fulgt studieprogrammet Kommunikasjonteknologi innen retningen Nett og Tjenester, med fordypning innen Telematikk og Samfunn. 

\noindent
Bakgrunnsinformasjon og -data er skaffet gjennom studier av tidligere arbeid og opplæringsmateriell tigjengelig for sykepleierne ved sykehuset. Det er også gjennomført observasjoner og intervjuer ved tre avdelinger ved sykehuset. 

\noindent
Vi vil først og fremst takke professor Pieter Touissant (IDI), som stilte opp som ansvarlig professor da Lill Kristiansen (ITEM) ble nødt til å trekke seg. Vi ønsker også å rette en stor takk til våre veiledere Ph.D. kandidatene Joakim Klemets og Katrien De Moor ved ITEM, for gode tilbakemeldinger, konstruktiv kritikk og støtte underveis i arbeidet.

\noindent
En spesiell takk går til involverte sykepleiere og seksjonsledere ved St. Olavs Hospital, som har tatt oss godt imot, lagt til rette for våre observasjoner, og stilt opp til intervju. 
Vi vil også takke Ivar Myrstad, IKT-rådgiver ved St. Olavs Hospital, for utfyllende informasjon om det nye sykehuset og pasientsignalsystemet.
En stor takk går til alle som har brukt tid på å hjelpe oss å lese korrektur på oppgaven.


\begin{center}
Til sist vil vi takke hverandre for et godt samarbeid og god stemning på kontoret. 
\end{center}


\centering

Trondheim, 10. juni 2014\\
Veronica Sund\\
Monika Hafredal
\end{abstract}