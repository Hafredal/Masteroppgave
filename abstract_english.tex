\pagestyle{empty}
\begin{abstract}
\noindent
When the new St. Olav's Hospital in Trondheim was constructed, the delivered ICT solution was referred to as Norway's most expensive and most complex ICT project. The implemented infrastructure includes a fixed and a wireless system that make up the nurse call system. Patient signals are triggered by patients in need of assistance, and are delivered to nurses through notification by telephone and/or wall panels. 

\noindent
Through initial observations differences in the nurses' use of the nurse call system became apparent. The system is used by many users in different contexts, and with different needs. It has therefore been made local adaptations of the system which differs from intended use. The motivation for the research has therefore been to identify these variations and to identify reasons for why they occur, with a socio-technical approach. Such an approach to ICT systems try to understand how interpersonal aspects and technical systems interact. 

\noindent
The findings are based on qualitative methods, where three departments at the hospital were observed and nurses at these departments were interviewed. The findings revealed some main reasons that imply that there is a lack of fit between nurses' work practices and the technical system.

\end{abstract}