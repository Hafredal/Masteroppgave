\section{Etnografi}
\label{section:etnografi} 

Etnografi er en kvalitativ metodikk som innebærer å studere en gruppe mennesker i en gitt setting, og er en hensiktsmessig tilnærming for å forstå deres handlinger og oppfatninger sett fra informantenes perspektiv \citep{Blomberg93, Reeves08, Nardi97}. Som en mulig følge av fremveksten av CSCW på 1980-tallet, oppsto samtidig en større interesse for å utforske anvendelsen av etnografiske metoder for å forstå gruppers arbeidsaktiviteter, og metodene har blitt stadig mer populære innen HCI-feltet \citep{Blomberg93, Millen00}. Tradisjonell etnografisk forskning innebærer ofte et bredt forskningsområde, hvor feltarbeidet utføres over en lengre tidsperiode. I prosjekter med begrenset tid for gjennomføring, vil det derimot være nyttig å anvende feltmetoder som kjennetegner \textit{hurtig etnografi} \citep{Millen00}. Dette innebærer å avgrense forskningsområdet, identifisere nøkkelinformanter, bruke interaktive observasjonsteknikker, og å utføre en kollektiv analyse av datamaterialet . Da arbeidet i dette prosjektet var begrenset av både tid og tilgang til feltet, ble alle disse elementene inkludert i utførelsen av feltarbeidet.



