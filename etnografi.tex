\section{Etnografi}
\label{section:etnografi} 

Etnografi er en kvalitativ metodikk som innebærer å studere en gruppe mennesker i en gitt setting, med den hensikt å få innsikt i deres handlinger og oppfatninger. Etnografi er dermed en hensiktsmessig tilnærming for å avdekke og forstå en gruppes arbeidsaktiviteter \cite{Blomberg, bmj, Nardi}. Tradisjonell etnografisk forskning innebærer ofte et bredt forskningsområde, hvor feltarbeidet utføres over en lengre tidsperiode. I prosjekter med begrenset tid for gjennomføring, vil det derimot være nyttig å anvende feltmetoder som kjennetegner \textit{hurtig etnografi}. Dette innebærer å avgrense forskningsområdet, identifisere nøkkelinformanter, ha flere forskere på optimale tidsperioder i feltet, og å utføre en kollektiv analyse av datamaterialet \citep{Millen00}. Da arbeidet i dette prosjektet var begrenset av både tid og tilgang til feltet, ble alle disse elementene inkludert i utførelsen av feltarbeidet.
\noindent
Innen etnografi er observasjon og intervju to metoder for datagenerering.



