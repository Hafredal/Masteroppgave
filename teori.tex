\chapter{Teori}
\label{chp:teori} 
Dette kapittelet belyser teori og tidligere arbeid av relevans for forskningsområdet. Da forskerne har arbeidet etter SDI-metoden (utdypet i kapittel \ref{section:sdi}), har funnene i aller største grad avgjort hvilken teori det er fokusert på. Der teorien overlapper med teori fra forskernes prosjektoppgave \citep{Sund13} er teksten hentet derfra. 

\noindent
Siden midten av 80-tallet har det blitt forsket på hvordan datasystemer kan være til støtte for samarbeid og kommunikasjon mellom individer \citep{Rogers94}, hvordan mennesker samarbeider for å utføre arbeidsaktiviteter og hvordan teknologi kan være til støtte for dette \citep{Ellis91}. Dette tverrfaglige området kalles CSCW, en forkortelse for det engelske $"$Computer-Supported Cooperative Work$"$, som på norsk kan oversettes til $"$datastøttet samarbeid$"$. Dette defineres som PC-baserte systemer som støtter grupper av mennesker engasjert i en felles oppgave, eller med et felles mål, som gir et grensesnitt til det delte miljøet \citep{Ellis91}.

\noindent
Forutsetningen for slike systemer er at de brukes av et tilstrekkelig antall brukere \citep{Ackermann00}. Det ville eksempelvis vært meningsløst å bruke e-post dersom ingen, eller få andre brukte det. Et typisk CSCW-system vil dermed brukes av et bredt spekter av brukere, med ulik bakgrunn, erfaringer og forhold til bruk av informasjonsystemer, noe som gjør utviklingen av et slikt system svært kompleks \citep{Berg99}.
