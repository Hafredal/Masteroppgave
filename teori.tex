\chapter{Teori}
\label{chp:teori} 

Dette kapittelet belyser teori og tidligere arbeid av relevans for forskningsområdet. Da forskerne har arbeidet etter SDI-metoden (utdypet i kapittel \ref{section:kvalitativ_analyse}), har funnene i aller største grad avgjort hvilken teori det er fokusert på. Der teorien overlapper med teori fra forskernes prosjektoppgave \citep{Sund13} er teksten hentet derfra. 

\noindent
Siden midten av 80-tallet har det blitt forsket på hvordan datasystemer kan være til støtte for samarbeid og kommunikasjon mellom individer \citep{Rogers94}, hvordan mennesker samarbeider for å utføre arbeidsaktiviteter, og hvordan teknologi kan være til støtte for dette \citep{Ellis91}. Dette tverrfaglige området kalles CSCW, en forkortelse for det engelske $"$Computer-Supported Cooperative Work$"$, som på norsk kan oversettes til $"$datastøttet samarbeid$"$. Dette defineres som PC-baserte systemer som støtter grupper av mennesker engasjert i en felles oppgave, eller med et felles mål, som gir et grensesnitt til det delte miljøet \citep{Ellis91}.

\noindent
Et typisk CSCW-system vil vanligvis bli brukt av et bredt spekter av brukere, med ulik bakgrunn, erfaringer og forhold til bruk av informasjonsystemer, noe som gjør utviklingen av et slikt system svært kompleks. Desverre er det ikke uvanlig at beslutningstakere tar avgjørelser basert på hvilke funksjoner som vil være fordelaktig for brukere som dem selv, og dermed overser hva andre brukere kan ha nytte av. Funksjonalitet som letter arbeidet for én gruppe brukere, kan gi merarbeid for en annen. Det kan også være vanskelig å lære fra tidligere feil, da slike systemer er svært komplekse og unike for hvert enkelt tilfelle, noe som vanskeliggjør evaluering i ettertid. Det er i tillegg utfordrende å gjenskape miljø og forhold som er essensielle i den virkelige konteksten hvor systemet skal implementeres, i et laboratorium. Feltobservasjoner kan gi feilaktige inntrykk, da det kan forekomme variasjoner i gruppesammensetning og miljømessige faktorer \citep{Berg99}.

\noindent
Det å utvikle et CSCW-system for helseomsorgen vil dermed kunne være en utfordrende prosess. Det er konflikt mellom det flytende samarbeidet og de tilsynelatende uforutsette arbeidsoppgavene til sykepleiere, og den formelle, standardiserte og relativt stive funksjonaliteten til et informasjonsystem. En av forutsetningene for et suksessfullt system i et slikt miljø er derfor å ikke forsøke å erstatte denne 'rotetheten' med en rasjonalitet og strømlinjeform som ofte er vanlig for slike systemer. Verktøy som kun har forutbestemte sekvensielle trinn, eller som kun tillater gitte typer input vil derfor ikke fungere sammen med sykepleiernes arbeidsmåter, og som en følge av dette ikke overleve \citep{Berg99}.
