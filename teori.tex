\chapter{Teori}
\label{chp:teori} 
Dette kapittelet belyser teori av relevans for forskningsområdet. Da forskerne har arbeidet etter SDI-metoden (utdypet i kapittel \ref{section:sdi}), har funnene avgjort hvilken teori det er fokusert på. I dette kapittelet vil den sosiotekniske tilnærmingen gjøres rede for (\ref{sec:sosioteknisk}), og det vil bli presentert teori som kan forklare hvorfor avvik fra tenkt bruk oppstår (\ref{sec:implementering}). Da redundans er sentralt både i pasientsignalsystemets infrastruktur og pleiernes bruk av systemet ble det naturlig å inkludere dette (\ref{sec:redundans}). I tillegg vil det bli presentert teori om støy og avbrytelser da dette stadig trekkes frem som en sentral utfordring ved systemet (\ref{sec:avbrudd}). Der teorien overlapper med teori fra forskernes prosjektoppgave \citep{Sund13} er teksten hentet derfra. 


