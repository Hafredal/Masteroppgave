\chapter{Observasjonsplan for første observasjonsperiode}
\label{chp:appendix_observasjonsplan1}

\textbf{Hvorfor observasjon?}

\noindent
Vi ønsker å avdekke sykepleiernes reelle arbeidspraksis ved bruk av pasientsignalsystemet, og hvordan og når denne interaksjonen skjer. 

\noindent
\textbf{Rolle}

\noindent
Vår rolle vil være interaktiv observatør, hvor vi vil være tilstede på sengetun ca 2 timer per observasjon.

\noindent
\textbf{Fokus}

\noindent
Vårt fokus er å avdekke sykepleiernes arbeidspraksis ved bruk av pasientsignalsystemet. I situasjoner hvor det ikke vil være forstyrrende, vil vi stille spørsmål for å utdype og  bedre forstå sykepleiernes handlinger .

\begin{itemize}
	\item Hvordan interagerer sykepleierne med systemet?
	\item Hvordan håndterer sykepleierne pasientsignaler?
		\begin{itemize}
			\item Aksepterer eller avviser sykepleieren pasientsignalet?
			\item Hvilke faktorer ligger til grunn for avgjørelsen?
			\item Hvilket forhold har sykepleieren til pasienten som har utløst signalet?
			\item Hvilken kontekst befinner sykepleieren seg i, og hvordan påvirker dette håndteringen av signalet?
			\item Er sykepleieren bevisst på sine kollegers tilgjengelighet/aktiviteter?
		\end{itemize}
\end{itemize}

\noindent
\textbf{Oppfølgingsspørsmål}

\noindent
«Hvorfor valgte du å godta/avvise pasientsignalet?»

\noindent
«Hvilken informasjon bruker du i din avgjørelse? Hva vektlegger du?»

\noindent
«Gir systemet tilstrekkelig informasjon?»

\noindent
«Er det annen informasjon som kunne hjulpet deg i din avgjørelse?»

\noindent
«Hadde du gjort deg noen tanker om hva signalet gjaldt før du gikk inn på rommet? I såfall var dine antagelser riktige?»

\noindent
«Er det stor variasjon fra dag til dag i hvor mange signaler du mottar? Er denne dagen «representativ»?»